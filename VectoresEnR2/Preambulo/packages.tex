\documentclass[11pt,a4paper]{beamer}
\usetheme{CambridgeUS}
\usecolortheme{dolphin}
\usepackage[spanish]{babel}				% Palabras reservadas traducidas al español 
\usepackage[utf8]{inputenc}				% Para poder imprimir caracteres especiales
\usepackage{fontenc}					% Para acentos
\usepackage{nameref}
\usepackage{translator}
%%%%%%%%%%%%%%%%%%%%%%
% Fuentes, funciones y teoremas matemáticos
\usepackage{amsmath}					% Carga los entornos matemáticos
\usepackage{amsfonts}					% Carga las fuentes matemáticas
\usepackage{amssymb}					% Carga caracteres matemáticos especiales
\usepackage{amsthm}						% Crea entornos de teoremas, definiciones, etc.
\usepackage{mdframed}					% Sirve para encerrar los teoremas de amsthm en cajas para darle un estilo más personalizado
\usepackage{mathtools}					% Extensión de símbolos matemáticos
\usepackage{cases}						% Para crear el entorno de definición de una expresión por casos
\usepackage{cancel}						% Permite tachar expresiones matemáticas			
\usepackage{leftindex}					% Para poder añadir subíndices y superíndices del lado izquierdo de una expresión
\usepackage[italic]{derivative}					% Extensión de las funciones de derivadas
\usepackage[intlimits]{esint}			% Extensión de símbolos de integrales
\usepackage{thmtools}
\spanishdecimal{.}
%%%%%%%%%%%%%%%%%%%%%%

\usepackage{listings}					% Para ingresar código fuente en diferentes lenguajes de programación
\usepackage{pifont}						% Inserta doodles

%%%%%%%%%%%%%%%%%%%%%%
% Para insertar pseudocódigo
\usepackage{algorithm}
\usepackage{algorithmic}
%%%%%%%%%%%%%%%%%%%%%%

\usepackage{xcolor}						% Permite modificar y definir colores
\usepackage{multicol}					% Permite crear múltiples columnas de texto en el documento
\usepackage{hyperref}
%%%%%%%%%%%%%%%%%%%%%%
% Para insertar y crear figuras
\usepackage{graphicx}					% Para insertar gráficos
\usepackage{tikz}						% Para crear figuras
\usepackage{venndiagram}				% Extensión de Tikz para crear diagramas de Venn
%%%%%%%%%%%%%%%%%%%%%%

\usepackage[many]{tcolorbox}			% Para poder crear cajas de color
\tcbuselibrary{theorems}					
\usepackage{enumerate}					% Para poder modificar el estilo de los items en el entorno enumerate
%%%%%%%%%%%%%%%%%%%%%%
\usepackage{ifthen}
\usepackage{xargs}