%%%%%%%%%%%%%%%%%%%%%%
% Definición de estilos para los teoremas
\mdfdefinestyle{morado}{
	linecolor = mypurple,						% Color de línea
	linewidth = 2pt,							% Ancho de la línea
	frametitlerule = true,						% Separar el título del contenido
	frametitlebackgroundcolor = mypurple!20,	% Color de fondo del título
	topline = false,							% No mostrar las líneas superiores e inferiores
	bottomline = false,
	rightline = false,
	innertopmargin = \topskip
}

\mdfdefinestyle{azul}{
	linecolor = myblue,
	linewidth = 2pt,
	frametitlerule = true,
	frametitlebackgroundcolor = myblue!20,
	topline = false,
	bottomline = false,
	rightline = false,
	innertopmargin = \topskip
}
\mdfdefinestyle{cian}{
	linecolor = mycyan,
	linewidth = 2pt,
	frametitlerule = true,
	frametitlebackgroundcolor = mycyan!20,
	topline = false,
	bottomline = false,
	rightline = false,
	innertopmargin = \topskip
}
\mdfdefinestyle{verde}{
	linecolor = mygreen,
	linewidth = 2pt,
	frametitlerule = true,
	frametitlebackgroundcolor = mygreen!20,
	topline = false,
	bottomline = false,
	rightline = false,
	innertopmargin = \topskip
}
\mdfdefinestyle{gris}{
	linecolor = gray,
	linewidth = 2pt,
	frametitlerule = true,
	frametitlebackgroundcolor = gray!20,
	topline = false,
	bottomline = false,
	rightline = false,
	innertopmargin = \topskip
}
\mdfdefinestyle{solution}{
	linecolor = gray,
	linewidth = 2pt,
	frametitlerule = false,
	topline = false,
	bottomline = false,
	rightline = false,
	leftline = true,
	innertopmargin = \topskip
}
% Definición de los entornos de teoremas
\mdtheorem[style = azul]{axiom}{Axioma}[section]
\mdtheorem[style = morado]{theorem}{Teorema}[section]
\mdtheorem[style = morado]{proposition}[theorem]{Proposición}
\mdtheorem[style = morado]{lemma}[theorem]{Lema}
\mdtheorem[style = cian]{definition}{Definición}[section]
\mdtheorem[style = verde]{coll}[theorem]{Corolario}
% Definición de estilos para cajas
\tcbset{
	breakable,
	enhanced,
	sharp corners,
	boxsep=1pt,
	attach boxed title to top left={yshift=-\tcboxedtitleheight,  yshifttext=-.75\baselineskip},
	boxed title style={boxsep=1pt,sharp corners},
	fonttitle=\bfseries\sffamily,
	%drop lifted shadow
}
% Definición de cajas
\newtcolorbox{remark}[1][]{
	title={\scalebox{1.75}{\raisebox{-.25ex}{\ding{43}}}~Observación},
	colframe=black!10!white,
	colback=black!10!white,
	coltitle=black,
	fontupper=\sffamily,
	boxed title style={colback=black!10!white},
	boxed title style={boxsep=1ex,sharp corners},%%
	overlay unbroken and first={ \node[below right,font=\normalsize,color=red,text width=.8\linewidth] at (title.north east) {#1};
	}
}
\newtcolorbox{note}[1][]{
	title={\scalebox{1.75}{\raisebox{-.25ex}{\ding{45}}}~Nota},
	colframe=black!10!white,
	colback=black!10!white,
	coltitle=black,
	fontupper=\sffamily,
	boxed title style={colback=black!10!white},
	boxed title style={boxsep=1ex,sharp corners},%%
	overlay unbroken and first={ \node[below right,font=\normalsize,color=red,text width=.8\linewidth] at (title.north east) {#1};
	}
}
\newtcolorbox{notation}[1][]{
title={\scalebox{1.75}{\raisebox{-.25ex}{\ding{46}}}~Notación},
colframe=black!10!white,
colback=black!10!white,
coltitle=black,
fontupper=\sffamily,
boxed title style={colback=black!10!white},
boxed title style={boxsep=1ex,sharp corners},%%
overlay unbroken and first={ \node[below right,font=\normalsize,color=red,text width=.8\linewidth] at (title.north east) {#1};
}
}
% Entornos auxiliares
\mdtheorem[style = azul, bottomline = true, topline = true, rightline = true]{exercise}{Ejercicios}[section]
\theoremstyle{remark}
\mdtheorem[style = gris, frametitlerule = false]{example}{Ejemplo}[section]
\surroundwithmdframed[style = solution]{proof}
\newenvironment{solution}{\emph{Soluci\'{o}n:}}{}
\surroundwithmdframed[style = solution]{solution}
%%%%%%%%%%%%%%%%%%%%%%
