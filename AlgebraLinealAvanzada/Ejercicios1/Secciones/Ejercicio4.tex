\begin{exercise}
	Encuentre bases para los 4 espacios fundamentales. Determine el rango y la nulidad de la matriz dada:
	\[
		A =
		\begin{bmatrix}
			1 & -1 & 2\\
			3 & 1 & 0
		\end{bmatrix}
	\]
\end{exercise}
\begin{solution}
	\begin{enumerate}[a)]
		\item Espacio columna\par 
		Sabemos que:
		\[ C_{A} =\mgen[1]{\set{
			\begin{bmatrix}
				1\\
				3
			\end{bmatrix},
			\begin{bmatrix}
				-1\\
				1
			\end{bmatrix},
			\begin{bmatrix}
				2\\
				0
			\end{bmatrix}
		}} \]
		Determinemos cuáles son los vectores linealmente independientes:
		\begin{align*}
			\begin{bmatrix}
				1 & -1 & 2\\
				3 & 1 & 0
			\end{bmatrix}
			\overset{R_{2} - 3R_{1}}{\to}
			\begin{bmatrix}
				1 & -1 & 2\\
				0 & 4 & -6
			\end{bmatrix}
			\overset{\frac{1}{2}R_{2}}{\to}
			\begin{bmatrix}
				1 & -1 & 2\\
				0 & 2 & -3
			\end{bmatrix}
			\overset{R_{1} +\frac{1}{2}R_{2}}{\to}
			\begin{bmatrix}
				1 & 0 & \frac{1}{2}\\
				0 & 2 & -3
			\end{bmatrix}
			\overset{\frac{1}{2}R_{2}}{\to}
			\begin{bmatrix}
				\CodeBefore
				\rectanglecolor{pgay4!50}{1-1}{2-2}
				\Body
				1 & 0 & \frac{1}{2}\\
				0 & 1 & -\frac{3}{2}
			\end{bmatrix}
		\end{align*}
		Entonces $\set{\begin{bsmallmatrix}
				1\\
				3
		\end{bsmallmatrix},
		\begin{bsmallmatrix}
			-1\\
			1
	\end{bsmallmatrix}}$ es L.I. y por tanto $\set{\begin{bsmallmatrix}
		1\\
		3
	\end{bsmallmatrix},
	\begin{bsmallmatrix}
		-1\\
		1
\end{bsmallmatrix}}$ es base de $C_{A}$. Además, se tiene que $\mdim{C_{A}} = 2$.
	\label{item:2}
	\item Espacio renglón\par 
	Sabemos que:
	\[ R_{A} =\mgen[1]{\set{
		\begin{bmatrix}
			1\\
			-1\\
			2
		\end{bmatrix},
		\begin{bmatrix}
			3\\
			1\\
			0
		\end{bmatrix}	
	}} \]
	Demostremos que $\beta =\set{
		\begin{bmatrix}
			1\\
			-1\\
			2
		\end{bmatrix},
		\begin{bmatrix}
			3\\
			1\\
			0
		\end{bmatrix}	
	}$ es L.I. Supongamos que $\beta$ no es L.I., entonces existiría $c\in\rmath$ tal que:
	\[
		\begin{bmatrix}
			1\\
			-1\\
			2
		\end{bmatrix}
		= c
		\begin{bmatrix}
			3\\
			1\\
			0
		\end{bmatrix}
		=
		\begin{bmatrix}
			3c\\
			1c\\
			0
		\end{bmatrix}
	\]
	de donde obtenemos que $2 = 0$, lo cual es una contradicción. Por lo tanto, $\beta$ es L.I. y por consiguiente base de $R_{A}$. Además, $\mdim{R_{A}} = 2$.
	\item Imagen\par 
	Sabemos que $\mim{A} = C_{A}$, entonces $\set{\begin{bsmallmatrix}
			1\\
			3
		\end{bsmallmatrix},
		\begin{bsmallmatrix}
			-1\\
			1
	\end{bsmallmatrix}}$ es base de $\mim{A}$ y además $\mrho{A} = 2$.
	\item Núcleo\par 
	Sabemos que $\mnuc{A} =\set{\hat{x}\in\rmath[3]: A\hat{x} = \hat{0}}$. Si $\hat{x} =\begin{bsmallmatrix}
		x\\
		y\\
		z
	\end{bsmallmatrix}$, entonces se tiene el sistema homogéneo:
	\[  
	\begin{bmatrix}
		1 & -1 & 2\\
		3 & 1 & 0
	\end{bmatrix}
	\begin{bmatrix}
		x\\
		y\\
		z
	\end{bmatrix}
	=
	\begin{bmatrix}
		0\\
		0
	\end{bmatrix}
	\]
	De la reducción del inciso (\ref{item:2}) se tiene que:
	\[ \begin{bmatrix}
		1 & 0 & \frac{1}{2}\\
		0 & 1 & -\frac{3}{2}
	\end{bmatrix}
	\begin{bmatrix}
		x\\
		y\\
		z
	\end{bmatrix}
	=
	\begin{bmatrix}
		0\\
		0
	\end{bmatrix}
	\implies
	\begin{bmatrix}
		x\\
		y\\
		z
	\end{bmatrix}
	=
	\begin{bmatrix}
		-\frac{1}{2}z\\
		\frac{3}{2}z\\
		z
	\end{bmatrix}
	= z
	\begin{bmatrix}
		-\frac{1}{2}\\
		\frac{3}{2}\\
		1
	\end{bmatrix}
	\]
	Por lo que $\mnuc{A} =\mgen[1]{\set{\begin{bsmallmatrix}
		-\frac{1}{2}\\
		\frac{3}{2}\\
		1	
	\end{bsmallmatrix}}} =\mgen[1]{\set{\begin{bsmallmatrix}
	-1\\
	3\\
	2	
\end{bsmallmatrix}}}$ y entonces $\set{\begin{bsmallmatrix}
-1\\
3\\
2	
\end{bsmallmatrix}}$ es base de $\mnuc{T}$. Además, $\mnu{A} = 2$.
	\end{enumerate} 
\end{solution}