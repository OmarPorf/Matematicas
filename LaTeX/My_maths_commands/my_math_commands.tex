\documentclass[10pt, a4paper]{article}
\usepackage[spanish]{babel}
%%%%%%%%%%%%%%%%%%%%%%
% Fuentes, funciones y teoremas matemáticos
\usepackage{amsmath}					% Carga los entornos matemáticos
\usepackage{amsfonts}					% Carga las fuentes matemáticas
\usepackage{amssymb}					% Carga caracteres matemáticos especiales
\usepackage{amsthm}						% Crea entornos de teoremas, definiciones, etc.
\usepackage{mdframed}					% Sirve para encerrar los teoremas de amsthm en cajas para darle un estilo más personalizado
\usepackage{mathtools}					% Extensión de símbolos matemáticos
\usepackage{cases}						% Para crear el entorno de definición de una expresión por casos
\usepackage{cancel}						% Permite tachar expresiones matemáticas			
\usepackage{leftindex}					% Para poder añadir subíndices y superíndices del lado izquierdo de una expresión
\usepackage[italic]{derivative}					% Extensión de las funciones de derivadas
%\usepackage[intlimits]{esint}			% Extensión de símbolos de integrales
\usepackage{thmtools}
%%%%%%%%%%%%%%%%%%%%%%
\usepackage[margin = 1in]{geometry}
%opening
\title{Comandos personalizados: matem\'{a}ticas}
\author{Omar Porfirio Garc\'{\i}a}

\usepackage{ifthen}
\usepackage{xargs}
%%%%%%%%%%%%%%%%%%%%%%%%%%%%%%%%%%%%%%%%%%%%%
\newcommand{\ds}{\displaystyle}
%%%%%%%%%%%%%%%%%%%%%%%%%%%%%%%%%%%%%%%%%%%%%
	% Conjuntos numéricos
\newcommand{\cmath}[1][]{\mathbb{C}^{#1}}
\newcommand{\rmath}[1][]{\mathbb{R}^{#1}}
\newcommand{\qmath}[1][]{\mathbb{Q}^{#1}}
\newcommand{\irrmath}[1][]{\mathbb{I}^{#1}}
\newcommand{\zmath}[1][]{\mathbb{Z}^{#1}}
\newcommand{\nmath}[1][]{\mathbb{N}^{#1}}
\newcommand{\pmath}[1][]{\mathbb{P}^{#1}}
\newcommand{\fmath}[1][]{\mathbb{F}^{#1}}
\newcommand{\hmath}[1][]{\mathbb{H}^{#1}}
\newcommand{\umath}[1][]{\mathbb{U}^{#1}}
%%%%%%%%%%%%%%%%%%%%%%%%%%%%%%%%%%%%%%%%%%%%%
	% Delimitadores
\newcommand{\pa}[1]{\left( #1 \right)}
\newcommand{\bracket}[1]{\left[ #1 \right]}
\newcommand{\set}[1]{\left\{ #1 \right\}}
\newcommand{\abs}[1]{\left\vert #1 \right\vert}
\newcommand{\norm}[1]{\left\Vert #1 \right\Vert}
\newcommand{\inprod}[1]{\left\langle #1 \right\rangle}
\newcommand{\floor}[1]{\left\lfloor #1 \right\rfloor}
\newcommand{\ceil}[1]{\left\lceil #1 \right\rceil}
\newcommand{\upcor}[1]{\left\ulcorner #1 \right\urcorner}
\newcommand{\lowcor}[1]{\left\llcorner #1 \right\lrcorner}
\newcommand{\lopen}[1]{\left( #1 \right]}
\newcommand{\ropen}[1]{\left[ #1 \right)}
\newcommand{\Lopen}[1]{\left] #1 \right]}
\newcommand{\Ropen}[1]{\left[ #1 \right[}
%%%%%%%%%%%%%%%%%%%%%%%%%%%%%%%%%%%%%%%%%%%%%
	% funciones estándar
\newcommandx{\minf}[3][1 = 0, 3 = { }, usedefault]{
	\ifthenelse{ \equal{#1}{0}}{
		\inf_{#3}\left( #2 \right)
	}{
		\inf_{#3}{#2} 
	}
}
\newcommandx{\msup}[3][1 = 0, 3 = { }, usedefault]{
	\ifthenelse{ \equal{#1}{0} }{
		\sup_{#3}\left( #2 \right)
	}{
		\sup_{#3}{#2}
	}
}
\newcommandx{\msin}[3][1 = 0, 3 = { }, usedefault]{
	\ifthenelse{ \equal{#1}{0} }{
		\sin^{#3}\left( #2 \right)
	}{
		\sin^{#3}{#2}
	}
}
\newcommandx{\mcos}[3][1 = 0, 3 = { }, usedefault]{
	\ifthenelse{ \equal{#1}{0} }{
		\cos^{#3}\left( #2 \right)
	}{
		\cos^{#3}{#2}
	}
}
\newcommandx{\mtan}[3][1 = 0, 3 = { }, usedefault]{
	\ifthenelse{ \equal{#1}{0} }{
		\tan^{#3}\left( #2 \right)
	}{
		\tan^{#3}{#2}
	}
}
\newcommandx{\msec}[3][1 = 0, 3 = { }]{
	\ifthenelse{ \equal{#1}{0} }{
		\sec^{#3}\left( #2 \right)
	}{
		\sec^{#3}{#2}
	}
}
\newcommandx{\mcsc}[3][1 = 0, 3 = { }]{
	\ifthenelse{ \equal{#1}{0} }{
		\csc^{#3}\left( #2 \right)
	}{
		\csc^{#3}{#2}
	}
}
\newcommandx{\mcot}[3][1 = 0, 3 = { }]{
	\ifthenelse{ \equal{#1}{0} }{
		\cot^{#3}\left( #2 \right)
	}{
		\cot^{#3}{#2}
	}
}
\newcommandx{\marcsin}[3][1 = 0, 3 = { }, usedefault]{
	\ifthenelse{ \equal{#1}{0} }{
		\arcsin^{#3}\left( #2 \right)
	}{
		\arcsin^{#3}{#2}
	}
}
\newcommandx{\marccos}[3][1 = 0, 3 = { }, usedefault]{
	\ifthenelse{ \equal{#1}{0} }{
		\arccos^{#3}\left( #2 \right)
	}{
		\arccos^{#3}{#2}
	}
}
\newcommandx{\marctan}[3][1 = 0, 3 = { }, usedefault]{
	\ifthenelse{ \equal{#1}{0} }{
		\arctan^{#3}\left( #2 \right)
	}{
		\arctan^{#3}{#2}
	}
}
\newcommandx{\marcsec}[3][1 = 0, 3 = { }, usedefault]{
	\ifthenelse{ \equal{#1}{0} }{
		\operatorname{arcsec}^{#3}\left( #2 \right)
	}{
		\operatorname{arcsec}^{#3}{#2}
	}
}
\newcommandx{\marccsc}[3][1 = 0, 3 = { }, usedefault]{
	\ifthenelse{ \equal{#1}{0} }{
		\operatorname{arccsc}^{#3}\left( #2 \right)
	}{
		\operatorname{arccsc}^{#3}{#2}
	}
}
\newcommandx{\marccot}[3][1 = 0, 3 = { }, usedefault]{
	\ifthenelse{ \equal{#1}{0} }{
		\operatorname{arccot}^{#3}\left( #2 \right)
	}{
		\operatorname{arccot}^{#3}{#2}
	}
}
\newcommand{\marg}[2][0]{
	\ifthenelse{ \equal{#1}{0} }{\arg\left( #2 \right)}{\arg{#2}}
}
\newcommandx{\mdeg}[3][1 = 0, 3 = { }]{
	\ifthenelse{ \equal{#1}{0}}{\deg_{#3}\left( #2 \right)}{\deg_{#3}{#2}}
}
\newcommandx{\mdet}[3][1 = 0, 3 = { }, usedefault]{
	\ifthenelse{ \equal{#1}{0} }{\det_{#3}\left( #2 \right)}{\det_{#3}{#2}}
}
\newcommand{\mdim}[2][0]{
	\ifthenelse{ \equal{#1}{0} }{\dim\left( #2 \right)}{\dim{#2}}
}
\newcommand{\mexp}[2][0]{
	\ifthenelse{ \equal{#1}{0} }{\exp\left( #2 \right)}{\exp{#2}}
}
\DeclareMathOperator*{\mcd}{mcd}
\newcommandx{\mmcd}[3][1 = 0, 3 = { }, usedefault]{
	\ifthenelse{ \equal{#1}{0} }{\mcd_{#3}\left( #2 \right)}{\mcd_{#3}{#2}}
}
\newcommand{\mln}[2][0]{
	\ifthenelse{ \equal{#1}{0} }{\ln\left( #2 \right)}{\ln{#2}}
}
\newcommandx{\mlog}[3][1 = 0, 3 = { }, usedefault]{
	\ifthenelse{ \equal{#1}{0} }{\log_{#3}\left( #2 \right)}{\log_{#3}{#2}}
}
\newcommandx{\mmax}[3][1 = 0, 3 = { }, usedefault]{
	\ifthenelse{ \equal{#1}{0} }{\max_{#3}\left( #2 \right)}{\max_{#3}{#2}}
}
\newcommandx{\mmin}[3][1 = 0, 3 = { }, usedefault]{
	\ifthenelse{ \equal{#1}{0} }{\min_{#3}\left( #2 \right)}{\min_{#3}{#2}}
}
%%%%%%%%%%%%%%%%%%%%%%%%%%%%%%%%%%%%%%%%%%%%%
	% Otras funciones
\newcommand{\mpow}[2][0]{
	\ifthenelse{ \equal{#1}{0} }{\mathfrak{P}\left( #2 \right)}{\mathfrak{P}{#2}}
}
\newcommand{\mnu}[2][0]{
	\ifthenelse{ \equal{#1}{0} }{\nu\left( #2 \right)}{\nu{#2}}
}
\newcommand{\mrho}[2][0]{
	\ifthenelse{ \equal{#1}{0} }{\rho\left( #2 \right)}{\rho{#2}}
}
\newcommand{\mnuc}[2][0]{
	\ifthenelse{ \equal{#1}{0} }{\operatorname{N}\left( #2 \right)}{\operatorname{N}{#2}}
}
\newcommand{\mgen}[2][0]{
	\ifthenelse{ \equal{#1}{0} }{\operatorname{gen}\left( #2 \right)}{\operatorname{gen}{#2}}
}
\newcommand{\mtr}[2][0]{
	\ifthenelse{ \equal{#1}{0} }{\operatorname{tr}\left( #2 \right)}{\operatorname{tr}{#2}}
}
\newcommand{\mdom}[2][0]{
	\ifthenelse{ \equal{#1}{0} }{\operatorname{Dom}\left( #2 \right)}{\operatorname{Dom}{#2}}
}
\newcommand{\mran}[2][0]{
	\ifthenelse{ \equal{#1}{0} }{\operatorname{Ran}\left( #2 \right)}{\operatorname{Ran}{#2}}
}
\newcommand{\mim}[2][0]{
	\ifthenelse{ \equal{#1}{0} }{\operatorname{Im}\left( #2 \right)}{\operatorname{Im}{#2}}
}
\newcommand{\ipart}[2][0]{
	\ifthenelse{ \equal{#1}{0} }{\Im\left( #2 \right)}{\Im{#2}}
}
\newcommand{\rpart}[2][0]{
	\ifthenelse{ \equal{#1}{0} }{\Re\left( #2 \right)}{\Re{#2}}
}
\newcommand{\mprob}[2][0]{
	\ifthenelse{ \equal{#1}{0} }{\mathbb{P}\left( #2 \right)}{\mathbb{P}{#2}}
}
\newcommand{\mmean}[2][0]{
	\ifthenelse{ \equal{#1}{0} }{\mathbb{E}\left( #2 \right)}{\mathbb{E}{#2}}
}
\newcommand{\mvar}[2][0]{
	\ifthenelse{ \equal{#1}{0} }{\mathbb{V}\left( #2 \right)}{\mathbb{V}{#2}}
}
\DeclareMathOperator*{\mcm}{mcm}
\newcommandx{\mmcm}[3][1 = 0, 3 = { }, usedefault]{
	\ifthenelse{ \equal{#1}{0} }{\mcm_{#3}\left( #2 \right)}{\mcm_{#3}{#2}}
}
\newcommandx{\mgrad}[3][1 = 0, 3 = { }]{
	\ifthenelse{ \equal{#1}{0} }{\operatorname{grad}_{#3}\left( #2 \right)}{\operatorname{grad}_{#3}{#2}}
}
\newcommand{\mleng}[2][0]{
	\ifthenelse{ \equal{#1}{0} }{\ell\left( #2 \right)}{\ell{#2}}
}
\DeclareMathOperator*{\inte}{int}
\newcommandx{\minte}[3][1 = 0, 3 = { }, usedefault]{
	\ifthenelse{ \equal{#1}{0} }{\inte_{#3}\left( #2 \right)}{\inte_{#3}{#2}}
} 
\DeclareMathOperator*{\exte}{ext}
\newcommandx{\mexte}[3][1 = 0, 3 = { }, usedefault]{
	\ifthenelse{ \equal{#1}{0} }{\exte_{#3}\left( #2 \right)}{\exte_{#3}{#2}}
}
\DeclareMathOperator*{\fr}{Fr}
\newcommandx{\mfr}[3][1 = 0, 3 = { }, usedefault]{
	\ifthenelse{ \equal{#1}{0} }{\fr_{#3}\left(#2\right)}{\fr_{#3}{#2}}
}
\DeclareMathOperator*{\cl}{cl}
\newcommandx{\mcl}[3][1 = 0, 3 = { }, usedefault]{
	\ifthenelse{ \equal{#1}{0} }{\cl_{#3}\left( #2 \right)}{\cl_{#3}{#2}}
}
\DeclareMathOperator*{\der}{der}
\newcommandx{\mder}[3][1 = 0, 3 = { }, usedefault]{
	\ifthenelse{ \equal{#1}{0} }{\der_{#3}\left( #2 \right)}{\der_{#3}{#2}}
}
\newcommand{\mtrans}[2][0]{
	\ifthenelse{ \equal{#1}{0} }{{#2}^{\mathsf{T}}}{\left( #2 \right)^{\mathsf{T}}}
}
\newcommand{\mcomp}[2][0]{
	\ifthenelse{ \equal{#1}{0} }{{#2}^{\mathsf{C}}}{\left( #2 \right)^{\mathsf{C}}}
}
\newcommand{\minv}[2][0]{
	\ifthenelse{ \equal{#1}{0} }{{#2}^{-1}}{\left( #2 \right)^{-1}}
}
\newcommand{\mperp}[2][0]{
	\ifthenelse{ \equal{#1}{0} }{{#2}^{\perp}}{\left( #2 \right)^{\perp}}
}
\newcommand{\madj}[2][0]{
	\ifthenelse{ \equal{#1}{0} }{{#2}^{*}}{\left( #2 \right)^{*}}
}
\newcommand{\mproj}[2]{\operatorname{proj}_{#2}\left(#1\right)}
\newcommandx{\meval}[3][2 = { }, 3 = { }]{
	\left[ #1 \right]_{#2}^{#3}
}
\newcommandx{\mint}[5][1 = 0, 4 = { }, 5 = { }]{
	\ifthenelse{ \equal{#1}{0} }{
		\int_{#4}^{#5}{#2}\ d{#3}
	}{
		\int_{#4}^{#5}\left( #2 \right)d{#3}
	}
}
\newcommand{\mpol}[2]{\mathcal{P}_{#2}\left( #1 \right)}
\newcommand{\mmatrix}[2]{\mathcal{M}_{#2}\left( #1 \right)}
\newcommand{\mlin}[1]{\mathcal{L}\left( #1 \right)}
\newcommand{\mfunc}[1]{\mathcal{F}\left( #1 \right)}
\newcommand{\mdis}[1]{\operatorname{d}\left( #1 \right)}
\newcommand{\msinf}[1]{\operatorname{\underline{S}}\left( #1 \right)}
\newcommand{\mssup}[1]{\operatorname{\overline{S}}\left( #1 \right)}
\newcommand{\mmeas}[1]{\operatorname{m}\left( #1 \right)}
\newcommand{\mclass}[1][]{\mathcal{C}^{#1}}
%%%%%%%%%%%%%%%%%%%%%%%%%%%%%%%%%%%%%%%%%%%%%
	% Otros símbolos
\newcommand{\void}{\varnothing}
\newcommand{\true}{\top}
\newcommand{\false}{\bot}
\newcommand{\ivec}{\hat{\imath}}
\newcommand{\jvec}{\hat{\jmath}}
\newcommand{\kvec}{\hat{k}}
%%%%%%%%%%%%%%%%%%%%%%%%%%%%%%%%%%%%%%%%%%%%%
\begin{document}

\maketitle

\begin{abstract}
Aqu\'{\i} se presentan los comandos personalizados para símbolos y funciones matemáticas que no se encuentran en los paquetes AMS.
\end{abstract}
Estos comandos hacen uso de comandos de los paquetes de AMS, \emph{ifthen} y \emph{xargs}.
\section{Conjuntos de n\'{u}meros}
Estos son de la forma:
\begin{center}
	\ttfamily
	\textbackslash numero[$ \inprod{complemento} $]
\end{center}
donde el \emph{complemento} es un super\'{\i}ndice.
\begin{center}
	\begin{tabular}{ll|ll}
		$ \cmath $ & \texttt{\textbackslash cmath} & $ \cmath[n] $ & \texttt{\textbackslash cmath[n]} \\
		$ \rmath $ & \texttt{\textbackslash rmath} & $ \rmath[n] $ & \texttt{\textbackslash rmath[n]}\\
		$ \qmath $ & \texttt{\textbackslash qmath} & $ \qmath[n] $ & \texttt{\textbackslash qmath[n]}\\
		$ \irrmath $ & \texttt{\textbackslash irrmath} & $ \irrmath[n] $ & \texttt{\textbackslash irrmath[n]}\\
		$ \zmath $ & \texttt{\textbackslash zmath} & $ \zmath[n] $ & \texttt{\textbackslash zmath[n]}\\
		$ \nmath $ & \texttt{\textbackslash nmath} & $ \nmath[n] $ & \texttt{\textbackslash nmath[n]}\\
		$ \pmath $ & \texttt{\textbackslash pmath} & $ \pmath[n] $ & \texttt{\textbackslash pmath[n]}\\
		$ \fmath $ & \texttt{\textbackslash fmath} & $ \fmath[n] $ & \texttt{\textbackslash fmath[n]}\\
		$ \hmath $ & \texttt{\textbackslash hmath} & $ \hmath[n] $ & \texttt{\textbackslash hmath[n]}\\
		$ \umath $ & \texttt{\textbackslash umath} & $ \umath[n] $ & \texttt{\textbackslash umath[n]}
	\end{tabular}
\end{center}
\section{Delimitadores}
\begin{center}
	\begin{tabular}{ll|ll}
		$ \pa{abc} $ & \texttt{\textbackslash pa\{abc\}} & $ \inprod{abc} $ & \texttt{\textbackslash inprod\{abc\}}\\
		$ \bracket{abc} $ & \texttt{\textbackslash bracket\{abc\}} & $ \floor{abc} $ & \texttt{\textbackslash floor\{abc\}}\\
		$ \set{abc} $ & \texttt{\textbackslash set\{abc\}} & $ \ceil{abc} $ & \texttt{\textbackslash ceil\{abc\}}\\
		$ \abs{abc} $ & \texttt{\textbackslash abs\{abc\}} & $ \upcor{abc} $ & \texttt{\textbackslash upcor\{abc\}}\\
		$ \norm{abc} $ & \texttt{\textbackslash norm\{abc\}} & $ \lowcor{abc} $ & \texttt{\textbackslash lowcor\{abc\}}\\
		$ \lopen{abc} $ & \texttt{\textbackslash lopen\{abc\}} & $ \ropen{abc} $ & \texttt{\textbackslash ropen\{abc\}}\\
		$ \Lopen{abc} $ & \texttt{\textbackslash Lopen\{abc\}} & $ \Ropen{abc} $ & \texttt{\textbackslash Ropen\{abc\}}
	\end{tabular}
\end{center}
Los delimitadores la incluyen el par \emph{left} y \emph{right}.
\section{Funciones est\'{a}ndar}
En estas la mayor\'{\i}a tiene la forma:
\begin{center}
	\ttfamily
	\textbackslash funcion[$ \inprod{delimitador} $]\{$ \inprod{argumento} $\}[$ \inprod{complemento} $]
\end{center}
donde el \emph{delimitador} indica si el argumento de la función estará delimitado por un paréntesis o no (incluye el par \emph{left} y \emph{right}). Esto se indica con $ 0 $ o vac\'{\i}o si será delimitado y con algún otro valor (de preferencia $ 1 $) si no.\par 
El \emph{complemento} es un sub\'{\i}ndice o super\'{\i}ndice según sea la función.
\begin{center}
	\begin{tabular}{ll|ll}
		$ \minf{A} $ & \texttt{\textbackslash minf\{A\}} & $ \minf[1]{A} $ & \texttt{\textbackslash minf[1]\{A\}} \\
		$ \ds\minf{f(x)}[x\in A] $ & \texttt{\textbackslash minf\{f(x)\}[x\textbackslash in A]} & $ \ds\minf[1]{f(x)}[x\in A] $ & \texttt{\textbackslash minf[1]\{f(x)\}[x\textbackslash in A]}\\
		$ \msup{A} $ & \texttt{\textbackslash msup\{A\}} & $ \msup[1]{A} $ & \texttt{\textbackslash msup[1]\{A\}} \\
		$ \ds\msup{f(x)}[x\in A] $ & \texttt{\textbackslash msup\{f(x)\}[x\textbackslash in A]} & $ \ds\msup[1]{f(x)}[x\in A] $ & \texttt{\textbackslash msup[1]\{f(x)\}[x\textbackslash in A]}\\
		$ \msin{x} $ & \texttt{\textbackslash msin\{x\}} & $ \msin[1]{x} $ & \texttt{\textbackslash msin[1]\{x\}}\\
		$ \msin{x}[n] $ & \texttt{\textbackslash msin\{x\}[n]} & $ \msin[1]{x}[n] $ & \texttt{\textbackslash msin[1]\{x\}[n]}\\
		$ \mcos{x} $ & \texttt{\textbackslash mcos\{x\}} & $ \mcos[1]{x} $ & \texttt{\textbackslash mcos[1]\{x\}}\\
		$ \mcos{x}[n] $ & \texttt{\textbackslash mcos\{x\}[n]} & $ \mcos[1]{x}[n] $ & \texttt{\textbackslash mcos[1]\{x\}[n]}\\
		$ \mtan{x} $ & \texttt{\textbackslash mtan\{x\}} & $ \mtan[1]{x} $ & \texttt{\textbackslash mtan[1]\{x\}}\\
		$ \mtan{x}[n] $ & \texttt{\textbackslash mtan\{x\}[n]} & $ \mtan[1]{x}[n] $ & \texttt{\textbackslash mtan[1]\{x\}[n]}\\
		$ \msec{x} $ & \texttt{\textbackslash msec\{x\}} & $ \msec[1]{x} $ & \texttt{\textbackslash msec[1]\{x\}}\\
		$ \msec{x}[n] $ & \texttt{\textbackslash msec\{x\}[n]} & $ \msec[1]{x}[n] $ & \texttt{\textbackslash msec[1]\{x\}[n]}\\
		$ \mcsc{x} $ & \texttt{\textbackslash mcsc\{x\}} & $ \mcsc[1]{x} $ & \texttt{\textbackslash mcsc[1]\{x\}}\\
		$ \mcsc{x}[n] $ & \texttt{\textbackslash mcsc\{x\}[n]} & $ \mcsc[1]{x}[n] $ & \texttt{\textbackslash mcsc[1]\{x\}[n]}\\
		$ \mcot{x} $ & \texttt{\textbackslash mcot\{x\}} & $ \mcot[1]{x} $ & \texttt{\textbackslash mcot[1]\{x\}}\\
		$ \mcot{x}[n] $ & \texttt{\textbackslash mcot\{x\}[n]} & $ \mcot[1]{x}[n] $ & \texttt{\textbackslash mcot[1]\{x\}[n]}\\
		$ \marcsin{x} $ & \texttt{\textbackslash marcsin\{x\}} & $ \marcsin[1]{x} $ & \texttt{\textbackslash marcsin[1]\{x\}}\\
		$ \marcsin{x}[n] $ & \texttt{\textbackslash marcsin\{x\}[n]} & $ \marcsin[1]{x}[n] $ & \texttt{\textbackslash marcsin[1]\{x\}[n]}\\
		$ \marccos{x} $ & \texttt{\textbackslash marccos\{x\}} & $ \marccos[1]{x} $ & \texttt{\textbackslash marccos[1]\{x\}}\\
		$ \marccos{x}[n] $ & \texttt{\textbackslash marccos\{x\}[n]} & $ \marccos[1]{x}[n] $ & \texttt{\textbackslash marccos[1]\{x\}[n]}\\
		$ \marctan{x} $ & \texttt{\textbackslash marctan\{x\}} & $ \marctan[1]{x} $ & \texttt{\textbackslash marctan[1]\{x\}}\\
		$ \marctan{x}[n] $ & \texttt{\textbackslash marctan\{x\}[n]} & $ \marctan[1]{x}[n] $ & \texttt{\textbackslash marctan[1]\{x\}[n]}\\
		$ \marcsec{x} $ & \texttt{\textbackslash marcsec\{x\}} & $ \marcsec[1]{x} $ & \texttt{\textbackslash marcsec[1]\{x\}}\\
		$ \marcsec{x}[n] $ & \texttt{\textbackslash marcsec\{x\}[n]} & $ \marcsec[1]{x}[n] $ & \texttt{\textbackslash marcsec[1]\{x\}[n]}\\
		$ \marccsc{x} $ & \texttt{\textbackslash marccsc\{x\}} & $ \marccsc[1]{x} $ & \texttt{\textbackslash marccsc[1]\{x\}}\\
		$ \marccsc{x}[n] $ & \texttt{\textbackslash marccsc\{x\}[n]} & $ \marccsc[1]{x}[n] $ & \texttt{\textbackslash marccsc[1]\{x\}[n]}\\
		$ \marccot{x} $ & \texttt{\textbackslash marccot\{x\}} & $ \marccot[1]{x} $ & \texttt{\textbackslash marccot[1]\{x\}}\\
		$ \marccot{x}[n] $ & \texttt{\textbackslash marccot\{x\}[n]} & $ \marccot[1]{x}[n] $ & \texttt{\textbackslash marccot[1]\{x\}[n]}\\
		$ \marg{x} $ & \texttt{\textbackslash marg\{x\}} & $ \marg[1]{x} $ & \texttt{\textbackslash marg[1]\{x\}}\\
		$ \mdeg{x} $ & \texttt{\textbackslash mdeg\{x\}} & $ \mdeg[1]{x} $ & \texttt{\textbackslash mdeg[1]\{x\}}\\
		$ \mdeg{v}[G] $ & \texttt{\textbackslash mdeg\{v\}[G]} & $ \mdeg[1]{v}[G] $ & \texttt{\textbackslash mdeg[1]\{v\}[G]}\\
		$ \mdet{A} $ & \texttt{\textbackslash mdet\{A\}} & $ \mdet[1]{A} $ & \texttt{\textbackslash mdet[1]\{A\}} \\
		$ \ds\mdet{f(x)}[x\in A] $ & \texttt{\textbackslash mdet\{f(x)\}[x\textbackslash in A]} & $ \ds\mdet[1]{f(x)}[x\in A] $ & \texttt{\textbackslash mdet[1]\{f(x)\}[x\textbackslash in A]}\\
		$ \mdim{x} $ & \texttt{\textbackslash mdim\{x\}} & $ \mdim[1]{x} $ & \texttt{\textbackslash mdim[1]\{x\}}\\
		$ \mexp{x} $ & \texttt{\textbackslash mexp\{x\}} & $ \mexp[1]{x} $ & \texttt{\textbackslash mexp[1]\{x\}}\\
		$ \mmcd{A} $ & \texttt{\textbackslash mmcd\{A\}} & $ \mmcd[1]{A} $ & \texttt{\textbackslash mmcd[1]\{A\}} \\
		$ \ds\mmcd{f(x)}[x\in A] $ & \texttt{\textbackslash mmcd\{f(x)\}[x\textbackslash in A]} & $ \ds\mmcd[1]{f(x)}[x\in A] $ & \texttt{\textbackslash mmcd[1]\{f(x)\}[x\textbackslash in A]}\\
		$ \mln{x} $ & \texttt{\textbackslash mln\{x\}} & $ \mln[1]{x} $ & \texttt{\textbackslash mln[1]\{x\}}\\
		$ \mlog{x} $ & \texttt{\textbackslash mlog\{x\}} & $ \mlog[1]{x} $ & \texttt{\textbackslash mlog[1]\{x\}} \\
		$ \ds\mlog{x}[a] $ & \texttt{\textbackslash mlog\{x\}[a]} & $ \ds\mlog[1]{x}[a] $ & \texttt{\textbackslash mlog[1]\{x\}[a]}\\
		$ \mmax{A} $ & \texttt{\textbackslash mmax\{A\}} & $ \mmax[1]{A} $ & \texttt{\textbackslash mmax[1]\{A\}} \\
		$ \ds\mmax{f(x)}[x\in A] $ & \texttt{\textbackslash mmax\{f(x)\}[x\textbackslash in A]} & $ \ds\mmax[1]{f(x)}[x\in A] $ & \texttt{\textbackslash mmax[1]\{f(x)\}[x\textbackslash in A]}\\
		$ \mmin{A} $ & \texttt{\textbackslash mmin\{A\}} & $ \mmin[1]{A} $ & \texttt{\textbackslash mmin[1]\{A\}} \\
		$ \ds\mmin{f(x)}[x\in A] $ & \texttt{\textbackslash mmin\{f(x)\}[x\textbackslash in A]} & $ \ds\mmin[1]{f(x)}[x\in A] $ & \texttt{\textbackslash mmin[1]\{f(x)\}[x\textbackslash in A]}\\
	\end{tabular}
\end{center}
\section{Otras funciones}
En estas la mayor\'{\i}a tiene la forma:
\begin{center}
	\ttfamily
	\textbackslash funcion[$ \inprod{delimitador} $]\{$ \inprod{argumento} $\}
\end{center}
donde el \emph{delimitador} indica si el argumento de la función estará delimitado por un paréntesis o no (incluye el par \emph{left} y \emph{right}). Esto se indica con $ 0 $ o vac\'{\i}o si será delimitado y con algún otro valor (de preferencia $ 1 $) si no.
\begin{center}
	\begin{tabular}{ll|ll}
		$ \mpow{A} $ & \texttt{\textbackslash mpow\{A\}} & $ \mpow[1]{A} $ & \texttt{\textbackslash mpow[1]\{A\}}\\
		$ \mnu{A} $ & \texttt{\textbackslash mnu\{A\}} & $ \mnu[1]{A} $ & \texttt{\textbackslash mnu[1]\{A\}}\\
		$ \mrho{A} $ & \texttt{\textbackslash mrho\{A\}} & $ \mrho[1]{A} $ & \texttt{\textbackslash mrho[1]\{A\}}\\
		$ \mnuc{A} $ & \texttt{\textbackslash mnuc\{A\}} & $ \mnuc[1]{A} $ & \texttt{\textbackslash mnuc[1]\{A\}}\\
		$ \mgen{A} $ & \texttt{\textbackslash mgen\{A\}} & $ \mgen[1]{A} $ & \texttt{\textbackslash mgen[1]\{A\}}\\
		$ \mtr{A} $ & \texttt{\textbackslash mtr\{A\}} & $ \mtr[1]{A} $ & \texttt{\textbackslash mtr[1]\{A\}}\\
		$ \mdom{A} $ & \texttt{\textbackslash mdom\{A\}} & $ \mdom[1]{A} $ & \texttt{\textbackslash mdom[1]\{A\}}\\
		$ \mran{A} $ & \texttt{\textbackslash mran\{A\}} & $ \mran[1]{A} $ & \texttt{\textbackslash mran[1]\{A\}}\\
		$ \mim{A} $ & \texttt{\textbackslash mim\{A\}} & $ \mim[1]{A} $ & \texttt{\textbackslash mim[1]\{A\}}\\
		$ \ipart{z} $ & \texttt{\textbackslash ipart\{z\}} & $ \ipart[1]{z} $ & \texttt{\textbackslash ipart[1]\{z\}}\\
		$ \rpart{z} $ & \texttt{\textbackslash rpart\{z\}} & $ \rpart[1]{z} $ & \texttt{\textbackslash rpart[1]\{z\}}\\
		$ \mprob{X} $ & \texttt{\textbackslash mprob\{X\}} & $ \mprob[1]{X} $ & \texttt{\textbackslash mprob[1]\{X\}}\\
		$ \mmean{X} $ & \texttt{\textbackslash mmean\{X\}} & $ \mmean[1]{X} $ & \texttt{\textbackslash mmean[1]\{X\}}\\
		$ \mvar{X} $ & \texttt{\textbackslash mvar\{X\}} & $ \mvar[1]{X} $ & \texttt{\textbackslash mvar[1]\{X\}}\\
		$ \mmcm{A} $ & \texttt{\textbackslash mmcm\{A\}} & $ \mmcm[1]{A} $ & \texttt{\textbackslash mmcm[1]\{A\}}\\
		$ \ds\mmcm{A}[x\in A] $ & \texttt{\textbackslash mmcm\{A\}[x\textbackslash in A]} & $ \ds\mmcm[1]{A}[x\in A] $ & \texttt{\textbackslash mmcm[1]\{A\}[x\textbackslash in A]}\\
		$ \mgrad{P} $ & \texttt{\textbackslash mgrad\{P\}} & $ \mgrad[1]{P} $ & \texttt{\textbackslash mgrad[1]\{P\}}\\
		$ \mgrad{v}[G] $ & \texttt{\textbackslash mgrad\{v\}[G]} & $ \mgrad[1]{v}[G] $ & \texttt{\textbackslash mgrad[1]\{v\}[G]}\\
		$ \mleng{T} $ & \texttt{\textbackslash mleng\{T\}} & $ \mleng[1]{T} $ & \texttt{\textbackslash mleng[1]\{T\}}\\
		$ \minte{A} $ & \texttt{\textbackslash minte\{A\}} & $ \minte[1]{A} $ & \texttt{\textbackslash minte[1]\{A\}}\\
		$ \ds\minte{f(x)}[x\in A] $ & \texttt{\textbackslash minte\{f(x)\}[x\textbackslash in A]} & $ \ds\minte[1]{f(x)}[x\in A] $ & \texttt{\textbackslash minte[1]\{f(x)\}[x\textbackslash in A]}\\
		$ \mexte{A} $ & \texttt{\textbackslash mexte\{A\}} & $ \mexte[1]{A} $ & \texttt{\textbackslash mexte[1]\{A\}}\\
		$ \ds\mexte{f(x)}[x\in A] $ & \texttt{\textbackslash mexte\{f(x)\}[x\textbackslash in A]} & $ \ds\mexte[1]{f(x)}[x\in A] $ & \texttt{\textbackslash mexte[1]\{f(x)\}[x\textbackslash in A]}\\
		$ \mfr{A} $ & \texttt{\textbackslash mfr\{A\}} & $ \mfr[1]{A} $ & \texttt{\textbackslash mfr[1]\{A\}}\\
		$ \ds\mfr{f(x)}[x\in A] $ & \texttt{\textbackslash mfr\{f(x)\}[x\textbackslash in A]} & $ \ds\mfr[1]{f(x)}[x\in A] $ & \texttt{\textbackslash mfr[1]\{f(x)\}[x\textbackslash in A]}\\
		$ \mcl{A} $ & \texttt{\textbackslash mcl\{A\}} & $ \mcl[1]{A} $ & \texttt{\textbackslash mcl[1]\{A\}}\\
		$ \ds\mcl{f(x)}[x\in A] $ & \texttt{\textbackslash mcl\{f(x)\}[x\textbackslash in A]} & $ \ds\mcl[1]{f(x)}[x\in A] $ & \texttt{\textbackslash mcl[1]\{f(x)\}[x\textbackslash in A]}\\
		$ \mder{A} $ & \texttt{\textbackslash mder\{A\}} & $ \mder[1]{A} $ & \texttt{\textbackslash mder[1]\{A\}}\\
		$ \ds\mder{f(x)}[x\in A] $ & \texttt{\textbackslash mder\{f(x)\}[x\textbackslash in A]} & $ \ds\mder[1]{f(x)}[x\in A] $ & \texttt{\textbackslash mder[1]\{f(x)\}[x\textbackslash in A]}\\
		$ \mtrans{A} $ & \texttt{\textbackslash mtrans\{A\}} & $ \mtrans[1]{A} $ & \texttt{\textbackslash mtrans[1]\{A\}}\\
		$ \mcomp{A} $ & \texttt{\textbackslash mcomp\{A\}} & $ \mcomp[1]{A} $ & \texttt{\textbackslash mcomp[1]\{A\}}\\
		$ \minv{A} $ & \texttt{\textbackslash minv\{A\}} & $ \minv[1]{A} $ & \texttt{\textbackslash minv[1]\{A\}}\\
		$ \mperp{A} $ & \texttt{\textbackslash mperp\{A\}} & $ \mperp[1]{A} $ & \texttt{\textbackslash mperp[1]\{A\}}\\
		$ \madj{A} $ & \texttt{\textbackslash madj\{A\}} & $ \madj[1]{A} $ & \texttt{\textbackslash madj[1]\{A\}}\\
		$ \meval{A}[\beta] $ & \texttt{\textbackslash meval\{A\}[\textbackslash beta]} & $ \meval{A}[\beta][\gamma] $ & \texttt{\textbackslash meval\{A\}[\textbackslash beta][\textbackslash gamma]}\\
		$ \ds\mint{f(u)}{u} $ & \texttt{\textbackslash mint\{f(u)\}\{u\}} & $ \ds\mint[1]{f(u)}{u} $ & \texttt{\textbackslash mint[1]\{f(u)\}\{u\}}\\
		$ \ds\mint{f(u)}{u}[a][b] $ & \texttt{\textbackslash mint\{f(u)\}\{u\}[a][b]} & $ \ds\mint[1]{f(u)}{u}[a][b] $ & \texttt{\textbackslash mint[1]\{f(u)\}\{u\}[a][b]}\\
		$ \mproj{u}{v} $ & \texttt{\textbackslash mproj\{u\}\{v\}} & $ \mmatrix{F}{nm} $ & \texttt{\textbackslash mmatrix\{F\}\{nm\}}\\
		$ \mlin{V} $ & \texttt{\textbackslash mlin\{V\}} & $ \mpol{F}{n} $ & \texttt{\textbackslash mpol\{F\}\{n\}}\\
		$ \mfunc{W} $ & \texttt{\textbackslash mfunc\{W\}} & $ \msinf{F} $ & \texttt{\textbackslash msinf\{F\}}\\
		$ \mssup{F} $ & \texttt{\textbackslash mssup\{F\}} & $ \mdis{x, y} $ & \texttt{\textbackslash mdis\{x, y\}}\\
		$\mmeas{A}$ & \textbackslash mmeas\{A\} & $\mclass[n]$ & \textbackslash mclass[n]
	\end{tabular}
\end{center}
\section{Otros s\'{\i}mbolos}
\begin{center}
	\begin{tabular}{ll|ll|ll}
		$ \void $ & \texttt{\textbackslash void} & $ \true $ & \texttt{\textbackslash true} & $ \false $ & \texttt{\textbackslash false}\\
		$ \ivec $ & \texttt{\textbackslash ivec} & $ \jvec $ & \texttt{\textbackslash jvec} & $ \kvec $ & \texttt{\textbackslash kvec}
	\end{tabular}
\end{center}
\end{document}
