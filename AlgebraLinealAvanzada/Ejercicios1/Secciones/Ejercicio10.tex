\begin{exercise}
	Encontrar una matriz ortogonal $Q$ cuyo primer renglón sea $\begin{bmatrix}
			1/3 & 2/3 & 2/3
	\end{bmatrix}$.
\end{exercise}
\begin{solution}
	Notemos que:
	\[  
		\begin{vmatrix}
			1/3 & 1 & 0\\
			2/3 & 0 & 0\\
			2/3 & 0 & 1
		\end{vmatrix}
		=
		\begin{vmatrix}
			1/3 & 1\\
			2/3 & 0
		\end{vmatrix}
		= -\dfrac{2}{3}\neq 0
	\]
	entonces $\set{
	v_{1} = \frac{1}{3}\begin{bsmallmatrix}
		1 \\ 2 \\ 2
	\end{bsmallmatrix},
	v_{2} = \begin{bsmallmatrix}
		1 \\ 0 \\ 0
	\end{bsmallmatrix},
	v_{3} = \begin{bsmallmatrix}
		0 \\ 0 \\ 1
	\end{bsmallmatrix}
	}$ es L.I. Ortogonalicemos:
	\begin{align*}
		u_{1} &= v_{1} = \dfrac{1}{3}
		\begin{bmatrix}
			1 \\ 2 \\ 2
		\end{bmatrix}\\
		u_{2} &= v_{2} - \dfrac{\inprod{v_{2}, u_{1}}}{\norm{u_{1}}^{2}}u_{1}\\
		& =
		\begin{bmatrix}
			1 \\ 0 \\ 0
		\end{bmatrix}
		-
		\dfrac{
		\begin{bmatrix}
			1 \\ 0 \\ 0
		\end{bmatrix}
		\cdot
		\begin{bmatrix}
			1/3 \\ 2/3 \\ 2/3
		\end{bmatrix}
		}{\norm{
		\begin{bmatrix}
			1/3 \\ 2/3 \\ 2/3
		\end{bmatrix}
		}^{2}}
		\begin{bmatrix}
			1/3 \\ 2/3 \\ 2/3
		\end{bmatrix}
		=
		\begin{bmatrix}
			1 \\ 0 \\ 0
		\end{bmatrix}
		-\dfrac{1}{3}
		\begin{bmatrix}
			1/3 \\ 2/3 \\ 2/3
		\end{bmatrix}\\
		&=
		\begin{bmatrix}
			1 \\ 0 \\ 0
		\end{bmatrix}
		-\dfrac{1}{9}
		\begin{bmatrix}
			1 \\ 2 \\ 2
		\end{bmatrix}
		=\dfrac{1}{9}
		\begin{bmatrix}
			8 \\ -2 \\ -2
		\end{bmatrix}\\
		u_{3} &= v_{3} - \dfrac{\inprod{v_{3}, u_{1}}}{\norm{u_{1}}^{2}}u_{1} - \dfrac{\inprod{v_{3}, u_{2}}}{\norm{u_{2}}^{2}}u_{2}\\
		&=
		\begin{bmatrix}
			0 \\ 0 \\ 1
		\end{bmatrix}
		-
		\dfrac{
		\begin{bmatrix}
			0 \\ 0 \\ 1
		\end{bmatrix}
		\cdot
		\begin{bmatrix}
			1/3 \\ 2/3 \\ 2/3
		\end{bmatrix}
		}{\norm{
		\begin{bmatrix}
			1/3 \\ 2/3 \\ 2/3
		\end{bmatrix}
		}^{2}}
		\begin{bmatrix}
			1/3 \\ 2/3 \\ 2/3
		\end{bmatrix}
	 	-
	 	\dfrac{
 		\begin{bmatrix}
 			0 \\ 0 \\ 1
 		\end{bmatrix}
 		\cdot
 		\begin{bmatrix}
 			8/9 \\ -2/9 \\ -2/9
 		\end{bmatrix}
	 	}{\norm{
		\begin{bmatrix}
			8/9 \\ -2/9 \\ -2/9
		\end{bmatrix}
 		}^{2}}
	 	\begin{bmatrix}
	 		8/9 \\ -2/9 \\ -2/9
	 	\end{bmatrix}\\
 		&= 
 		\begin{bmatrix}
 			0 \\ 0 \\ 1
 		\end{bmatrix}
 		-\dfrac{2}{3}
 		\begin{bmatrix}
 			1/3 \\ 2/3 \\ 2/3
 		\end{bmatrix}
 		+\dfrac{1}{4}
 		\begin{bmatrix}
 			8/9 \\ -2/9 \\ -2/9
 		\end{bmatrix} = \dfrac{1}{2}
 		\begin{bmatrix}
 			0 \\ -1 \\ 1
 		\end{bmatrix}
	\end{align*}
	Entonces el conjunto $\set{u_{1}, u_{2}, u_{3}}$ es un conjunto ortogonal. Normalicemos:
	\begin{align*}
		\norm{u_{1}} &=\norm{\dfrac{1}{3}
		\begin{bmatrix}
			1 \\ 2 \\ 2
		\end{bmatrix}
		} =\dfrac{1}{3}\norm{
		\begin{bmatrix}
			1 \\ 2 \\ 2
		\end{bmatrix}
		} =\dfrac{1}{3}\sqrt{1 + 4 + 4} = 1 \implies u_{1}' =
		\begin{bmatrix}
			1/3 \\ 2/3 \\ 2/3
		\end{bmatrix}\\
		\norm{u_{2}} &=\norm{\dfrac{1}{9}
		\begin{bmatrix}
			8 \\ -2 \\ -2
		\end{bmatrix}	
		} =\dfrac{1}{9}\norm{
		\begin{bmatrix}
			8 \\ -2 \\ -2
		\end{bmatrix}
		} =\dfrac{1}{9}\sqrt{64 + 4 + 4} =\dfrac{2\sqrt{2}}{3} \implies u_{2}' =
		\begin{bmatrix}
			2\sqrt{2}/3 \\ -\sqrt{2}/6 \\ -\sqrt{2}/6
		\end{bmatrix}\\
		\norm{u_{3}} &=\norm{\dfrac{1}{2}
		\begin{bmatrix}
			0 \\ -1 \\ 1
		\end{bmatrix}
		} =\dfrac{1}{2}\norm{
		\begin{bmatrix}
			0 \\ -1 \\ 1
		\end{bmatrix}
		} =\dfrac{1}{2}\sqrt{1 + 1} =\dfrac{\sqrt{2}}{2}\implies u_{3}' =
		\begin{bmatrix}
			0 \\ -\sqrt{2}/2 \\ \sqrt{2}/2
		\end{bmatrix}
	\end{align*}
	Por lo tanto, el conjunto $\set{u_{1}', u_{2}', u_{3}'}$ es ortonormal y entonces la matriz:
	\[ \begin{bmatrix}
		1/3 & 2/3 & 2/3\\
		2\sqrt{2}/3 & -\sqrt{2}/6 & -\sqrt{2}/6\\
		0 & -\sqrt{2}/2 & \sqrt{2}/2
	\end{bmatrix} \]
	es ortogonal.
\end{solution}