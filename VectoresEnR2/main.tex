\documentclass[11pt,a4paper]{beamer}
\usetheme{CambridgeUS}
\usecolortheme{dolphin}
\usepackage[spanish]{babel}				% Palabras reservadas traducidas al español 
\usepackage[utf8]{inputenc}				% Para poder imprimir caracteres especiales
\usepackage{fontenc}					% Para acentos
\usepackage{nameref}
\usepackage{translator}
%%%%%%%%%%%%%%%%%%%%%%
% Fuentes, funciones y teoremas matemáticos
\usepackage{amsmath}					% Carga los entornos matemáticos
\usepackage{amsfonts}					% Carga las fuentes matemáticas
\usepackage{amssymb}					% Carga caracteres matemáticos especiales
\usepackage{amsthm}						% Crea entornos de teoremas, definiciones, etc.
\usepackage{mdframed}					% Sirve para encerrar los teoremas de amsthm en cajas para darle un estilo más personalizado
\usepackage{mathtools}					% Extensión de símbolos matemáticos
\usepackage{cases}						% Para crear el entorno de definición de una expresión por casos
\usepackage{cancel}						% Permite tachar expresiones matemáticas			
\usepackage{leftindex}					% Para poder añadir subíndices y superíndices del lado izquierdo de una expresión
\usepackage[italic]{derivative}					% Extensión de las funciones de derivadas
\usepackage[intlimits]{esint}			% Extensión de símbolos de integrales
\usepackage{thmtools}
\spanishdecimal{.}
%%%%%%%%%%%%%%%%%%%%%%

\usepackage{listings}					% Para ingresar código fuente en diferentes lenguajes de programación
\usepackage{pifont}						% Inserta doodles

%%%%%%%%%%%%%%%%%%%%%%
% Para insertar pseudocódigo
\usepackage{algorithm}
\usepackage{algorithmic}
%%%%%%%%%%%%%%%%%%%%%%

\usepackage{xcolor}						% Permite modificar y definir colores
\usepackage{multicol}					% Permite crear múltiples columnas de texto en el documento
\usepackage{hyperref}
%%%%%%%%%%%%%%%%%%%%%%
% Para insertar y crear figuras
\usepackage{graphicx}					% Para insertar gráficos
\usepackage{tikz}						% Para crear figuras
\usepackage{venndiagram}				% Extensión de Tikz para crear diagramas de Venn
%%%%%%%%%%%%%%%%%%%%%%

\usepackage[many]{tcolorbox}			% Para poder crear cajas de color
\tcbuselibrary{theorems}					
\usepackage{enumerate}					% Para poder modificar el estilo de los items en el entorno enumerate
%%%%%%%%%%%%%%%%%%%%%%
\usepackage{ifthen}
\usepackage{xargs}
\usetikzlibrary{automata, arrows.meta, positioning}
\input{Preambulo/colors.tex}
\setbeamercolor*{palette primary}{bg=pgay3, fg = white}
\setbeamercolor*{palette secondary}{bg=pgay2, fg = white}
\setbeamercolor*{palette tertiary}{bg=pgay1, fg = white}
\setbeamercolor*{titlelike}{fg=pgay1}
\setbeamercolor*{title}{bg=pgay1, fg = white}
\setbeamercolor*{item}{fg=pgay1}
\setbeamercolor*{caption name}{fg=pgay1}
\usefonttheme{professionalfonts}
%------------------------------------------------------------
\titlegraphic{\includegraphics[height=1cm]{images/ESCUDO_ESFM.png}\includegraphics[height=1cm]{images/ipn2.png}}


\setbeamerfont{title}{size=\large}
\setbeamerfont{subtitle}{size=\small}
\setbeamerfont{author}{size=\small}
\setbeamerfont{date}{size=\small}
\setbeamerfont{institute}{size=\small}
\author{Omar Porfirio García}
\institute{ESFM}
\date{\today}
\setbeamercovered{dynamic}
%------------------------------------------------------------
\AtBeginSection[]{
	\begin{frame}
		\vfill
		\centering
		\begin{beamercolorbox}[sep=8pt,center,shadow=true,rounded=true]{title}
			\usebeamerfont{title}\insertsectionhead\par%
		\end{beamercolorbox}
		\vfill
	\end{frame}
}
%%%%%%%%%%%%%%%%%%%%%%%%%%%%%%%%%%%%
% Change standard block colors

% 1- Block title (background and text)
\setbeamercolor{block title}{bg=pgay1, fg=white}

% 2- Block body (background)
\setbeamercolor{block body}{bg=pgay1!10}
% Change alert block colors

% 1- Block title (background and text)
\setbeamercolor{block title alerted}{fg=white, bg=red}

% 2- Block body (background)
\setbeamercolor{block body alerted}{bg=red!10}
% Change example block colors

% 1- Block title (background and text)
\setbeamercolor{block title example}{fg=white, bg=teal}

% 2- Block body (background)
\setbeamercolor{block body example}{bg=teal!25}
\setbeamertemplate{theorems}[numbered]
\theoremstyle{definition}
\newtheorem{mtheorem}{Teorema}[section]
\newtheorem{mdefinition}{Definición}[section]
\AtBeginEnvironment{mdefinition}{
    \setbeamercolor{block title}{bg=pgay6, fg=white}
	\setbeamercolor{block body}{bg=pgay6!10!white, fg=black}
}
\AtBeginEnvironment{mtheorem}{
	\setbeamercolor{block title}{bg=pgay5, fg=white}
	\setbeamercolor{block body}{bg=pgay5!10!white, fg=black}
}
\newtheorem{axiom}{Axioma}[section]
\newtheorem{coll}[mtheorem]{Corolario}
\AtBeginEnvironment{axiom}{
    \setbeamercolor{block title}{bg=pgay4, fg=white}
	\setbeamercolor{block body}{bg=pgay4!10!white, fg=black}
}
\AtBeginEnvironment{coll}{
    \setbeamercolor{block title}{bg=pgay3, fg=white}
	\setbeamercolor{block body}{bg=pgay3!10!white, fg=black}
}

\newenvironment{exercise}{\begin{block}{Ejercicio:}}{\end{block}}
\renewenvironment{example}{\begin{exampleblock}{Ejemplo: }}{\end{exampleblock}}
\newenvironment{remark}{\begin{alertblock}{Observación:}}{\end{alertblock}}
\documentclass[10pt, a4paper]{article}
\usepackage[spanish]{babel}
%%%%%%%%%%%%%%%%%%%%%%
% Fuentes, funciones y teoremas matemáticos
\usepackage{amsmath}					% Carga los entornos matemáticos
\usepackage{amsfonts}					% Carga las fuentes matemáticas
\usepackage{amssymb}					% Carga caracteres matemáticos especiales
\usepackage{amsthm}						% Crea entornos de teoremas, definiciones, etc.
\usepackage{mdframed}					% Sirve para encerrar los teoremas de amsthm en cajas para darle un estilo más personalizado
\usepackage{mathtools}					% Extensión de símbolos matemáticos
\usepackage{cases}						% Para crear el entorno de definición de una expresión por casos
\usepackage{cancel}						% Permite tachar expresiones matemáticas			
\usepackage{leftindex}					% Para poder añadir subíndices y superíndices del lado izquierdo de una expresión
\usepackage[italic]{derivative}					% Extensión de las funciones de derivadas
%\usepackage[intlimits]{esint}			% Extensión de símbolos de integrales
\usepackage{thmtools}
%%%%%%%%%%%%%%%%%%%%%%
\usepackage[margin = 1in]{geometry}
%opening
\title{Comandos personalizados: matem\'{a}ticas}
\author{Omar Porfirio Garc\'{\i}a}

\usepackage{ifthen}
\usepackage{xargs}
%%%%%%%%%%%%%%%%%%%%%%%%%%%%%%%%%%%%%%%%%%%%%
\newcommand{\ds}{\displaystyle}
%%%%%%%%%%%%%%%%%%%%%%%%%%%%%%%%%%%%%%%%%%%%%
	% Conjuntos numéricos
\newcommand{\cmath}[1][]{\mathbb{C}^{#1}}
\newcommand{\rmath}[1][]{\mathbb{R}^{#1}}
\newcommand{\qmath}[1][]{\mathbb{Q}^{#1}}
\newcommand{\irrmath}[1][]{\mathbb{I}^{#1}}
\newcommand{\zmath}[1][]{\mathbb{Z}^{#1}}
\newcommand{\nmath}[1][]{\mathbb{N}^{#1}}
\newcommand{\pmath}[1][]{\mathbb{P}^{#1}}
\newcommand{\fmath}[1][]{\mathbb{F}^{#1}}
\newcommand{\hmath}[1][]{\mathbb{H}^{#1}}
\newcommand{\umath}[1][]{\mathbb{U}^{#1}}
%%%%%%%%%%%%%%%%%%%%%%%%%%%%%%%%%%%%%%%%%%%%%
	% Delimitadores
\newcommand{\pa}[1]{\left( #1 \right)}
\newcommand{\bracket}[1]{\left[ #1 \right]}
\newcommand{\set}[1]{\left\{ #1 \right\}}
\newcommand{\abs}[1]{\left\vert #1 \right\vert}
\newcommand{\norm}[1]{\left\Vert #1 \right\Vert}
\newcommand{\inprod}[1]{\left\langle #1 \right\rangle}
\newcommand{\floor}[1]{\left\lfloor #1 \right\rfloor}
\newcommand{\ceil}[1]{\left\lceil #1 \right\rceil}
\newcommand{\upcor}[1]{\left\ulcorner #1 \right\urcorner}
\newcommand{\lowcor}[1]{\left\llcorner #1 \right\lrcorner}
\newcommand{\lopen}[1]{\left( #1 \right]}
\newcommand{\ropen}[1]{\left[ #1 \right)}
\newcommand{\Lopen}[1]{\left] #1 \right]}
\newcommand{\Ropen}[1]{\left[ #1 \right[}
%%%%%%%%%%%%%%%%%%%%%%%%%%%%%%%%%%%%%%%%%%%%%
	% funciones estándar
\newcommandx{\minf}[3][1 = 0, 3 = { }, usedefault]{
	\ifthenelse{ \equal{#1}{0}}{
		\inf_{#3}\left( #2 \right)
	}{
		\inf_{#3}{#2} 
	}
}
\newcommandx{\msup}[3][1 = 0, 3 = { }, usedefault]{
	\ifthenelse{ \equal{#1}{0} }{
		\sup_{#3}\left( #2 \right)
	}{
		\sup_{#3}{#2}
	}
}
\newcommandx{\msin}[3][1 = 0, 3 = { }, usedefault]{
	\ifthenelse{ \equal{#1}{0} }{
		\sin^{#3}\left( #2 \right)
	}{
		\sin^{#3}{#2}
	}
}
\newcommandx{\mcos}[3][1 = 0, 3 = { }, usedefault]{
	\ifthenelse{ \equal{#1}{0} }{
		\cos^{#3}\left( #2 \right)
	}{
		\cos^{#3}{#2}
	}
}
\newcommandx{\mtan}[3][1 = 0, 3 = { }, usedefault]{
	\ifthenelse{ \equal{#1}{0} }{
		\tan^{#3}\left( #2 \right)
	}{
		\tan^{#3}{#2}
	}
}
\newcommandx{\msec}[3][1 = 0, 3 = { }]{
	\ifthenelse{ \equal{#1}{0} }{
		\sec^{#3}\left( #2 \right)
	}{
		\sec^{#3}{#2}
	}
}
\newcommandx{\mcsc}[3][1 = 0, 3 = { }]{
	\ifthenelse{ \equal{#1}{0} }{
		\csc^{#3}\left( #2 \right)
	}{
		\csc^{#3}{#2}
	}
}
\newcommandx{\mcot}[3][1 = 0, 3 = { }]{
	\ifthenelse{ \equal{#1}{0} }{
		\cot^{#3}\left( #2 \right)
	}{
		\cot^{#3}{#2}
	}
}
\newcommandx{\marcsin}[3][1 = 0, 3 = { }, usedefault]{
	\ifthenelse{ \equal{#1}{0} }{
		\arcsin^{#3}\left( #2 \right)
	}{
		\arcsin^{#3}{#2}
	}
}
\newcommandx{\marccos}[3][1 = 0, 3 = { }, usedefault]{
	\ifthenelse{ \equal{#1}{0} }{
		\arccos^{#3}\left( #2 \right)
	}{
		\arccos^{#3}{#2}
	}
}
\newcommandx{\marctan}[3][1 = 0, 3 = { }, usedefault]{
	\ifthenelse{ \equal{#1}{0} }{
		\arctan^{#3}\left( #2 \right)
	}{
		\arctan^{#3}{#2}
	}
}
\newcommandx{\marcsec}[3][1 = 0, 3 = { }, usedefault]{
	\ifthenelse{ \equal{#1}{0} }{
		\operatorname{arcsec}^{#3}\left( #2 \right)
	}{
		\operatorname{arcsec}^{#3}{#2}
	}
}
\newcommandx{\marccsc}[3][1 = 0, 3 = { }, usedefault]{
	\ifthenelse{ \equal{#1}{0} }{
		\operatorname{arccsc}^{#3}\left( #2 \right)
	}{
		\operatorname{arccsc}^{#3}{#2}
	}
}
\newcommandx{\marccot}[3][1 = 0, 3 = { }, usedefault]{
	\ifthenelse{ \equal{#1}{0} }{
		\operatorname{arccot}^{#3}\left( #2 \right)
	}{
		\operatorname{arccot}^{#3}{#2}
	}
}
\newcommand{\marg}[2][0]{
	\ifthenelse{ \equal{#1}{0} }{\arg\left( #2 \right)}{\arg{#2}}
}
\newcommandx{\mdeg}[3][1 = 0, 3 = { }]{
	\ifthenelse{ \equal{#1}{0}}{\deg_{#3}\left( #2 \right)}{\deg_{#3}{#2}}
}
\newcommandx{\mdet}[3][1 = 0, 3 = { }, usedefault]{
	\ifthenelse{ \equal{#1}{0} }{\det_{#3}\left( #2 \right)}{\det_{#3}{#2}}
}
\newcommand{\mdim}[2][0]{
	\ifthenelse{ \equal{#1}{0} }{\dim\left( #2 \right)}{\dim{#2}}
}
\newcommand{\mexp}[2][0]{
	\ifthenelse{ \equal{#1}{0} }{\exp\left( #2 \right)}{\exp{#2}}
}
\DeclareMathOperator*{\mcd}{mcd}
\newcommandx{\mmcd}[3][1 = 0, 3 = { }, usedefault]{
	\ifthenelse{ \equal{#1}{0} }{\mcd_{#3}\left( #2 \right)}{\mcd_{#3}{#2}}
}
\newcommand{\mln}[2][0]{
	\ifthenelse{ \equal{#1}{0} }{\ln\left( #2 \right)}{\ln{#2}}
}
\newcommandx{\mlog}[3][1 = 0, 3 = { }, usedefault]{
	\ifthenelse{ \equal{#1}{0} }{\log_{#3}\left( #2 \right)}{\log_{#3}{#2}}
}
\newcommandx{\mmax}[3][1 = 0, 3 = { }, usedefault]{
	\ifthenelse{ \equal{#1}{0} }{\max_{#3}\left( #2 \right)}{\max_{#3}{#2}}
}
\newcommandx{\mmin}[3][1 = 0, 3 = { }, usedefault]{
	\ifthenelse{ \equal{#1}{0} }{\min_{#3}\left( #2 \right)}{\min_{#3}{#2}}
}
%%%%%%%%%%%%%%%%%%%%%%%%%%%%%%%%%%%%%%%%%%%%%
	% Otras funciones
\newcommand{\mpow}[2][0]{
	\ifthenelse{ \equal{#1}{0} }{\mathfrak{P}\left( #2 \right)}{\mathfrak{P}{#2}}
}
\newcommand{\mnu}[2][0]{
	\ifthenelse{ \equal{#1}{0} }{\nu\left( #2 \right)}{\nu{#2}}
}
\newcommand{\mrho}[2][0]{
	\ifthenelse{ \equal{#1}{0} }{\rho\left( #2 \right)}{\rho{#2}}
}
\newcommand{\mnuc}[2][0]{
	\ifthenelse{ \equal{#1}{0} }{\operatorname{N}\left( #2 \right)}{\operatorname{N}{#2}}
}
\newcommand{\mgen}[2][0]{
	\ifthenelse{ \equal{#1}{0} }{\operatorname{gen}\left( #2 \right)}{\operatorname{gen}{#2}}
}
\newcommand{\mtr}[2][0]{
	\ifthenelse{ \equal{#1}{0} }{\operatorname{tr}\left( #2 \right)}{\operatorname{tr}{#2}}
}
\newcommand{\mdom}[2][0]{
	\ifthenelse{ \equal{#1}{0} }{\operatorname{Dom}\left( #2 \right)}{\operatorname{Dom}{#2}}
}
\newcommand{\mran}[2][0]{
	\ifthenelse{ \equal{#1}{0} }{\operatorname{Ran}\left( #2 \right)}{\operatorname{Ran}{#2}}
}
\newcommand{\mim}[2][0]{
	\ifthenelse{ \equal{#1}{0} }{\operatorname{Im}\left( #2 \right)}{\operatorname{Im}{#2}}
}
\newcommand{\ipart}[2][0]{
	\ifthenelse{ \equal{#1}{0} }{\Im\left( #2 \right)}{\Im{#2}}
}
\newcommand{\rpart}[2][0]{
	\ifthenelse{ \equal{#1}{0} }{\Re\left( #2 \right)}{\Re{#2}}
}
\newcommand{\mprob}[2][0]{
	\ifthenelse{ \equal{#1}{0} }{\mathbb{P}\left( #2 \right)}{\mathbb{P}{#2}}
}
\newcommand{\mmean}[2][0]{
	\ifthenelse{ \equal{#1}{0} }{\mathbb{E}\left( #2 \right)}{\mathbb{E}{#2}}
}
\newcommand{\mvar}[2][0]{
	\ifthenelse{ \equal{#1}{0} }{\mathbb{V}\left( #2 \right)}{\mathbb{V}{#2}}
}
\DeclareMathOperator*{\mcm}{mcm}
\newcommandx{\mmcm}[3][1 = 0, 3 = { }, usedefault]{
	\ifthenelse{ \equal{#1}{0} }{\mcm_{#3}\left( #2 \right)}{\mcm_{#3}{#2}}
}
\newcommandx{\mgrad}[3][1 = 0, 3 = { }]{
	\ifthenelse{ \equal{#1}{0} }{\operatorname{grad}_{#3}\left( #2 \right)}{\operatorname{grad}_{#3}{#2}}
}
\newcommand{\mleng}[2][0]{
	\ifthenelse{ \equal{#1}{0} }{\ell\left( #2 \right)}{\ell{#2}}
}
\DeclareMathOperator*{\inte}{int}
\newcommandx{\minte}[3][1 = 0, 3 = { }, usedefault]{
	\ifthenelse{ \equal{#1}{0} }{\inte_{#3}\left( #2 \right)}{\inte_{#3}{#2}}
} 
\DeclareMathOperator*{\exte}{ext}
\newcommandx{\mexte}[3][1 = 0, 3 = { }, usedefault]{
	\ifthenelse{ \equal{#1}{0} }{\exte_{#3}\left( #2 \right)}{\exte_{#3}{#2}}
}
\DeclareMathOperator*{\fr}{Fr}
\newcommandx{\mfr}[3][1 = 0, 3 = { }, usedefault]{
	\ifthenelse{ \equal{#1}{0} }{\fr_{#3}\left(#2\right)}{\fr_{#3}{#2}}
}
\DeclareMathOperator*{\cl}{cl}
\newcommandx{\mcl}[3][1 = 0, 3 = { }, usedefault]{
	\ifthenelse{ \equal{#1}{0} }{\cl_{#3}\left( #2 \right)}{\cl_{#3}{#2}}
}
\DeclareMathOperator*{\der}{der}
\newcommandx{\mder}[3][1 = 0, 3 = { }, usedefault]{
	\ifthenelse{ \equal{#1}{0} }{\der_{#3}\left( #2 \right)}{\der_{#3}{#2}}
}
\newcommand{\mtrans}[2][0]{
	\ifthenelse{ \equal{#1}{0} }{{#2}^{\mathsf{T}}}{\left( #2 \right)^{\mathsf{T}}}
}
\newcommand{\mcomp}[2][0]{
	\ifthenelse{ \equal{#1}{0} }{{#2}^{\mathsf{C}}}{\left( #2 \right)^{\mathsf{C}}}
}
\newcommand{\minv}[2][0]{
	\ifthenelse{ \equal{#1}{0} }{{#2}^{-1}}{\left( #2 \right)^{-1}}
}
\newcommand{\mperp}[2][0]{
	\ifthenelse{ \equal{#1}{0} }{{#2}^{\perp}}{\left( #2 \right)^{\perp}}
}
\newcommand{\madj}[2][0]{
	\ifthenelse{ \equal{#1}{0} }{{#2}^{*}}{\left( #2 \right)^{*}}
}
\newcommand{\mproj}[2]{\operatorname{proj}_{#2}\left(#1\right)}
\newcommandx{\meval}[3][2 = { }, 3 = { }]{
	\left[ #1 \right]_{#2}^{#3}
}
\newcommandx{\mint}[5][1 = 0, 4 = { }, 5 = { }]{
	\ifthenelse{ \equal{#1}{0} }{
		\int_{#4}^{#5}{#2}\ d{#3}
	}{
		\int_{#4}^{#5}\left( #2 \right)d{#3}
	}
}
\newcommand{\mpol}[2]{\mathcal{P}_{#2}\left( #1 \right)}
\newcommand{\mmatrix}[2]{\mathcal{M}_{#2}\left( #1 \right)}
\newcommand{\mlin}[1]{\mathcal{L}\left( #1 \right)}
\newcommand{\mfunc}[1]{\mathcal{F}\left( #1 \right)}
\newcommand{\mdis}[1]{\operatorname{d}\left( #1 \right)}
\newcommand{\msinf}[1]{\operatorname{\underline{S}}\left( #1 \right)}
\newcommand{\mssup}[1]{\operatorname{\overline{S}}\left( #1 \right)}
\newcommand{\mmeas}[1]{\operatorname{m}\left( #1 \right)}
\newcommand{\mclass}[1][]{\mathcal{C}^{#1}}
%%%%%%%%%%%%%%%%%%%%%%%%%%%%%%%%%%%%%%%%%%%%%
	% Otros símbolos
\newcommand{\void}{\varnothing}
\newcommand{\true}{\top}
\newcommand{\false}{\bot}
\newcommand{\ivec}{\hat{\imath}}
\newcommand{\jvec}{\hat{\jmath}}
\newcommand{\kvec}{\hat{k}}
%%%%%%%%%%%%%%%%%%%%%%%%%%%%%%%%%%%%%%%%%%%%%
\begin{document}

\maketitle

\begin{abstract}
Aqu\'{\i} se presentan los comandos personalizados para símbolos y funciones matemáticas que no se encuentran en los paquetes AMS.
\end{abstract}
Estos comandos hacen uso de comandos de los paquetes de AMS, \emph{ifthen} y \emph{xargs}.
\section{Conjuntos de n\'{u}meros}
Estos son de la forma:
\begin{center}
	\ttfamily
	\textbackslash numero[$ \inprod{complemento} $]
\end{center}
donde el \emph{complemento} es un super\'{\i}ndice.
\begin{center}
	\begin{tabular}{ll|ll}
		$ \cmath $ & \texttt{\textbackslash cmath} & $ \cmath[n] $ & \texttt{\textbackslash cmath[n]} \\
		$ \rmath $ & \texttt{\textbackslash rmath} & $ \rmath[n] $ & \texttt{\textbackslash rmath[n]}\\
		$ \qmath $ & \texttt{\textbackslash qmath} & $ \qmath[n] $ & \texttt{\textbackslash qmath[n]}\\
		$ \irrmath $ & \texttt{\textbackslash irrmath} & $ \irrmath[n] $ & \texttt{\textbackslash irrmath[n]}\\
		$ \zmath $ & \texttt{\textbackslash zmath} & $ \zmath[n] $ & \texttt{\textbackslash zmath[n]}\\
		$ \nmath $ & \texttt{\textbackslash nmath} & $ \nmath[n] $ & \texttt{\textbackslash nmath[n]}\\
		$ \pmath $ & \texttt{\textbackslash pmath} & $ \pmath[n] $ & \texttt{\textbackslash pmath[n]}\\
		$ \fmath $ & \texttt{\textbackslash fmath} & $ \fmath[n] $ & \texttt{\textbackslash fmath[n]}\\
		$ \hmath $ & \texttt{\textbackslash hmath} & $ \hmath[n] $ & \texttt{\textbackslash hmath[n]}\\
		$ \umath $ & \texttt{\textbackslash umath} & $ \umath[n] $ & \texttt{\textbackslash umath[n]}
	\end{tabular}
\end{center}
\section{Delimitadores}
\begin{center}
	\begin{tabular}{ll|ll}
		$ \pa{abc} $ & \texttt{\textbackslash pa\{abc\}} & $ \inprod{abc} $ & \texttt{\textbackslash inprod\{abc\}}\\
		$ \bracket{abc} $ & \texttt{\textbackslash bracket\{abc\}} & $ \floor{abc} $ & \texttt{\textbackslash floor\{abc\}}\\
		$ \set{abc} $ & \texttt{\textbackslash set\{abc\}} & $ \ceil{abc} $ & \texttt{\textbackslash ceil\{abc\}}\\
		$ \abs{abc} $ & \texttt{\textbackslash abs\{abc\}} & $ \upcor{abc} $ & \texttt{\textbackslash upcor\{abc\}}\\
		$ \norm{abc} $ & \texttt{\textbackslash norm\{abc\}} & $ \lowcor{abc} $ & \texttt{\textbackslash lowcor\{abc\}}\\
		$ \lopen{abc} $ & \texttt{\textbackslash lopen\{abc\}} & $ \ropen{abc} $ & \texttt{\textbackslash ropen\{abc\}}\\
		$ \Lopen{abc} $ & \texttt{\textbackslash Lopen\{abc\}} & $ \Ropen{abc} $ & \texttt{\textbackslash Ropen\{abc\}}
	\end{tabular}
\end{center}
Los delimitadores la incluyen el par \emph{left} y \emph{right}.
\section{Funciones est\'{a}ndar}
En estas la mayor\'{\i}a tiene la forma:
\begin{center}
	\ttfamily
	\textbackslash funcion[$ \inprod{delimitador} $]\{$ \inprod{argumento} $\}[$ \inprod{complemento} $]
\end{center}
donde el \emph{delimitador} indica si el argumento de la función estará delimitado por un paréntesis o no (incluye el par \emph{left} y \emph{right}). Esto se indica con $ 0 $ o vac\'{\i}o si será delimitado y con algún otro valor (de preferencia $ 1 $) si no.\par 
El \emph{complemento} es un sub\'{\i}ndice o super\'{\i}ndice según sea la función.
\begin{center}
	\begin{tabular}{ll|ll}
		$ \minf{A} $ & \texttt{\textbackslash minf\{A\}} & $ \minf[1]{A} $ & \texttt{\textbackslash minf[1]\{A\}} \\
		$ \ds\minf{f(x)}[x\in A] $ & \texttt{\textbackslash minf\{f(x)\}[x\textbackslash in A]} & $ \ds\minf[1]{f(x)}[x\in A] $ & \texttt{\textbackslash minf[1]\{f(x)\}[x\textbackslash in A]}\\
		$ \msup{A} $ & \texttt{\textbackslash msup\{A\}} & $ \msup[1]{A} $ & \texttt{\textbackslash msup[1]\{A\}} \\
		$ \ds\msup{f(x)}[x\in A] $ & \texttt{\textbackslash msup\{f(x)\}[x\textbackslash in A]} & $ \ds\msup[1]{f(x)}[x\in A] $ & \texttt{\textbackslash msup[1]\{f(x)\}[x\textbackslash in A]}\\
		$ \msin{x} $ & \texttt{\textbackslash msin\{x\}} & $ \msin[1]{x} $ & \texttt{\textbackslash msin[1]\{x\}}\\
		$ \msin{x}[n] $ & \texttt{\textbackslash msin\{x\}[n]} & $ \msin[1]{x}[n] $ & \texttt{\textbackslash msin[1]\{x\}[n]}\\
		$ \mcos{x} $ & \texttt{\textbackslash mcos\{x\}} & $ \mcos[1]{x} $ & \texttt{\textbackslash mcos[1]\{x\}}\\
		$ \mcos{x}[n] $ & \texttt{\textbackslash mcos\{x\}[n]} & $ \mcos[1]{x}[n] $ & \texttt{\textbackslash mcos[1]\{x\}[n]}\\
		$ \mtan{x} $ & \texttt{\textbackslash mtan\{x\}} & $ \mtan[1]{x} $ & \texttt{\textbackslash mtan[1]\{x\}}\\
		$ \mtan{x}[n] $ & \texttt{\textbackslash mtan\{x\}[n]} & $ \mtan[1]{x}[n] $ & \texttt{\textbackslash mtan[1]\{x\}[n]}\\
		$ \msec{x} $ & \texttt{\textbackslash msec\{x\}} & $ \msec[1]{x} $ & \texttt{\textbackslash msec[1]\{x\}}\\
		$ \msec{x}[n] $ & \texttt{\textbackslash msec\{x\}[n]} & $ \msec[1]{x}[n] $ & \texttt{\textbackslash msec[1]\{x\}[n]}\\
		$ \mcsc{x} $ & \texttt{\textbackslash mcsc\{x\}} & $ \mcsc[1]{x} $ & \texttt{\textbackslash mcsc[1]\{x\}}\\
		$ \mcsc{x}[n] $ & \texttt{\textbackslash mcsc\{x\}[n]} & $ \mcsc[1]{x}[n] $ & \texttt{\textbackslash mcsc[1]\{x\}[n]}\\
		$ \mcot{x} $ & \texttt{\textbackslash mcot\{x\}} & $ \mcot[1]{x} $ & \texttt{\textbackslash mcot[1]\{x\}}\\
		$ \mcot{x}[n] $ & \texttt{\textbackslash mcot\{x\}[n]} & $ \mcot[1]{x}[n] $ & \texttt{\textbackslash mcot[1]\{x\}[n]}\\
		$ \marcsin{x} $ & \texttt{\textbackslash marcsin\{x\}} & $ \marcsin[1]{x} $ & \texttt{\textbackslash marcsin[1]\{x\}}\\
		$ \marcsin{x}[n] $ & \texttt{\textbackslash marcsin\{x\}[n]} & $ \marcsin[1]{x}[n] $ & \texttt{\textbackslash marcsin[1]\{x\}[n]}\\
		$ \marccos{x} $ & \texttt{\textbackslash marccos\{x\}} & $ \marccos[1]{x} $ & \texttt{\textbackslash marccos[1]\{x\}}\\
		$ \marccos{x}[n] $ & \texttt{\textbackslash marccos\{x\}[n]} & $ \marccos[1]{x}[n] $ & \texttt{\textbackslash marccos[1]\{x\}[n]}\\
		$ \marctan{x} $ & \texttt{\textbackslash marctan\{x\}} & $ \marctan[1]{x} $ & \texttt{\textbackslash marctan[1]\{x\}}\\
		$ \marctan{x}[n] $ & \texttt{\textbackslash marctan\{x\}[n]} & $ \marctan[1]{x}[n] $ & \texttt{\textbackslash marctan[1]\{x\}[n]}\\
		$ \marcsec{x} $ & \texttt{\textbackslash marcsec\{x\}} & $ \marcsec[1]{x} $ & \texttt{\textbackslash marcsec[1]\{x\}}\\
		$ \marcsec{x}[n] $ & \texttt{\textbackslash marcsec\{x\}[n]} & $ \marcsec[1]{x}[n] $ & \texttt{\textbackslash marcsec[1]\{x\}[n]}\\
		$ \marccsc{x} $ & \texttt{\textbackslash marccsc\{x\}} & $ \marccsc[1]{x} $ & \texttt{\textbackslash marccsc[1]\{x\}}\\
		$ \marccsc{x}[n] $ & \texttt{\textbackslash marccsc\{x\}[n]} & $ \marccsc[1]{x}[n] $ & \texttt{\textbackslash marccsc[1]\{x\}[n]}\\
		$ \marccot{x} $ & \texttt{\textbackslash marccot\{x\}} & $ \marccot[1]{x} $ & \texttt{\textbackslash marccot[1]\{x\}}\\
		$ \marccot{x}[n] $ & \texttt{\textbackslash marccot\{x\}[n]} & $ \marccot[1]{x}[n] $ & \texttt{\textbackslash marccot[1]\{x\}[n]}\\
		$ \marg{x} $ & \texttt{\textbackslash marg\{x\}} & $ \marg[1]{x} $ & \texttt{\textbackslash marg[1]\{x\}}\\
		$ \mdeg{x} $ & \texttt{\textbackslash mdeg\{x\}} & $ \mdeg[1]{x} $ & \texttt{\textbackslash mdeg[1]\{x\}}\\
		$ \mdeg{v}[G] $ & \texttt{\textbackslash mdeg\{v\}[G]} & $ \mdeg[1]{v}[G] $ & \texttt{\textbackslash mdeg[1]\{v\}[G]}\\
		$ \mdet{A} $ & \texttt{\textbackslash mdet\{A\}} & $ \mdet[1]{A} $ & \texttt{\textbackslash mdet[1]\{A\}} \\
		$ \ds\mdet{f(x)}[x\in A] $ & \texttt{\textbackslash mdet\{f(x)\}[x\textbackslash in A]} & $ \ds\mdet[1]{f(x)}[x\in A] $ & \texttt{\textbackslash mdet[1]\{f(x)\}[x\textbackslash in A]}\\
		$ \mdim{x} $ & \texttt{\textbackslash mdim\{x\}} & $ \mdim[1]{x} $ & \texttt{\textbackslash mdim[1]\{x\}}\\
		$ \mexp{x} $ & \texttt{\textbackslash mexp\{x\}} & $ \mexp[1]{x} $ & \texttt{\textbackslash mexp[1]\{x\}}\\
		$ \mmcd{A} $ & \texttt{\textbackslash mmcd\{A\}} & $ \mmcd[1]{A} $ & \texttt{\textbackslash mmcd[1]\{A\}} \\
		$ \ds\mmcd{f(x)}[x\in A] $ & \texttt{\textbackslash mmcd\{f(x)\}[x\textbackslash in A]} & $ \ds\mmcd[1]{f(x)}[x\in A] $ & \texttt{\textbackslash mmcd[1]\{f(x)\}[x\textbackslash in A]}\\
		$ \mln{x} $ & \texttt{\textbackslash mln\{x\}} & $ \mln[1]{x} $ & \texttt{\textbackslash mln[1]\{x\}}\\
		$ \mlog{x} $ & \texttt{\textbackslash mlog\{x\}} & $ \mlog[1]{x} $ & \texttt{\textbackslash mlog[1]\{x\}} \\
		$ \ds\mlog{x}[a] $ & \texttt{\textbackslash mlog\{x\}[a]} & $ \ds\mlog[1]{x}[a] $ & \texttt{\textbackslash mlog[1]\{x\}[a]}\\
		$ \mmax{A} $ & \texttt{\textbackslash mmax\{A\}} & $ \mmax[1]{A} $ & \texttt{\textbackslash mmax[1]\{A\}} \\
		$ \ds\mmax{f(x)}[x\in A] $ & \texttt{\textbackslash mmax\{f(x)\}[x\textbackslash in A]} & $ \ds\mmax[1]{f(x)}[x\in A] $ & \texttt{\textbackslash mmax[1]\{f(x)\}[x\textbackslash in A]}\\
		$ \mmin{A} $ & \texttt{\textbackslash mmin\{A\}} & $ \mmin[1]{A} $ & \texttt{\textbackslash mmin[1]\{A\}} \\
		$ \ds\mmin{f(x)}[x\in A] $ & \texttt{\textbackslash mmin\{f(x)\}[x\textbackslash in A]} & $ \ds\mmin[1]{f(x)}[x\in A] $ & \texttt{\textbackslash mmin[1]\{f(x)\}[x\textbackslash in A]}\\
	\end{tabular}
\end{center}
\section{Otras funciones}
En estas la mayor\'{\i}a tiene la forma:
\begin{center}
	\ttfamily
	\textbackslash funcion[$ \inprod{delimitador} $]\{$ \inprod{argumento} $\}
\end{center}
donde el \emph{delimitador} indica si el argumento de la función estará delimitado por un paréntesis o no (incluye el par \emph{left} y \emph{right}). Esto se indica con $ 0 $ o vac\'{\i}o si será delimitado y con algún otro valor (de preferencia $ 1 $) si no.
\begin{center}
	\begin{tabular}{ll|ll}
		$ \mpow{A} $ & \texttt{\textbackslash mpow\{A\}} & $ \mpow[1]{A} $ & \texttt{\textbackslash mpow[1]\{A\}}\\
		$ \mnu{A} $ & \texttt{\textbackslash mnu\{A\}} & $ \mnu[1]{A} $ & \texttt{\textbackslash mnu[1]\{A\}}\\
		$ \mrho{A} $ & \texttt{\textbackslash mrho\{A\}} & $ \mrho[1]{A} $ & \texttt{\textbackslash mrho[1]\{A\}}\\
		$ \mnuc{A} $ & \texttt{\textbackslash mnuc\{A\}} & $ \mnuc[1]{A} $ & \texttt{\textbackslash mnuc[1]\{A\}}\\
		$ \mgen{A} $ & \texttt{\textbackslash mgen\{A\}} & $ \mgen[1]{A} $ & \texttt{\textbackslash mgen[1]\{A\}}\\
		$ \mtr{A} $ & \texttt{\textbackslash mtr\{A\}} & $ \mtr[1]{A} $ & \texttt{\textbackslash mtr[1]\{A\}}\\
		$ \mdom{A} $ & \texttt{\textbackslash mdom\{A\}} & $ \mdom[1]{A} $ & \texttt{\textbackslash mdom[1]\{A\}}\\
		$ \mran{A} $ & \texttt{\textbackslash mran\{A\}} & $ \mran[1]{A} $ & \texttt{\textbackslash mran[1]\{A\}}\\
		$ \mim{A} $ & \texttt{\textbackslash mim\{A\}} & $ \mim[1]{A} $ & \texttt{\textbackslash mim[1]\{A\}}\\
		$ \ipart{z} $ & \texttt{\textbackslash ipart\{z\}} & $ \ipart[1]{z} $ & \texttt{\textbackslash ipart[1]\{z\}}\\
		$ \rpart{z} $ & \texttt{\textbackslash rpart\{z\}} & $ \rpart[1]{z} $ & \texttt{\textbackslash rpart[1]\{z\}}\\
		$ \mprob{X} $ & \texttt{\textbackslash mprob\{X\}} & $ \mprob[1]{X} $ & \texttt{\textbackslash mprob[1]\{X\}}\\
		$ \mmean{X} $ & \texttt{\textbackslash mmean\{X\}} & $ \mmean[1]{X} $ & \texttt{\textbackslash mmean[1]\{X\}}\\
		$ \mvar{X} $ & \texttt{\textbackslash mvar\{X\}} & $ \mvar[1]{X} $ & \texttt{\textbackslash mvar[1]\{X\}}\\
		$ \mmcm{A} $ & \texttt{\textbackslash mmcm\{A\}} & $ \mmcm[1]{A} $ & \texttt{\textbackslash mmcm[1]\{A\}}\\
		$ \ds\mmcm{A}[x\in A] $ & \texttt{\textbackslash mmcm\{A\}[x\textbackslash in A]} & $ \ds\mmcm[1]{A}[x\in A] $ & \texttt{\textbackslash mmcm[1]\{A\}[x\textbackslash in A]}\\
		$ \mgrad{P} $ & \texttt{\textbackslash mgrad\{P\}} & $ \mgrad[1]{P} $ & \texttt{\textbackslash mgrad[1]\{P\}}\\
		$ \mgrad{v}[G] $ & \texttt{\textbackslash mgrad\{v\}[G]} & $ \mgrad[1]{v}[G] $ & \texttt{\textbackslash mgrad[1]\{v\}[G]}\\
		$ \mleng{T} $ & \texttt{\textbackslash mleng\{T\}} & $ \mleng[1]{T} $ & \texttt{\textbackslash mleng[1]\{T\}}\\
		$ \minte{A} $ & \texttt{\textbackslash minte\{A\}} & $ \minte[1]{A} $ & \texttt{\textbackslash minte[1]\{A\}}\\
		$ \ds\minte{f(x)}[x\in A] $ & \texttt{\textbackslash minte\{f(x)\}[x\textbackslash in A]} & $ \ds\minte[1]{f(x)}[x\in A] $ & \texttt{\textbackslash minte[1]\{f(x)\}[x\textbackslash in A]}\\
		$ \mexte{A} $ & \texttt{\textbackslash mexte\{A\}} & $ \mexte[1]{A} $ & \texttt{\textbackslash mexte[1]\{A\}}\\
		$ \ds\mexte{f(x)}[x\in A] $ & \texttt{\textbackslash mexte\{f(x)\}[x\textbackslash in A]} & $ \ds\mexte[1]{f(x)}[x\in A] $ & \texttt{\textbackslash mexte[1]\{f(x)\}[x\textbackslash in A]}\\
		$ \mfr{A} $ & \texttt{\textbackslash mfr\{A\}} & $ \mfr[1]{A} $ & \texttt{\textbackslash mfr[1]\{A\}}\\
		$ \ds\mfr{f(x)}[x\in A] $ & \texttt{\textbackslash mfr\{f(x)\}[x\textbackslash in A]} & $ \ds\mfr[1]{f(x)}[x\in A] $ & \texttt{\textbackslash mfr[1]\{f(x)\}[x\textbackslash in A]}\\
		$ \mcl{A} $ & \texttt{\textbackslash mcl\{A\}} & $ \mcl[1]{A} $ & \texttt{\textbackslash mcl[1]\{A\}}\\
		$ \ds\mcl{f(x)}[x\in A] $ & \texttt{\textbackslash mcl\{f(x)\}[x\textbackslash in A]} & $ \ds\mcl[1]{f(x)}[x\in A] $ & \texttt{\textbackslash mcl[1]\{f(x)\}[x\textbackslash in A]}\\
		$ \mder{A} $ & \texttt{\textbackslash mder\{A\}} & $ \mder[1]{A} $ & \texttt{\textbackslash mder[1]\{A\}}\\
		$ \ds\mder{f(x)}[x\in A] $ & \texttt{\textbackslash mder\{f(x)\}[x\textbackslash in A]} & $ \ds\mder[1]{f(x)}[x\in A] $ & \texttt{\textbackslash mder[1]\{f(x)\}[x\textbackslash in A]}\\
		$ \mtrans{A} $ & \texttt{\textbackslash mtrans\{A\}} & $ \mtrans[1]{A} $ & \texttt{\textbackslash mtrans[1]\{A\}}\\
		$ \mcomp{A} $ & \texttt{\textbackslash mcomp\{A\}} & $ \mcomp[1]{A} $ & \texttt{\textbackslash mcomp[1]\{A\}}\\
		$ \minv{A} $ & \texttt{\textbackslash minv\{A\}} & $ \minv[1]{A} $ & \texttt{\textbackslash minv[1]\{A\}}\\
		$ \mperp{A} $ & \texttt{\textbackslash mperp\{A\}} & $ \mperp[1]{A} $ & \texttt{\textbackslash mperp[1]\{A\}}\\
		$ \madj{A} $ & \texttt{\textbackslash madj\{A\}} & $ \madj[1]{A} $ & \texttt{\textbackslash madj[1]\{A\}}\\
		$ \meval{A}[\beta] $ & \texttt{\textbackslash meval\{A\}[\textbackslash beta]} & $ \meval{A}[\beta][\gamma] $ & \texttt{\textbackslash meval\{A\}[\textbackslash beta][\textbackslash gamma]}\\
		$ \ds\mint{f(u)}{u} $ & \texttt{\textbackslash mint\{f(u)\}\{u\}} & $ \ds\mint[1]{f(u)}{u} $ & \texttt{\textbackslash mint[1]\{f(u)\}\{u\}}\\
		$ \ds\mint{f(u)}{u}[a][b] $ & \texttt{\textbackslash mint\{f(u)\}\{u\}[a][b]} & $ \ds\mint[1]{f(u)}{u}[a][b] $ & \texttt{\textbackslash mint[1]\{f(u)\}\{u\}[a][b]}\\
		$ \mproj{u}{v} $ & \texttt{\textbackslash mproj\{u\}\{v\}} & $ \mmatrix{F}{nm} $ & \texttt{\textbackslash mmatrix\{F\}\{nm\}}\\
		$ \mlin{V} $ & \texttt{\textbackslash mlin\{V\}} & $ \mpol{F}{n} $ & \texttt{\textbackslash mpol\{F\}\{n\}}\\
		$ \mfunc{W} $ & \texttt{\textbackslash mfunc\{W\}} & $ \msinf{F} $ & \texttt{\textbackslash msinf\{F\}}\\
		$ \mssup{F} $ & \texttt{\textbackslash mssup\{F\}} & $ \mdis{x, y} $ & \texttt{\textbackslash mdis\{x, y\}}\\
		$\mmeas{A}$ & \textbackslash mmeas\{A\} & $\mclass[n]$ & \textbackslash mclass[n]
	\end{tabular}
\end{center}
\section{Otros s\'{\i}mbolos}
\begin{center}
	\begin{tabular}{ll|ll|ll}
		$ \void $ & \texttt{\textbackslash void} & $ \true $ & \texttt{\textbackslash true} & $ \false $ & \texttt{\textbackslash false}\\
		$ \ivec $ & \texttt{\textbackslash ivec} & $ \jvec $ & \texttt{\textbackslash jvec} & $ \kvec $ & \texttt{\textbackslash kvec}
	\end{tabular}
\end{center}
\end{document}


\author{Omar Porfirio García}

\title{Beamer}
\subtitle{Subtítulo}
\begin{document}
    \frame{\titlepage}
	\begin{frame}
		\frametitle{Contenido}
		\tableofcontents
	\end{frame}
	%%%%%%%%%%%%%%%%%%%%%%%%%%%%%%%%%%%%%%%%%%%%%%%%%%%%%%%%%%%%%%%%%%%%%%%%%%%%%
    \section{Vectores en $\rmath[2]$}
	\begin{frame}{Plano $\rmath[2]$}
	    \begin{mdefinition}[El plano cartesiano]
	        Sea $\rmath$ el conjunto de números reales. Definimos a $\rmath[2]$ como:
	        \[\rmath[2]\coloneqq\rmath\times\rmath=\set{(a,b):a,b\in\rmath}\]
	        \begin{figure}
	            \centering

\tikzset{every picture/.style={line width=0.75pt}} %set default line width to 0.75pt        

\begin{tikzpicture}[x=0.75pt,y=0.75pt,yscale=-1,xscale=1, scale=0.4]
%uncomment if require: \path (0,300); %set diagram left start at 0, and has height of 300

%Straight Lines [id:da31875425075049346] 
\draw[stealth-stealth](331.3,13.87) -- (331.3,279) ;

%Straight Lines [id:da7321937935224139] 
\draw[stealth-stealth](151,168.96) -- (511.61,168.96) ;

%Straight Lines [id:da914186055379401] 
\draw [color={rgb, 255:red, 155; green, 155; blue, 155 }  ,draw opacity=1 ] [dash pattern={on 4.5pt off 4.5pt}]  (409.14,95.65) -- (409.14,168.72) ;
%Straight Lines [id:da11993693614621659] 
\draw [color={rgb, 255:red, 155; green, 155; blue, 155 }  ,draw opacity=1 ] [dash pattern={on 4.5pt off 4.5pt}]  (331.04,96.14) -- (409.14,95.65) ;
%Shape: Ellipse [id:dp09496336303029529] 
\draw  [fill={rgb, 255:red, 74; green, 144; blue, 226 }  ,fill opacity=1 ] (402.67,95.65) .. controls (402.67,92.29) and (405.57,89.57) .. (409.14,89.57) .. controls (412.71,89.57) and (415.6,92.29) .. (415.6,95.65) .. controls (415.6,99.02) and (412.71,101.74) .. (409.14,101.74) .. controls (405.57,101.74) and (402.67,99.02) .. (402.67,95.65) -- cycle ;

% Text Node
\draw (517.95,174.76) node [anchor=north west][inner sep=0.75pt]    {$x$};
% Text Node
\draw (352.45,11.58) node [anchor=north west][inner sep=0.75pt]    {$y$};
% Text Node
\draw (411.64,64.21) node [anchor=north west][inner sep=0.75pt]  [font=\small]  {$( a,b)$};
% Text Node
\draw (404.17,173.54) node [anchor=north west][inner sep=0.75pt]    {$a$};
% Text Node
\draw (300.67,86.47) node [anchor=north west][inner sep=0.75pt]    {$b$};


\end{tikzpicture}

	        \end{figure}
	        A cada elemento de $\rmath[2]$ le denominamos \emph{punto} o \emph{vector}. A cada vector lo denotaremos como $\hat{x},\hat{y},\hat{a},\ldots$. A un elemento $a\in\rmath$ le denominaremos \emph{escalar}.
	    \end{mdefinition}    
	\end{frame}
	\section{Operaciones con vectores}
    \begin{frame}{Operaciones con vectores}
        \begin{mdefinition}[Operaciones con vectores]
            Sean $\hat{x}=(a,b),\hat{y}=(c,d)\in\rmath[2]$ vectores y $\alpha\in\rmath$. Entonces definimos:
            \begin{itemize}
                \item Suma \[\hat{x}+\hat{y}\coloneqq(a+c,b+d)\]
                \begin{figure}
                    \centering

\tikzset{every picture/.style={line width=0.75pt}} %set default line width to 0.75pt        

\begin{tikzpicture}[x=0.75pt,y=0.75pt,yscale=-1,xscale=1, scale = 0.3]
%uncomment if require: \path (0,300); %set diagram left start at 0, and has height of 300

%Shape: Axis 2D [id:dp2895144271926746] 
\draw[stealth-stealth](50,237.8) -- (353.67,237.8);
\draw[stealth-stealth](80.37,20) -- (80.37,262);
%Straight Lines [id:da4428976100423787] 
\draw [color={rgb, 255:red, 74; green, 144; blue, 226 }  ,draw opacity=1, -stealth]   (80.37,237.8) -- (127.95,91.05) ;
%Straight Lines [id:da10061366308583053] 
\draw [color={rgb, 255:red, 245; green, 166; blue, 35 }  ,draw opacity=1, -stealth]   (80.37,237.8) -- (223.06,193.45) ;
%Straight Lines [id:da09026142462636333] 
\draw [color={rgb, 255:red, 65; green, 117; blue, 5 }  ,draw opacity=1, -stealth]   (80.37,237.8) -- (271.76,45.62) ;
%Straight Lines [id:da04660663408257815] 
\draw [color={rgb, 255:red, 155; green, 155; blue, 155 }  ,draw opacity=1 ] [dash pattern={on 0.84pt off 2.51pt}]  (128.57,89.14) -- (271.26,44.79) ;
%Straight Lines [id:da9523317652250853] 
\draw [color={rgb, 255:red, 74; green, 74; blue, 74 }  ,draw opacity=1 ] [dash pattern={on 0.84pt off 2.51pt}]  (224.97,192.86) -- (272.55,46.1) ;

% Text Node
\draw (99.28,30) node [anchor=north west][inner sep=0.75pt]    {$\hat{x}$};
% Text Node
\draw (250,180) node [anchor=north west][inner sep=0.75pt]    {$\hat{y}$};
% Text Node
\draw (287.84,31.64) node [anchor=north west][inner sep=0.75pt]    {$\hat{x} +\hat{y}$};


\end{tikzpicture}

                \end{figure}
                \item Producto escalar \[\alpha\hat{x}\coloneqq(\alpha a,\alpha b)\]
                \begin{figure}
                    \centering

\tikzset{every picture/.style={line width=0.75pt}} %set default line width to 0.75pt        

\begin{tikzpicture}[x=0.75pt,y=0.75pt,yscale=-1,xscale=1, scale = 0.6]
%uncomment if require: \path (0,300); %set diagram left start at 0, and has height of 300

%Shape: Axis 2D [id:dp2895144271926746] 
\draw[stealth-stealth](48.6,97.21) -- (168,97.21);
\draw[stealth-stealth](60.54,22.61) -- (60.54,105.5);
%Straight Lines [id:da012847358875807346] 
\draw [color={rgb, 255:red, 65; green, 117; blue, 5 }  ,draw opacity=1, -stealth]   (63.73,93.95) -- (96.55,63.27) -- (132.07,30.06) ;
%Straight Lines [id:da9259906796128414] 
\draw [color={rgb, 255:red, 74; green, 144; blue, 226 }  ,draw opacity=1, -stealth]   (60.54,97.21) -- (98.29,61.38) ;

% Text Node
\draw (97.82,71.3) node [anchor=north west][inner sep=0.75pt]    {$\hat{x}$};
% Text Node
\draw (139.12,22.31) node [anchor=north west][inner sep=0.75pt]    {$\alpha \hat{x}$};


\end{tikzpicture}

                \end{figure}
            \end{itemize}
        \end{mdefinition}
    \end{frame}
    \begin{frame}
        \begin{remark}
        Definimos lo siguiente:
        \[ \hat{0}\coloneqq(0,0)\]
        \[-\hat{x}\coloneqq-1\cdot\hat{x}\]
        \end{remark}
    \end{frame}
    \begin{frame}{Ejemplos}
        \begin{example}
            Determine el vector resultante de cada operación con vectores donde $\ds\hat{x}=(1,2),\hat{y}=(3,5)$.
            \begin{itemize}
                \only{\item $\ds\hat{x}+\hat{y} $}<1>
                \only{\item $\ds3\hat{x}-\hat{y}$}<2>
                \only{\item $\ds\hat{y}-\hat{x}$}<3>
                \only{\item $\ds\dfrac{1}{2}\hat{x}-\dfrac{1}{3}\hat{y}$}<4>
            \end{itemize}
        \end{example}
    \end{frame}
    \begin{frame}{Ejemplos}
        \begin{example}
            Demuestre cada una de las propiedades con vectores.
            \begin{itemize}
                \only{\item $\ds\hat{x}+\hat{y}=\hat{y}+\hat{x}$}<1>
                \only{\item $\ds\pa{\hat{x}+\hat{y}}+\hat{z}=\hat{x}+\pa{\hat{y}+\hat{z}}$}<2>
                \only{\item $\ds0\cdot\hat{x}= \hat{0}$}<3>
                \only{\item $\ds\alpha\pa{\beta\hat{x}}=\pa{\alpha\beta}\hat{x}$}<4>
                \only{\item $\ds\alpha\pa{\hat{x}+\hat{y}}=\alpha\hat{x}+\alpha\hat{y}$}<5>
                \only{\item $\ds\pa{\alpha+\beta}\hat{x}=\alpha\hat{x}+\beta\hat{x}$}<6>
            \end{itemize}
        \end{example}
    \end{frame}
    \section{Norma euclideana}
    \begin{frame}{Norma euclideana}
        \begin{mdefinition}[Norma euclideana]
            Sea $\hat{x}=(a,b)\in\rmath[2]$ un vector. Definimos a la \emph{norma} de $\hat{x}$ como:
            \[\norm{\hat{x}}\coloneqq\sqrt{a^2+b^2}\]
            \begin{figure}
                \centering

\tikzset{every picture/.style={line width=0.75pt}} %set default line width to 0.75pt        

\begin{tikzpicture}[x=0.75pt,y=0.75pt,yscale=-1,xscale=1, scale = 1.5]
%uncomment if require: \path (0,300); %set diagram left start at 0, and has height of 300

%Shape: Axis 2D [id:dp2895144271926746] 
\draw[stealth-stealth](48.6,97.21) -- (168,97.21);
\draw[stealth-stealth](60.54,22.61) -- (60.54,105.5);
%Straight Lines [id:da012847358875807346] 
\draw [color={rgb, 255:red, 65; green, 117; blue, 5 }  ,draw opacity=1, -stealth]   (60.54,97.21) -- (96.55,63.27) -- (132.07,30.06) ;
%Straight Lines [id:da5918449803235333] 
\draw [color={rgb, 255:red, 128; green, 128; blue, 128 }  ,draw opacity=1 ] [dash pattern={on 0.84pt off 2.51pt}]  (52.54,85.94) -- (88.55,52) -- (125.53,17.43) ;
\draw [shift={(125.53,17.43)}, rotate = 136.93] [color={rgb, 255:red, 128; green, 128; blue, 128 }  ,draw opacity=1 ][line width=0.75]    (0,5.59) -- (0,-5.59)   ;
\draw [shift={(52.54,85.94)}, rotate = 136.69] [color={rgb, 255:red, 128; green, 128; blue, 128 }  ,draw opacity=1 ][line width=0.75]    (0,5.59) -- (0,-5.59)   ;

% Text Node
\draw (139.16,18.3) node [anchor=north west][inner sep=0.75pt]    {$\hat{x}$};
% Text Node
\draw (70,41.14) node [anchor=north west][inner sep=0.75pt]  [rotate=-317.42]  {$\Vert \hat{x}\Vert $};


\end{tikzpicture}

            \end{figure}
        \end{mdefinition}
    \end{frame}
    \begin{frame}{Ejemplos}
        \begin{example}
            Sean $\hat{x}=(1,4),\hat{y}=(2,3)\in\rmath[2]$ vectores. Calcule la norma de cada uno de los siguientes vectores:
            \begin{itemize}
                \only{\item $\ds\norm{\hat{x}}$}<1>
                \only{\item $\ds\norm{\hat{y}-\hat{x}}$}<2>
                \only{\item $\ds\norm{3\hat{x}+2\hat{y}}$}<3>
                \only{\item $\ds\norm{4\hat{y}-\hat{x}}$}<4>
            \end{itemize}
        \end{example}
    \end{frame}
    \begin{frame}{Ejemplos}
        \begin{example}
            Demuestre cada una de las siguientes propiedades de la norma euclideana:
            \begin{itemize}
                \only{\item $\ds\norm{\hat{x}}\geq0$}<1>
                \only{\item $\ds\norm{\hat{x}}=0\iff\hat{x}= \hat{0}$}<2>
                \only{\item $\ds\norm{\alpha\hat{x}}=\abs{\alpha}\norm{\hat{x}}$}<3>
                \only{\item $\ds\norm{\hat{x}+\hat{y}}\leq\norm{\hat{x}}+\norm{\hat{y}}$}<4>
            \end{itemize}
        \end{example}
    \end{frame}
    \section{Distancia}
    \begin{frame}{Distancia entre dos vectores}
        \begin{mdefinition}[Distancia]
            Sean $\hat{x},\hat{y}\in\rmath[2]$ vectores. Definimos la distancia entre $\hat{x}$ y $\hat{y}$ como:
            \[\mdis{\hat{x}, \hat{y}}\coloneqq\norm{\hat{y}-\hat{x}}\]
            \begin{figure}
                \centering

\tikzset{every picture/.style={line width=0.75pt}} %set default line width to 0.75pt        

\begin{tikzpicture}[x=0.75pt,y=0.75pt,yscale=-1,xscale=1, scale=1.5]
%uncomment if require: \path (0,300); %set diagram left start at 0, and has height of 300

%Shape: Axis 2D [id:dp2895144271926746] 
\draw[stealth-stealth](48.6,95.03) -- (168,95.03);
\draw[stealth-stealth](60.54,0.83) -- (60.54,105.5);
%Straight Lines [id:da012847358875807346] 
\draw [color={rgb, 255:red, 65; green, 117; blue, 5 }  ,draw opacity=1, -stealth]   (60.54,95.03) -- (79.5,21.94) ;
%Straight Lines [id:da4678990026795582] 
\draw [color={rgb, 255:red, 74; green, 144; blue, 226 }  ,draw opacity=1, -stealth]   (60.54,95.03) -- (148.14,60.73) ;
%Straight Lines [id:da42874013324357696] 
\draw [color={rgb, 255:red, 128; green, 128; blue, 128 }  ,draw opacity=1 ] [dash pattern={on 0.84pt off 2.51pt}]  (80,20) -- (150,60) ;
\draw [shift={(150,60)}, rotate = 209.74] [color={rgb, 255:red, 128; green, 128; blue, 128 }  ,draw opacity=1 ][line width=0.75]    (0,5.59) -- (0,-5.59)   ;
\draw [shift={(80,20)}, rotate = 209.74] [color={rgb, 255:red, 128; green, 128; blue, 128 }  ,draw opacity=1 ][line width=0.75]    (0,5.59) -- (0,-5.59)   ;

% Text Node
\draw (140.82,71.3) node [anchor=north west][inner sep=0.75pt]    {$\hat{x}$};
% Text Node
\draw (80.16,43.4) node [anchor=north west][inner sep=0.75pt]    {$\hat{y}$};
% Text Node
\draw (110,15) node [anchor=north west][inner sep=0.75pt]  [rotate=-32.17]  {$\operatorname{d}(\hat{x} ,\hat{y})$};


\end{tikzpicture}

            \end{figure}
        \end{mdefinition}
    \end{frame}
    \begin{frame}{Ejemplos}
        \begin{example}
            Calcule la distancia entre cada uno de los siguientes vectores:
            \begin{itemize}
                \only{\item $\ds\hat{x}=(2,3),\hat{y}=(4,9)$}<1>
                \only{\item $\ds\hat{x}=(1,4),\hat{y}=(1,7)$}<2>
                \only{\item $\ds\hat{x}=(5,7),\hat{y}=(1,7)$}<3>
                \only{\item $\ds\hat{x}=(3,6),\hat{y}=(7,8)$}<4>
            \end{itemize}
        \end{example}
    \end{frame}
    \begin{frame}{Ejemplos}
        \begin{example}
            Demuestre cada una de las siguientes propiedades de la distancia:
            \begin{itemize}
                \only{\item $\ds\mdis{\hat{x}, \hat{y}}=0\iff\hat{x}=\hat{y}$}<1>
                \only{\item $\ds\mdis{\hat{x}, \hat{y}}=\mdis{\hat{y}, \hat{x}}$}<2>
                \only{\item $\ds\mdis{\hat{x}, \hat{y}}\leq\mdis{\hat{x}, \hat{z}}+\mdis{\hat{z}, \hat{y}}$}<3>
                \only{\item $\ds\mdis{\hat{x}, \hat{y}}\geq0$}<4>
            \end{itemize}
        \end{example}
    \end{frame}
    \section{Vectores posicionados}
    \begin{frame}{Vectores posicionados}
        Hasta ahora solo hemos visto vectores que van del origen a un punto del plano. Sin embargo, podemos tener vectores que van de un punto $a\in\rmath[2]$ a un punto $b\in\rmath[2]$. 
        \begin{mdefinition}[Vector posicionado]
            Sean $\hat{x},\hat{y}\in\rmath[2]$ vectores. Al vector $\hat{z}$ que va de $\hat{x}$ a $\hat{y}$ le denominamos \emph{vector posicionado}.
        \end{mdefinition}
    \end{frame}
    \begin{frame}{}
        \begin{figure}
            \centering

\tikzset{every picture/.style={line width=0.75pt}} %set default line width to 0.75pt        

\begin{tikzpicture}[x=0.75pt,y=0.75pt,yscale=-1,xscale=1]
%uncomment if require: \path (0,300); %set diagram left start at 0, and has height of 300

%Shape: Axis 2D [id:dp6906193974786576] 
\draw[stealth-stealth](38,200.25) -- (251,200.25);
\draw[stealth-stealth](59.3,54) -- (59.3,216.5);
%Straight Lines [id:da8914511696206657] 
\draw [color={rgb, 255:red, 144; green, 19; blue, 254 }  ,draw opacity=1, -stealth]   (59.3,200.25) -- (101,121) ;
%Straight Lines [id:da8583685196096289] 
\draw [color={rgb, 255:red, 74; green, 144; blue, 226 }  ,draw opacity=1, -stealth]   (59.3,200.25) -- (247.2,70.04) ;
%Straight Lines [id:da9063258732277746] 
\draw[-stealth](101,121.5) -- (247.2,70.04) ;

% Text Node
\draw (84,111.4) node [anchor=north west][inner sep=0.75pt]    {$\hat{x}$};
% Text Node
\draw (242,80.07) node [anchor=north west][inner sep=0.75pt]    {$\hat{y}$};
% Text Node
\draw (154.67,75.4) node [anchor=north west][inner sep=0.75pt]    {$\hat{z}$};


\end{tikzpicture}

        \end{figure}
    \end{frame}
    \begin{frame}{Ejemplos}
        \begin{example}
            Dibuje al vector posicionado que va de $\hat{x}$ a $\hat{y}$.
            \begin{itemize}
                \only{\item $\hat{x} = (1, 3), \hat{y} = (2, 6)$}<1>
                \only{\item $\hat{x} = (4, 1), \hat{y} = (6, 2)$}<2>
                \only{\item $\hat{x} = (3, 3), \hat{y} = (2, 5)$}<3>
                \only{\item $\hat{x} = (4, 3), \hat{y} = (2, 1)$}<4>
            \end{itemize}
        \end{example}
    \end{frame}
    \begin{frame}{Operaciones de vectores posicionados}
        \begin{mdefinition}[Operaciones de vectores posicionados]
            Sean $\hat{w},\hat{x},\hat{y},\hat{z}\in\rmath[2]$ vectores y $\hat{u}, \hat{v}$ vectores posicionados que van de $\hat{w}$ a $\hat{x}$ y de $\hat{y}$ a $\hat{z}$ respectivamente.
            \begin{itemize}
                \item La suma de $\hat{u}$ y $\hat{v}$ es el vector posicionado que va de $\hat{w}+\hat{y}$ a $\hat{x}+\hat{z}$.
                \item El producto escalar de $\alpha\in\rmath$ por $\hat{u}$ es el vector posicionado que va de $\alpha\hat{w}$ a $\alpha\hat{x}$.
            \end{itemize}
        \end{mdefinition}
    \end{frame}
    \begin{frame}{}
        \begin{figure}
            \centering

\tikzset{every picture/.style={line width=0.75pt}} %set default line width to 0.75pt        

\begin{tikzpicture}[x=0.75pt,y=0.75pt,yscale=-1,xscale=1]
%uncomment if require: \path (0,300); %set diagram left start at 0, and has height of 300

%Shape: Axis 2D [id:dp6906193974786576] 
\draw  (38,200.25) -- (251,200.25)(59.3,54) -- (59.3,216.5) (244,195.25) -- (251,200.25) -- (244,205.25) (54.3,61) -- (59.3,54) -- (64.3,61)  ;
%Straight Lines [id:da8914511696206657] 
\draw [color={rgb, 255:red, 144; green, 19; blue, 254 }  ,draw opacity=1 ]   (99,173.67) -- (74.89,119.74) ;
\draw [shift={(73.67,117)}, rotate = 65.91] [fill={rgb, 255:red, 144; green, 19; blue, 254 }  ,fill opacity=1 ][line width=0.08]  [draw opacity=0] (8.93,-4.29) -- (0,0) -- (8.93,4.29) -- cycle    ;
%Straight Lines [id:da8583685196096289] 
\draw [color={rgb, 255:red, 74; green, 144; blue, 226 }  ,draw opacity=1 ]   (138.33,176.33) -- (202.01,180.15) ;
\draw [shift={(205,180.33)}, rotate = 183.43] [fill={rgb, 255:red, 74; green, 144; blue, 226 }  ,fill opacity=1 ][line width=0.08]  [draw opacity=0] (8.93,-4.29) -- (0,0) -- (8.93,4.29) -- cycle    ;
%Shape: Axis 2D [id:dp3672532999660203] 
\draw  (278.67,200.25) -- (457.67,200.25)(296.57,54) -- (296.57,216.5) (450.67,195.25) -- (457.67,200.25) -- (450.67,205.25) (291.57,61) -- (296.57,54) -- (301.57,61)  ;
%Straight Lines [id:da7163759864035493] 
\draw [color={rgb, 255:red, 0; green, 0; blue, 0 }  ,draw opacity=1 ]   (338.33,172.33) -- (419.73,75.96) ;
\draw [shift={(421.67,73.67)}, rotate = 130.18] [fill={rgb, 255:red, 0; green, 0; blue, 0 }  ,fill opacity=1 ][line width=0.08]  [draw opacity=0] (8.93,-4.29) -- (0,0) -- (8.93,4.29) -- cycle    ;
%Straight Lines [id:da9608884744681323] 
\draw [color={rgb, 255:red, 144; green, 19; blue, 254 }  ,draw opacity=1 ]   (338.33,172.33) -- (375.07,128.63) ;
\draw [shift={(377,126.33)}, rotate = 130.05] [fill={rgb, 255:red, 144; green, 19; blue, 254 }  ,fill opacity=1 ][line width=0.08]  [draw opacity=0] (8.93,-4.29) -- (0,0) -- (8.93,4.29) -- cycle    ;
%Straight Lines [id:da1534851362303924] 
\draw [color={rgb, 255:red, 144; green, 19; blue, 254 }  ,draw opacity=1 ] [dash pattern={on 0.84pt off 2.51pt}]  (59.3,200.25) -- (99,173.67) ;
%Straight Lines [id:da6654247155563924] 
\draw [color={rgb, 255:red, 144; green, 19; blue, 254 }  ,draw opacity=1 ] [dash pattern={on 0.84pt off 2.51pt}]  (59.3,200.25) -- (138.33,176.33) ;
%Straight Lines [id:da44653055353225923] 
\draw [color={rgb, 255:red, 144; green, 19; blue, 254 }  ,draw opacity=1 ] [dash pattern={on 0.84pt off 2.51pt}]  (138.33,176.33) -- (178.03,149.75) ;
%Straight Lines [id:da5504250166545934] 
\draw [color={rgb, 255:red, 144; green, 19; blue, 254 }  ,draw opacity=1 ] [dash pattern={on 0.84pt off 2.51pt}]  (99,173.67) -- (178.03,149.75) ;
%Straight Lines [id:da9269863810697978] 
\draw [color={rgb, 255:red, 144; green, 19; blue, 254 }  ,draw opacity=1 ] [dash pattern={on 0.84pt off 2.51pt}]  (73.67,117) -- (59.3,200.25) ;
%Straight Lines [id:da21226703585722917] 
\draw [color={rgb, 255:red, 144; green, 19; blue, 254 }  ,draw opacity=1 ] [dash pattern={on 0.84pt off 2.51pt}]  (59.3,200.25) -- (205,180.33) ;
%Straight Lines [id:da5605056385385692] 
\draw    (178.03,149.75) -- (217.51,99.44) ;
\draw [shift={(219.37,97.08)}, rotate = 128.13] [fill={rgb, 255:red, 0; green, 0; blue, 0 }  ][line width=0.08]  [draw opacity=0] (8.93,-4.29) -- (0,0) -- (8.93,4.29) -- cycle    ;
%Straight Lines [id:da737848432808621] 
\draw [color={rgb, 255:red, 144; green, 19; blue, 254 }  ,draw opacity=1 ] [dash pattern={on 0.84pt off 2.51pt}]  (219.37,97.08) -- (205,180.33) ;
%Straight Lines [id:da2576900790278136] 
\draw [color={rgb, 255:red, 144; green, 19; blue, 254 }  ,draw opacity=1 ] [dash pattern={on 0.84pt off 2.51pt}]  (73.67,117) -- (219.37,97.08) ;

% Text Node
\draw (64,95.4) node [anchor=north west][inner sep=0.75pt]    {$\hat{u}$};
% Text Node
\draw (208.67,171.4) node [anchor=north west][inner sep=0.75pt]    {$\hat{v}$};
% Text Node
\draw (201.33,72.4) node [anchor=north west][inner sep=0.75pt]    {$\hat{u} +\hat{v}$};
% Text Node
\draw (341.33,128.07) node [anchor=north west][inner sep=0.75pt]    {$\hat{u}$};
% Text Node
\draw (426.67,67.07) node [anchor=north west][inner sep=0.75pt]    {$\alpha \hat{u}$};


\end{tikzpicture}

        \end{figure}
    \end{frame}
    \begin{frame}{Ejemplos}
        \begin{example}
            Calcule cada operación entre los vectores posicionados $\hat{u},\hat{v}$ que van de $(2,4)$ a $(1,3)$ y de $(1,4)$ a $(2,3)$ respectivamente.
            \begin{itemize}
                \only{\item $\hat{u}+\hat{v}$}<1>
                \only{\item $\hat{u}-3\hat{v}$}<2>
                \only{\item $\hat{v}+2\hat{u}$}<3>
                \only{\item $2\hat{u}-2\hat{v}$}<4>
            \end{itemize}
        \end{example}
    \end{frame}
    \section{Vectores equivalentes}
    \begin{frame}{Equivalencia entre vectores}
        \begin{mdefinition}[Vectores equivalentes]
            Sean $\hat{w},\hat{x},\hat{y},\hat{z}\in\rmath[2]$ vectores y $\hat{u}, \hat{v}$ vectores posicionados que van de $\hat{w}$ a $\hat{x}$ y de $\hat{y}$ a $\hat{z}$ respectivamente. Decimos que $\hat{u}$ y $\hat{v}$ son \emph{equivalentes} o \emph{equipolentes} si y solo si se cumple que:
            \[\hat{x} - \hat{w} = \hat{z} - \hat{y}\]
            Si es el caso, entonces lo denotamos como $\hat{u}\sim\hat{v}$.
        \end{mdefinition}
    \end{frame}
    \begin{frame}{Ejemplos}
        \begin{example}
            Determine si cada par de vectores posicionados son equivalentes.
            \begin{itemize}
                \only{\item $\hat{u},\hat{v}$ que van de $(2,4)$ a $(1,3)$ y de $(1,4)$ a $(2,3)$ respectivamente}<1>
                \only{\item $\hat{u},\hat{v}$ que van de $(3,4)$ a $(5,2)$ y de $(0,0)$ a $(2,-2)$ respectivamente}<2>
                \only{\item $\hat{u},\hat{v}$ que van de $(7,4)$ a $(2,8)$ y de $(4,4)$ a $(-1,0)$ respectivamente}<3>
                \only{\item $\hat{u},\hat{v}$ que van de $(1,3)$ a $(5,-2)$ y de $(3,7)$ a $(-1,-4)$ respectivamente}<4>
            \end{itemize}
        \end{example}
    \end{frame}
    \begin{frame}{Propiedades de los vectores equivalentes}
        \begin{mtheorem}{Propiedades de los vectores equivalentes}
            \begin{enumerate}
                \item Todo vector posicionado es equivalente a sí mismo, es decir, si $\hat{u}$ es un vector posicionado, entonces $\hat{u}\sim\hat{u}$.
                \item Sean dos vectores $\hat{u},\hat{v}$ posicionados. Si $\hat{u}\sim\hat{v}$, entonces $\hat{v}\sim\hat{u}$.
                \item Sean $\hat{u},\hat{v},\hat{w}$ vectores posicionados. Si $\hat{u}\sim\hat{v}$ y $\hat{v}\sim\hat{w}$, entonces $\hat{u}\sim\hat{w}$.
            \end{enumerate}
        \end{mtheorem}
    \end{frame}
    \begin{frame}{Demostración}
        \begin{proof}
            \begin{enumerate}
                \item Sea $\hat{u}$ un vector posicionado que va de $\hat{x}$ a $\hat{y}$, entonces $\hat{y}-\hat{x}=\hat{y}-\hat{x}$. Por lo tanto, $\hat{u}\sim\hat{u}$.
                \item Sean $\hat{u},\hat{v}$ vectores posicionados que van de $\hat{w}$ a $\hat{x}$ y de $\hat{y}$ a $\hat{z}$. Si $\hat{u}\sim\hat{v}$, entonces $\hat{x}-\hat{w}=\hat{z}-\hat{y}$ pero también $\hat{z}-\hat{y}=\hat{x}-\hat{w}$. Por lo tanto, $\hat{v}\sim\hat{u}$.
                \item Sean $\hat{u},\hat{v},\hat{w}$ dos vectores posicionados que van de $\hat{x}$ a $\hat{y}$, de $\hat{z}$ a $\hat{a}$ y de $\hat{b}$ a $\hat{c}$. Si $\hat{u}\sim\hat{v}$ y $\hat{v}\sim\hat{w}$, entonces $\hat{y}-\hat{x}=\hat{a}-\hat{z}$ y $\hat{a}-\hat{z}=\hat{c}-\hat{b}$. Luego, se tiene que $\hat{y}-\hat{x}=\hat{c}-\hat{b}$. Por lo tanto, $\hat{u}\sim\hat{w}$.
            \end{enumerate}
        \end{proof}
    \end{frame}
    \begin{frame}{Clases de vectores}
        \begin{mdefinition}[Clases de vectores]
            Sea $\hat{u}$ un vector posicionado. Denominamos al siguiente conjunto como \emph{clase de vectores} de $\hat{u}$:
            \[\bracket{\hat{u}}\coloneqq\set{\hat{v}:\hat{u}\sim\hat{v}}\]
        \end{mdefinition}
    \end{frame}
    \begin{frame}{Propiedades de las clases de vectores}
        \begin{mtheorem}{Propiedades de las clases de vectores}
            Sean $\hat{u},\hat{v}$ vectores posicionados.
            \begin{enumerate}
                \item $\bracket{\hat{u}}\neq\void$
                \item $\hat{v}\in\bracket{\hat{u}}\iff\bracket{\hat{u}}=\bracket{\hat{v}}$
                \item $\bracket{\hat{u}}\cap\bracket{\hat{v}}=\void\iff\bracket{\hat{u}}\neq\bracket{\hat{v}}$
            \end{enumerate}
        \end{mtheorem}
    \end{frame}
    \begin{frame}{Demostración}
        \begin{proof}
            \only{
                \begin{enumerate}
                    \item 
                    \label{item : clase_no_vacia}
                    Si $\hat{u}$ es un vector posicionado, entonces $\hat{u}\in\bracket{\hat{u}}$ pues $\hat{u}\sim\hat{u}$. Por lo tanto, $\bracket{\hat{u}}\neq\void$.
                    \item 
                    \label{item : clases_iguales}
                    \begin{itemize}
                        \item[$\implies)$] Si $\hat{v}\in\bracket{\hat{u}}$, entonces $\hat{u}\sim\hat{v}$. Además, si $\hat{w}\in\bracket{\hat{u}}$ es otro vector posicionado, entonces $\hat{w}\sim\hat{u}$ y luego $\hat{w}\sim\hat{v}$. Por lo tanto, $\hat{w}\in\bracket{\hat{v}}$ y entonces $\bracket{\hat{u}}\subseteq\bracket{\hat{v}}$.\par 
                        Ahora, si $\hat{w}\in\bracket{\hat{v}}$ es otro vector posicionado, entonces $\hat{w}\sim\hat{v}$ y luego $\hat{w}\sim\hat{u}$. Por lo tanto, $\hat{w}\in\bracket{\hat{u}}$ y entonces $\bracket{\hat{v}}\subseteq\bracket{\hat{u}}$. En conclusión, $\bracket{\hat{u}}=\bracket{\hat{v}}$.
                        \item[$\impliedby)$] Si $\bracket{\hat{u}}=\bracket{\hat{v}}$, entonces $\hat{v}\in\bracket{\hat{v}}=\bracket{\hat{u}}$.
                    \end{itemize}
                \end{enumerate}
            }<1>
            \only{
                \begin{enumerate}
                    \setcounter{enumi}{2}
                    \item\begin{itemize}
                        \item[$\implies)$] Si $\bracket{\hat{u}}\cap\bracket{\hat{v}}=\void$ y además por (\ref{item : clase_no_vacia}) $\bracket{u},\bracket{\hat{v}}\neq\void$, eso implica que $\bracket{\hat{u}}\neq\bracket{\hat{v}}$.
                        \item[$\impliedby)$] Probemos por contradicción. Supongamos que $\bracket{\hat{u}}\neq\bracket{\hat{v}}$ pero $\bracket{\hat{u}}\cap\bracket{\hat{v}}\neq\void$. Eso implica que existe al menos un $\hat{w}\in\bracket{\hat{u}}\cap\bracket{\hat{v}}$, es decir que $\hat{w}\sim\hat{u}$ y $\hat{w}\sim\hat{v}$. Por transitividad tenemos que $\hat{u}\sim\hat{v}$, pero eso implica que $\hat{u}\in\bracket{\hat{v}}$ y entonces por (\ref{item : clases_iguales}) se tiene que $\bracket{\hat{u}}=\bracket{\hat{v}}$, lo cual es una contradicción. Por lo tanto, $\bracket{\hat{u}}\cap\bracket{\hat{v}}=\void$.
                    \end{itemize}
                \end{enumerate}
            }<2>
        \end{proof}
    \end{frame}
    \begin{frame}{Conjunto de representantes}
        \begin{mdefinition}[Conjunto de representantes]
            Sea $\hat{u}\in\bracket{\hat{v}}$ con $\hat{u},\hat{v}$ vectores posicionados. $\hat{u}$ se denomina \emph{representante} de la clase $\bracket{\hat{v}}$ si y solo si el origen de $\hat{u}$ es $\hat{0}$, es decir, $\hat{u}$ es un vector que parte del origen.\par 
            Definimos a $S$ como el conjunto de todos los representantes de cada clase de vectores.
            \[S\coloneqq\set{\hat{u}:\hat{u}\text{ es representante de alguna clase }\bracket{\hat{v}}}\]
        \end{mdefinition}
        \begin{mtheorem}
            Toda clase de vectores tiene un vector representante.
        \end{mtheorem}
    \end{frame}
    \begin{frame}{Demostración}
        \begin{proof}
            Sea $\bracket{\hat{v}}$ una clase de vectores y $\hat{v}$ un vector cualquiera de la clase que va de un vector $\hat{x}$ a un vector $\hat{y}$. Resulta que el vector $\hat{y} -\hat{x}$ tiene como origen el vector $\hat{0}$, y es claro que $\hat{y} -\hat{x}\in\bracket{\hat{v}}$. Por lo tanto, $\hat{y} -\hat{x}$ es representante de $\bracket{\hat{v}}$.
        \end{proof}
    \end{frame}
    \begin{frame}{Unicidad del vector representante}
        \begin{mtheorem}[Unicidad del vector representante]
            El vector representante de una clase de vectores es único.
        \end{mtheorem}
        \begin{proof}
            Sean $\hat{u}, \hat{v}$ vectores representantes de la misma clase, luego $\hat{u}\sim\hat{v}$, por lo tanto si $\hat{u}$ va de $\hat{w}$ a $\hat{x}$ y $\hat{v}$ va de $\hat{y}$ a $\hat{z}$, entonces:
            \[\hat{x} -\hat{w} =\hat{z} -\hat{y}\]
            pero dado que $\hat{u}, \hat{v}$ son representantes, entonces $\hat{w} =\hat{y} =\hat{0}$. Por lo tanto, $\hat{x} =\hat{z}$ y entonces necesariamente $\hat{u} =\hat{v}$. En conclusión, el vector representante es único.
        \end{proof}
    \end{frame}
    \begin{frame}{Conjunto cociente de vectores}
        \begin{mdefinition}[Conjunto cociente]
            Sea $S$ el conjunto de representantes de las clases de vectores. Definimos al siguiente conjunto como el \emph{conjunto cociente} de vectores de $\rmath[2]$:
            \[\rmath[2]/\sim\coloneqq\set{\bracket{\hat{u}}:\hat{u}\in S}\]
        \end{mdefinition}
    \end{frame}
    \begin{frame}{Operaciones en el conjunto cociente}
        \begin{mdefinition}[Operaciones en el conjunto cociente]
            Sean $\bracket{\hat{u}},\bracket{\hat{v}}\in\rmath[2]/\sim$ clases de vectores. Definimos lo siguiente:
            \begin{itemize}
                \item Suma de clases
                \[\bracket{\hat{u}} +\bracket{\hat{v}}\coloneqq\bracket{\hat{u} +\hat{v}}\]
                \item Producto escalar de una clase
                \[\alpha\bracket{\hat{u}}\coloneqq\bracket{\alpha\hat{u}}\]
            \end{itemize}
        \end{mdefinition}
    \end{frame}
    \begin{frame}{}
        Falta probar que las operaciones descritas están \emph{bien definidas}, es decir, que la operación no dependa de los representantes. Debemos demostrar que si $\hat{u}\sim\hat{u}'$ y $\hat{v}\sim\hat{v}'$, entonces $\bracket{\hat{u} +\hat{v}} =\bracket{\hat{u}' +\hat{v}'}$ y $\bracket{\alpha\hat{u}} =\bracket{\alpha\hat{u}'}$.
    \end{frame}
    \begin{frame}{Demostración}
        \begin{proof}
            \only{
                Sean $\hat{u}\sim\hat{u}',\hat{v}\sim\hat{v}'$ vectores posicionados. Supongamos que $\hat{u}$ va de $\hat{w}$ a $\hat{x}$, $\hat{u}'$ va de $\hat{w}'$ a $\hat{x}'$, $\hat{v}$ va de $\hat{y}$ a $\hat{z}$ y $\hat{v}'$ va de $\hat{y}'$ a $\hat{z}'$.\par 
                Sabemos que si $\hat{u}\sim\hat{u}',\hat{v}\sim\hat{v}'$, entonces:
                \begin{align*}
                    \hat{x} -\hat{w} &=\hat{x}' -\hat{w}'\\
                    \hat{z} -\hat{y} &=\hat{z}' -\hat{y}'
                \end{align*}
                 Luego:
                 \[\pa{\hat{x} +\hat{z}} -\pa{\hat{w} +\hat{y}} =\pa{\hat{x}' +\hat{z}'} -\pa{\hat{w}' +\hat{y}'}\]
                 es decir, que $\hat{u} +\hat{v}\sim\hat{u}' +\hat{v}'$. Por lo tanto, $\bracket{\hat{u} +\hat{v}} =\bracket{\hat{u}' +\hat{v}'}$.
            }<1>
            \only{
                Por otra parte, tenemos que:
                \[\alpha\hat{x} -\alpha\hat{w} =\alpha\hat{x}' -\alpha\hat{w}'\]
                es decir, que $\alpha\hat{u}\sim\alpha\hat{u}'$. Por lo tanto, $\bracket{\alpha\hat{u}} =\bracket{\alpha\hat{u}'}$.\par 
                En conclusión, ambas operaciones están bien definidas.
            }<2>
        \end{proof}
    \end{frame}
    \section{Isomorfismo entre vectores}
    \begin{frame}{Isomorfismo entre $\rmath[2]$ y $\rmath[2]/\sim$}
        \begin{mdefinition}[Transformación lineal]
            Sea $T: V\to W$ una función sobre dos conjuntos $V, W$ tales que tienen operaciones suma y producto escalar en $\rmath$. $T$ se denomina \emph{transformación lineal} o simplemente \emph{lineal} si cumple que:
            \begin{enumerate}
                \item $T\pa{\hat{v} +\hat{w}} = T\pa{\hat{v}} + T\pa{\hat{w}}$
                \item $T\pa{\alpha\hat{v}} = \alpha T\pa{\hat{v}}$
            \end{enumerate}
            donde $\hat{v},\hat{w}\in V$ y $\alpha\in\rmath$.
        \end{mdefinition}
    \end{frame}
    \begin{frame}{Isomorfismo}
        \begin{mdefinition}[Isomorfismo]
            Sea $T: V\to W$ una transformación lineal sobre $V, W$. $T$ se dice \emph{isomorfismo} si y solo si tiene inversa.\par 
            Si $T$ es isomorfismo, se dice que $V$ y $W$ son \emph{isomorfos} y lo denotamos como $V\cong W$.
        \end{mdefinition}
    \end{frame}
    \begin{frame}{}
        \begin{mtheorem}
            $\rmath[2]$ y $\rmath[2]/\sim$ son isomorfos y además la transformación lineal $T:\rmath[2]\to\rmath[2]/\sim$ definida por:
            \[T(\hat{v}) =\bracket{\hat{v}}\]
            es un isomorfismo entre $\rmath[2]$ y $\rmath[2]/\sim$.
        \end{mtheorem}
    \end{frame}
\end{document}
