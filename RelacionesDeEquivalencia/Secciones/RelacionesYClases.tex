\section{Algunas relaciones de equivalencia}
\begin{frame}{Fracciones equivalentes}
	\begin{exercise}
		Sean $ (a, b), (c, d)\in\nmath[2] $. Diremos que $ (a, b)\sim(c, d) $ si y solo si:
		\[ ad = cb \]
		Demuestre que esta relación es de equivalencia.
	\end{exercise}
\end{frame}
\begin{frame}{Demostración}
	\begin{proof}
		\begin{itemize}
			\only{
				\item Reflexividad\\
				Sea $ (a, b)\in\nmath[2] $. Claramente se tiene que:
				\[ ab = ba \]
				Por lo tanto, $ (a, b)\sim(a, b) $.
			}<1>
			\only{
				\item Simetría\\
				Sean $ (a, b), (c, d)\in\nmath[2] $ tales que $ (a, b)\sim(c, d) $. Entonces tenemos que:
				\begin{align*}
					ad &= cb\\
					cb &= ad
				\end{align*}
				Por lo tanto, $ (c, d)\sim(a, b) $.
			}<2>
		\only{
			\item Transitividad\\
			Sean $ (a, b), (c, d), (e, f)\in\nmath[2] $ tales que $ (a, b)\sim(c, d) $ y $ (c, d)\sim(e, f) $. Entonces se tiene que:
			\begin{align*}
				ad = cb &\implies c =\frac{ad}{b}\\
				cf = ed &\implies c =\frac{ed}{f}\\
				\frac{ad}{b} =\frac{ed}{f} &\implies adf = bed\\
				af &= eb
			\end{align*}
			Por lo tanto, $ (a, b)\sim(e, f) $.
		}<3>
		\end{itemize}
	\end{proof}
\end{frame}
\begin{frame}{Fracciones equivalentes}
	Esta relación entre puntos de números enteros es como podemos definir a las fracciones equivalentes. Dos fracciones $ \frac{a}{b}, \frac{c}{d} $ las podemos ver como pares $ (a, b), (c, d) $ donde estas son equivalentes si y solo si:
	\[ ad = bc \]
	\begin{mdefinition}[Fracciones equivalentes]
		Sean $ \frac{a}{b}, \frac{c}{d}\in\qmath $. Decimos que $ \frac{a}{b} $ y $ \frac{c}{d} $ son \emph{equivalentes} si y solo si $ (a, b)\sim(c, d) $ con la relación de equivalencia descrita anteriormente.
	\end{mdefinition}
\end{frame}
\begin{frame}
	
\end{frame}
\begin{frame}{Módulos}
	\begin{exercise}
		Sean $ a, b, n\in\zmath $. Decimos que $ a\sim b $ si y solo si
		\[ n\mid(a - b) \]
		Demuestre que $ \sim $ es de equivalencia.
	\end{exercise}
\end{frame}
\begin{frame}{Demostración}
	\begin{proof}
		\begin{itemize}
			\only{
				\item Reflexividad\\
				Sea $ a\in\zmath $. Es claro que:
				\[ n\mid(a - a) \]
				Por lo tanto, $ a\sim a $.
			}<1>
			\only{
				\item Simetría\\
				 Sean $ a, b\in\zmath $ tales que $ a\sim b $. Entonces se tiene que:
				 \begin{align*}
					n\mid(a - b)\\
					n\mid(b - a)
				 \end{align*} 
				 Por lo tanto, $ b\sim a $.
			}<2>
			\only{
				\item Transitividad\\
				Sean $ a, b, c\in\zmath $ tales que $ a\sim b $ y $ b\sim c $. Entonces se tiene que existen $ k, m $ tales que:
				\begin{align*}
					nk &= a - b\\
					nm &= b - c
				\end{align*}
				De esto vemos que:
				\begin{align*}
					nk + nm &= (a - b) + (b - c)\\
					n(k + m) &= a - c
				\end{align*}
				Por lo tanto, $ a\sim c $.			
			}<3>
		\end{itemize}
	\end{proof}
\end{frame}
\begin{frame}{Módulos}
	Dos números relacionados de esta forma se dicen \emph{iguales módulo} $ n $.
	\begin{mdefinition}[Módulo]
		Sean $ a, b, n\in\zmath $. Decimos que $ a = b $ módulo $ n $, denotado por $ a = b\mod n $, si y solo si $ a\sim b $ con la relación de equivalencia descrita anteriormente.
	\end{mdefinition}
\end{frame}
\begin{frame}
	
\end{frame}
\begin{frame}{Relación mediante una función}
	\begin{exercise}
		Sean $ f:\rmath\to\rmath $ y $ a, b\in\mdom{f} $. $ a\sim b $ si y solo si $ f(a) = f(b) $.\par 
		Demuestre que esta relación es de equivalencia.
	\end{exercise}
\end{frame}
\begin{frame}{Demostración}
	\begin{proof}
		\begin{itemize}
			\only{
				\item Reflexividad\\
				Sea $ a\in\mdom{f} $. Claramente $ f(a) = f(a) $ y entonces $ a\sim a $.
			}<1>
			\only{
				\item Simetría\\
				Sean $ a, b\in\mdom{f} $ tales que $ a\sim b $, entonces $ f(a) = f(b) $. Dado que $ f(b) = f(a) $, entonces $ b\sim a $.
			}<2>
			\only{
				\item Transitividad\\
				Sean $ a, b, c\in\mdom{f} $ tales que $ a\sim b $ y $ b\sim c $, entonces $ f(a) = f(b) $ y $ f(b) = f(c) $. Por lo tanto, $ f(a) = f(c) $ y entonces $ a\sim c $.
			}<3>
		\end{itemize}
	\end{proof}
\end{frame}
\begin{frame}
	
\end{frame}
