\section{Lógica proposicional}
\begin{definition}[Proposición]
	Una \emph{proposición} es un conjunto de símbolos con un significado (tales como una oración, afirmación o expresión matemática) el cual puede determinarse como \emph{verdadero} o \emph{falso}, pero no ambas.
\end{definition}
Ejemplos de \emph{proposiciones} pueden ser las siguientes:
\begin{enumerate}
	\item $2 + 2 = 7$
	\item El día de hoy está soleado.
	\item Esto es un texto.
	\item Existe un infinito número de primos.
\end{enumerate}
pues cada una puede determinarse como verdadera o falsa, pero no ambas.
Hay algunas otras oraciones o expresiones que no se les puede determinar si son verdaderas o falsas, y por tanto no pueden considerarse proposiciones:
\begin{enumerate}
	\item ¿Qué día es hoy?
	\item ¡Vamos!
	\item Tú y yo.
	\item $+3* = /1-$
\end{enumerate}
Las oraciones interrogativas e imperativas (como las primeras dos oraciones) no se les puede asignar un \emph{valor de verdad} (verdadero o falso), por lo que no se les puede denominar proposiciones.\par 
En el caso de ``tú y yo'', aunque es una oración declarativa, dado que no es posible y no tiene sentido asignarle un valor de verdad, entonces no puede ser denominada proposición tampoco.\par
En el último caso de la expresión $+3* = /1-$, dado que no tiene sentido y por tanto no tiene un significado, no se le puede asignar un valor de verdad. Por lo tanto, no es una proposición.\par 
Un ejemplo muy peculiar y clásico de oración que parecería ser una proposición, pero que en realidad no lo es, es:
\begin{center}
	``Esta afirmación es falsa''
\end{center}
Supongamos que fuera verdadera, entonces por su afirmación sería falsa, por lo que no sería verdadera. Supongamos lo contrario, que es falsa, entonces afirmaría que no es falsa, es decir, verdadera, por lo que no sería falsa. Dado que nos lleva a contradicciones (algo que es pero no es al mismo tiempo), no es posible asignarle un valor de verdad y entonces no sería una proposición.\par 
Debemos tomar en cuenta que la definición dada de proposición no es muy precisa y vaga. Sin embargo, nos es útil para introducirnos al razonamiento matemático y en realidad lo esencial es poder determinar cuando una afirmación o expresión es verdadera o falsa, pues nos ayudará para poder hacer deducciones más adelante.
\begin{notation}
	Para denotar a una proposición, haremos uso de literales como $p, q, r, s,\ldots$.\par 
	Para denotar específicamente a qué proposición hace referencia cada literal haremos uso de la notación:
	\begin{center}
		$p:$ ``proposición''
	\end{center}
	Por ejemplo:
	\begin{center}
		$s:$ ``América es un continente''
	\end{center}
	significa que la literal $s$ hará referencia a la proposición ``América es un continente''.\par 
	Si no se especifica a qué proposición en particular una literal hace referencia, esta podrá hacer referencia a cualquier proposición y se denominará \emph{variable lógica}. Por lo general, literales distintas harán referencia a proposiciones distintas, a menos que se especifique lo contrario.
\end{notation}
\subsection{Conectores lógicos}
Para el estudio de las proposiciones es útil clasificar a estas en dos grupos: \emph{simples} o \emph{atómicas} y \emph{compuestas}.\par 
Las \emph{proposiciones simples} o \emph{atómicas} son aquellas que no pueden dividirse en proposiciones ``más pequeñas''; su valor de verdad no depende de otras proposiciones. Por ejemplo, considere la siguiente proposición:
\begin{center}
	$p:$ ``Hoy está soleado pero hace frío''
\end{center}
La proposición $p$ puede dividirse en dos proposiciones más pequeñas, las cuales son:
\begin{center}
	$p_{1}:$ ``Hoy está soleado''\\
	$p_{2}:$ ``Hace frío''
\end{center}
Dado que $p$ puede dividirse en $p_{1}$ y $p_{2}$, las cuales son proposiciones ``más pequeñas'', $p$ no se considera atómica. Sin embargo, las proposiciones $p_{1}$ y $p_{2}$ no pueden dividirse en proposiciones más pequeñas, por lo que estas si se consideran atómicas o simples.\par 
Por otro lado, las \emph{proposiciones compuestas} son aquellas que están conformadas de varias proposiciones atómicas, como el caso de $p$ que se conformaba de $p_{1}$ y $p_{2}$. Otro ejemplo es:
\begin{center}
	$q:$ ``Él sabe cocinar y yo sé patinar''
\end{center}
la cual puede dividirse en dos \emph{subproposiciones}:
\begin{center}
	$q_{1}:$ ``Él sabe cocinar''\\
	$q_{2}:$ ``Yo sé patinar''
\end{center}
Otros ejemplos de proposiciones compuestas son las siguientes:
\begin{enumerate}
	\item Un número real es o racional o irracional.
	\item La tierra no es plana pero si redonda.
	\item Si estudiar para el examen, lo apruebas.
	\item No eres ni malo ni bueno.
\end{enumerate}
Hay que notar que para unir proposiciones atómicas y así formar proposiciones compuestas hacemos uso de ciertas palabras como \emph{y}, \emph{o}; \emph{si$\ldots$, entonces$\ldots$}; \emph{o$\ldots$ o$\ldots$}, etc. A este tipo de palabras que nos ayudan a formar proposiciones compuestas se les denomina \emph{conectores lógicos}. Aunque se pueden enumerar varios conectores lógicos, los tres más básicos que podemos usar para formar proposiciones compuestas son la \emph{negación}, \emph{disyunción} y \emph{conjunción}.