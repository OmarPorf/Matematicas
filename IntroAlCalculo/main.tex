%%%%%%%%%%%%%%%%%%%%%%%%%%%%%%%%%%%%%%%%%
% The Legrand Orange Book
% LaTeX Template
% Version 2.0 (9/2/15)
%
% This template has been downloaded from:
% http://www.LaTeXTemplates.com
%
% Mathias Legrand (legrand.mathias@gmail.com) with modifications by:
% Vel (vel@latextemplates.com)
%
% License:
% CC BY-NC-SA 3.0 (http://creativecommons.org/licenses/by-nc-sa/3.0/)
%
% Compiling this template:
% This template uses biber for its bibliography and makeindex for its index.
% When you first open the template, compile it from the command line with the 
% commands below to make sure your LaTeX distribution is configured correctly:
%
% 1) pdflatex main
% 2) makeindex main.idx -s StyleInd.ist
% 3) biber main
% 4) pdflatex main x 2
%
% After this, when you wish to update the bibliography/index use the appropriate
% command above and make sure to compile with pdflatex several times 
% afterwards to propagate your changes to the document.
%
% This template also uses a number of packages which may need to be
% updated to the newest versions for the template to compile. It is strongly
% recommended you update your LaTeX distribution if you have any
% compilation errors.
%
% Important note:
% Chapter heading images should have a 2:1 width:height ratio,
% e.g. 920px width and 460px height.
%
%%%%%%%%%%%%%%%%%%%%%%%%%%%%%%%%%%%%%%%%%

%----------------------------------------------------------------------------------------
%	PACKAGES AND OTHER DOCUMENT CONFIGURATIONS
%----------------------------------------------------------------------------------------


%----------------------------------------------------------------------------------------

%%%%%%%%%%%%%%%%%%%%%%%%%%%%%%%%%%%%%%%%%
% The Legrand Orange Book
% Structural Definitions File
% Version 2.0 (9/2/15)
%
% Original author:
% Mathias Legrand (legrand.mathias@gmail.com) with modifications by:
% Vel (vel@latextemplates.com)
% 
% This file has been downloaded from:
% http://www.LaTeXTemplates.com
%
% License:
% CC BY-NC-SA 3.0 (http://creativecommons.org/licenses/by-nc-sa/3.0/)
%
%%%%%%%%%%%%%%%%%%%%%%%%%%%%%%%%%%%%%%%%%

%----------------------------------------------------------------------------------------
%	VARIOUS REQUIRED PACKAGES AND CONFIGURATIONS
%----------------------------------------------------------------------------------------
\documentclass[11pt,a4paper]{beamer}
\usetheme{CambridgeUS}
\usecolortheme{dolphin}
\usepackage[spanish]{babel}				% Palabras reservadas traducidas al español 
\usepackage[utf8]{inputenc}				% Para poder imprimir caracteres especiales
\usepackage{fontenc}					% Para acentos
\usepackage{nameref}
\usepackage{translator}
%%%%%%%%%%%%%%%%%%%%%%
% Fuentes, funciones y teoremas matemáticos
\usepackage{amsmath}					% Carga los entornos matemáticos
\usepackage{amsfonts}					% Carga las fuentes matemáticas
\usepackage{amssymb}					% Carga caracteres matemáticos especiales
\usepackage{amsthm}						% Crea entornos de teoremas, definiciones, etc.
\usepackage{mdframed}					% Sirve para encerrar los teoremas de amsthm en cajas para darle un estilo más personalizado
\usepackage{mathtools}					% Extensión de símbolos matemáticos
\usepackage{cases}						% Para crear el entorno de definición de una expresión por casos
\usepackage{cancel}						% Permite tachar expresiones matemáticas			
\usepackage{leftindex}					% Para poder añadir subíndices y superíndices del lado izquierdo de una expresión
\usepackage[italic]{derivative}					% Extensión de las funciones de derivadas
\usepackage[intlimits]{esint}			% Extensión de símbolos de integrales
\usepackage{thmtools}
\spanishdecimal{.}
%%%%%%%%%%%%%%%%%%%%%%

\usepackage{listings}					% Para ingresar código fuente en diferentes lenguajes de programación
\usepackage{pifont}						% Inserta doodles

%%%%%%%%%%%%%%%%%%%%%%
% Para insertar pseudocódigo
\usepackage{algorithm}
\usepackage{algorithmic}
%%%%%%%%%%%%%%%%%%%%%%

\usepackage{xcolor}						% Permite modificar y definir colores
\usepackage{multicol}					% Permite crear múltiples columnas de texto en el documento
\usepackage{hyperref}
%%%%%%%%%%%%%%%%%%%%%%
% Para insertar y crear figuras
\usepackage{graphicx}					% Para insertar gráficos
\usepackage{tikz}						% Para crear figuras
\usepackage{venndiagram}				% Extensión de Tikz para crear diagramas de Venn
%%%%%%%%%%%%%%%%%%%%%%

\usepackage[many]{tcolorbox}			% Para poder crear cajas de color
\tcbuselibrary{theorems}					
\usepackage{enumerate}					% Para poder modificar el estilo de los items en el entorno enumerate
%%%%%%%%%%%%%%%%%%%%%%
\usepackage{ifthen}
\usepackage{xargs}
\usetikzlibrary{automata, arrows.meta, positioning}
\input{Preambulo/colors.tex}
\documentclass[10pt, a4paper]{article}
\usepackage[spanish]{babel}
%%%%%%%%%%%%%%%%%%%%%%
% Fuentes, funciones y teoremas matemáticos
\usepackage{amsmath}					% Carga los entornos matemáticos
\usepackage{amsfonts}					% Carga las fuentes matemáticas
\usepackage{amssymb}					% Carga caracteres matemáticos especiales
\usepackage{amsthm}						% Crea entornos de teoremas, definiciones, etc.
\usepackage{mdframed}					% Sirve para encerrar los teoremas de amsthm en cajas para darle un estilo más personalizado
\usepackage{mathtools}					% Extensión de símbolos matemáticos
\usepackage{cases}						% Para crear el entorno de definición de una expresión por casos
\usepackage{cancel}						% Permite tachar expresiones matemáticas			
\usepackage{leftindex}					% Para poder añadir subíndices y superíndices del lado izquierdo de una expresión
\usepackage[italic]{derivative}					% Extensión de las funciones de derivadas
%\usepackage[intlimits]{esint}			% Extensión de símbolos de integrales
\usepackage{thmtools}
%%%%%%%%%%%%%%%%%%%%%%
\usepackage[margin = 1in]{geometry}
%opening
\title{Comandos personalizados: matem\'{a}ticas}
\author{Omar Porfirio Garc\'{\i}a}

\usepackage{ifthen}
\usepackage{xargs}
%%%%%%%%%%%%%%%%%%%%%%%%%%%%%%%%%%%%%%%%%%%%%
\newcommand{\ds}{\displaystyle}
%%%%%%%%%%%%%%%%%%%%%%%%%%%%%%%%%%%%%%%%%%%%%
	% Conjuntos numéricos
\newcommand{\cmath}[1][]{\mathbb{C}^{#1}}
\newcommand{\rmath}[1][]{\mathbb{R}^{#1}}
\newcommand{\qmath}[1][]{\mathbb{Q}^{#1}}
\newcommand{\irrmath}[1][]{\mathbb{I}^{#1}}
\newcommand{\zmath}[1][]{\mathbb{Z}^{#1}}
\newcommand{\nmath}[1][]{\mathbb{N}^{#1}}
\newcommand{\pmath}[1][]{\mathbb{P}^{#1}}
\newcommand{\fmath}[1][]{\mathbb{F}^{#1}}
\newcommand{\hmath}[1][]{\mathbb{H}^{#1}}
\newcommand{\umath}[1][]{\mathbb{U}^{#1}}
%%%%%%%%%%%%%%%%%%%%%%%%%%%%%%%%%%%%%%%%%%%%%
	% Delimitadores
\newcommand{\pa}[1]{\left( #1 \right)}
\newcommand{\bracket}[1]{\left[ #1 \right]}
\newcommand{\set}[1]{\left\{ #1 \right\}}
\newcommand{\abs}[1]{\left\vert #1 \right\vert}
\newcommand{\norm}[1]{\left\Vert #1 \right\Vert}
\newcommand{\inprod}[1]{\left\langle #1 \right\rangle}
\newcommand{\floor}[1]{\left\lfloor #1 \right\rfloor}
\newcommand{\ceil}[1]{\left\lceil #1 \right\rceil}
\newcommand{\upcor}[1]{\left\ulcorner #1 \right\urcorner}
\newcommand{\lowcor}[1]{\left\llcorner #1 \right\lrcorner}
\newcommand{\lopen}[1]{\left( #1 \right]}
\newcommand{\ropen}[1]{\left[ #1 \right)}
\newcommand{\Lopen}[1]{\left] #1 \right]}
\newcommand{\Ropen}[1]{\left[ #1 \right[}
%%%%%%%%%%%%%%%%%%%%%%%%%%%%%%%%%%%%%%%%%%%%%
	% funciones estándar
\newcommandx{\minf}[3][1 = 0, 3 = { }, usedefault]{
	\ifthenelse{ \equal{#1}{0}}{
		\inf_{#3}\left( #2 \right)
	}{
		\inf_{#3}{#2} 
	}
}
\newcommandx{\msup}[3][1 = 0, 3 = { }, usedefault]{
	\ifthenelse{ \equal{#1}{0} }{
		\sup_{#3}\left( #2 \right)
	}{
		\sup_{#3}{#2}
	}
}
\newcommandx{\msin}[3][1 = 0, 3 = { }, usedefault]{
	\ifthenelse{ \equal{#1}{0} }{
		\sin^{#3}\left( #2 \right)
	}{
		\sin^{#3}{#2}
	}
}
\newcommandx{\mcos}[3][1 = 0, 3 = { }, usedefault]{
	\ifthenelse{ \equal{#1}{0} }{
		\cos^{#3}\left( #2 \right)
	}{
		\cos^{#3}{#2}
	}
}
\newcommandx{\mtan}[3][1 = 0, 3 = { }, usedefault]{
	\ifthenelse{ \equal{#1}{0} }{
		\tan^{#3}\left( #2 \right)
	}{
		\tan^{#3}{#2}
	}
}
\newcommandx{\msec}[3][1 = 0, 3 = { }]{
	\ifthenelse{ \equal{#1}{0} }{
		\sec^{#3}\left( #2 \right)
	}{
		\sec^{#3}{#2}
	}
}
\newcommandx{\mcsc}[3][1 = 0, 3 = { }]{
	\ifthenelse{ \equal{#1}{0} }{
		\csc^{#3}\left( #2 \right)
	}{
		\csc^{#3}{#2}
	}
}
\newcommandx{\mcot}[3][1 = 0, 3 = { }]{
	\ifthenelse{ \equal{#1}{0} }{
		\cot^{#3}\left( #2 \right)
	}{
		\cot^{#3}{#2}
	}
}
\newcommandx{\marcsin}[3][1 = 0, 3 = { }, usedefault]{
	\ifthenelse{ \equal{#1}{0} }{
		\arcsin^{#3}\left( #2 \right)
	}{
		\arcsin^{#3}{#2}
	}
}
\newcommandx{\marccos}[3][1 = 0, 3 = { }, usedefault]{
	\ifthenelse{ \equal{#1}{0} }{
		\arccos^{#3}\left( #2 \right)
	}{
		\arccos^{#3}{#2}
	}
}
\newcommandx{\marctan}[3][1 = 0, 3 = { }, usedefault]{
	\ifthenelse{ \equal{#1}{0} }{
		\arctan^{#3}\left( #2 \right)
	}{
		\arctan^{#3}{#2}
	}
}
\newcommandx{\marcsec}[3][1 = 0, 3 = { }, usedefault]{
	\ifthenelse{ \equal{#1}{0} }{
		\operatorname{arcsec}^{#3}\left( #2 \right)
	}{
		\operatorname{arcsec}^{#3}{#2}
	}
}
\newcommandx{\marccsc}[3][1 = 0, 3 = { }, usedefault]{
	\ifthenelse{ \equal{#1}{0} }{
		\operatorname{arccsc}^{#3}\left( #2 \right)
	}{
		\operatorname{arccsc}^{#3}{#2}
	}
}
\newcommandx{\marccot}[3][1 = 0, 3 = { }, usedefault]{
	\ifthenelse{ \equal{#1}{0} }{
		\operatorname{arccot}^{#3}\left( #2 \right)
	}{
		\operatorname{arccot}^{#3}{#2}
	}
}
\newcommand{\marg}[2][0]{
	\ifthenelse{ \equal{#1}{0} }{\arg\left( #2 \right)}{\arg{#2}}
}
\newcommandx{\mdeg}[3][1 = 0, 3 = { }]{
	\ifthenelse{ \equal{#1}{0}}{\deg_{#3}\left( #2 \right)}{\deg_{#3}{#2}}
}
\newcommandx{\mdet}[3][1 = 0, 3 = { }, usedefault]{
	\ifthenelse{ \equal{#1}{0} }{\det_{#3}\left( #2 \right)}{\det_{#3}{#2}}
}
\newcommand{\mdim}[2][0]{
	\ifthenelse{ \equal{#1}{0} }{\dim\left( #2 \right)}{\dim{#2}}
}
\newcommand{\mexp}[2][0]{
	\ifthenelse{ \equal{#1}{0} }{\exp\left( #2 \right)}{\exp{#2}}
}
\DeclareMathOperator*{\mcd}{mcd}
\newcommandx{\mmcd}[3][1 = 0, 3 = { }, usedefault]{
	\ifthenelse{ \equal{#1}{0} }{\mcd_{#3}\left( #2 \right)}{\mcd_{#3}{#2}}
}
\newcommand{\mln}[2][0]{
	\ifthenelse{ \equal{#1}{0} }{\ln\left( #2 \right)}{\ln{#2}}
}
\newcommandx{\mlog}[3][1 = 0, 3 = { }, usedefault]{
	\ifthenelse{ \equal{#1}{0} }{\log_{#3}\left( #2 \right)}{\log_{#3}{#2}}
}
\newcommandx{\mmax}[3][1 = 0, 3 = { }, usedefault]{
	\ifthenelse{ \equal{#1}{0} }{\max_{#3}\left( #2 \right)}{\max_{#3}{#2}}
}
\newcommandx{\mmin}[3][1 = 0, 3 = { }, usedefault]{
	\ifthenelse{ \equal{#1}{0} }{\min_{#3}\left( #2 \right)}{\min_{#3}{#2}}
}
%%%%%%%%%%%%%%%%%%%%%%%%%%%%%%%%%%%%%%%%%%%%%
	% Otras funciones
\newcommand{\mpow}[2][0]{
	\ifthenelse{ \equal{#1}{0} }{\mathfrak{P}\left( #2 \right)}{\mathfrak{P}{#2}}
}
\newcommand{\mnu}[2][0]{
	\ifthenelse{ \equal{#1}{0} }{\nu\left( #2 \right)}{\nu{#2}}
}
\newcommand{\mrho}[2][0]{
	\ifthenelse{ \equal{#1}{0} }{\rho\left( #2 \right)}{\rho{#2}}
}
\newcommand{\mnuc}[2][0]{
	\ifthenelse{ \equal{#1}{0} }{\operatorname{N}\left( #2 \right)}{\operatorname{N}{#2}}
}
\newcommand{\mgen}[2][0]{
	\ifthenelse{ \equal{#1}{0} }{\operatorname{gen}\left( #2 \right)}{\operatorname{gen}{#2}}
}
\newcommand{\mtr}[2][0]{
	\ifthenelse{ \equal{#1}{0} }{\operatorname{tr}\left( #2 \right)}{\operatorname{tr}{#2}}
}
\newcommand{\mdom}[2][0]{
	\ifthenelse{ \equal{#1}{0} }{\operatorname{Dom}\left( #2 \right)}{\operatorname{Dom}{#2}}
}
\newcommand{\mran}[2][0]{
	\ifthenelse{ \equal{#1}{0} }{\operatorname{Ran}\left( #2 \right)}{\operatorname{Ran}{#2}}
}
\newcommand{\mim}[2][0]{
	\ifthenelse{ \equal{#1}{0} }{\operatorname{Im}\left( #2 \right)}{\operatorname{Im}{#2}}
}
\newcommand{\ipart}[2][0]{
	\ifthenelse{ \equal{#1}{0} }{\Im\left( #2 \right)}{\Im{#2}}
}
\newcommand{\rpart}[2][0]{
	\ifthenelse{ \equal{#1}{0} }{\Re\left( #2 \right)}{\Re{#2}}
}
\newcommand{\mprob}[2][0]{
	\ifthenelse{ \equal{#1}{0} }{\mathbb{P}\left( #2 \right)}{\mathbb{P}{#2}}
}
\newcommand{\mmean}[2][0]{
	\ifthenelse{ \equal{#1}{0} }{\mathbb{E}\left( #2 \right)}{\mathbb{E}{#2}}
}
\newcommand{\mvar}[2][0]{
	\ifthenelse{ \equal{#1}{0} }{\mathbb{V}\left( #2 \right)}{\mathbb{V}{#2}}
}
\DeclareMathOperator*{\mcm}{mcm}
\newcommandx{\mmcm}[3][1 = 0, 3 = { }, usedefault]{
	\ifthenelse{ \equal{#1}{0} }{\mcm_{#3}\left( #2 \right)}{\mcm_{#3}{#2}}
}
\newcommandx{\mgrad}[3][1 = 0, 3 = { }]{
	\ifthenelse{ \equal{#1}{0} }{\operatorname{grad}_{#3}\left( #2 \right)}{\operatorname{grad}_{#3}{#2}}
}
\newcommand{\mleng}[2][0]{
	\ifthenelse{ \equal{#1}{0} }{\ell\left( #2 \right)}{\ell{#2}}
}
\DeclareMathOperator*{\inte}{int}
\newcommandx{\minte}[3][1 = 0, 3 = { }, usedefault]{
	\ifthenelse{ \equal{#1}{0} }{\inte_{#3}\left( #2 \right)}{\inte_{#3}{#2}}
} 
\DeclareMathOperator*{\exte}{ext}
\newcommandx{\mexte}[3][1 = 0, 3 = { }, usedefault]{
	\ifthenelse{ \equal{#1}{0} }{\exte_{#3}\left( #2 \right)}{\exte_{#3}{#2}}
}
\DeclareMathOperator*{\fr}{Fr}
\newcommandx{\mfr}[3][1 = 0, 3 = { }, usedefault]{
	\ifthenelse{ \equal{#1}{0} }{\fr_{#3}\left(#2\right)}{\fr_{#3}{#2}}
}
\DeclareMathOperator*{\cl}{cl}
\newcommandx{\mcl}[3][1 = 0, 3 = { }, usedefault]{
	\ifthenelse{ \equal{#1}{0} }{\cl_{#3}\left( #2 \right)}{\cl_{#3}{#2}}
}
\DeclareMathOperator*{\der}{der}
\newcommandx{\mder}[3][1 = 0, 3 = { }, usedefault]{
	\ifthenelse{ \equal{#1}{0} }{\der_{#3}\left( #2 \right)}{\der_{#3}{#2}}
}
\newcommand{\mtrans}[2][0]{
	\ifthenelse{ \equal{#1}{0} }{{#2}^{\mathsf{T}}}{\left( #2 \right)^{\mathsf{T}}}
}
\newcommand{\mcomp}[2][0]{
	\ifthenelse{ \equal{#1}{0} }{{#2}^{\mathsf{C}}}{\left( #2 \right)^{\mathsf{C}}}
}
\newcommand{\minv}[2][0]{
	\ifthenelse{ \equal{#1}{0} }{{#2}^{-1}}{\left( #2 \right)^{-1}}
}
\newcommand{\mperp}[2][0]{
	\ifthenelse{ \equal{#1}{0} }{{#2}^{\perp}}{\left( #2 \right)^{\perp}}
}
\newcommand{\madj}[2][0]{
	\ifthenelse{ \equal{#1}{0} }{{#2}^{*}}{\left( #2 \right)^{*}}
}
\newcommand{\mproj}[2]{\operatorname{proj}_{#2}\left(#1\right)}
\newcommandx{\meval}[3][2 = { }, 3 = { }]{
	\left[ #1 \right]_{#2}^{#3}
}
\newcommandx{\mint}[5][1 = 0, 4 = { }, 5 = { }]{
	\ifthenelse{ \equal{#1}{0} }{
		\int_{#4}^{#5}{#2}\ d{#3}
	}{
		\int_{#4}^{#5}\left( #2 \right)d{#3}
	}
}
\newcommand{\mpol}[2]{\mathcal{P}_{#2}\left( #1 \right)}
\newcommand{\mmatrix}[2]{\mathcal{M}_{#2}\left( #1 \right)}
\newcommand{\mlin}[1]{\mathcal{L}\left( #1 \right)}
\newcommand{\mfunc}[1]{\mathcal{F}\left( #1 \right)}
\newcommand{\mdis}[1]{\operatorname{d}\left( #1 \right)}
\newcommand{\msinf}[1]{\operatorname{\underline{S}}\left( #1 \right)}
\newcommand{\mssup}[1]{\operatorname{\overline{S}}\left( #1 \right)}
\newcommand{\mmeas}[1]{\operatorname{m}\left( #1 \right)}
\newcommand{\mclass}[1][]{\mathcal{C}^{#1}}
%%%%%%%%%%%%%%%%%%%%%%%%%%%%%%%%%%%%%%%%%%%%%
	% Otros símbolos
\newcommand{\void}{\varnothing}
\newcommand{\true}{\top}
\newcommand{\false}{\bot}
\newcommand{\ivec}{\hat{\imath}}
\newcommand{\jvec}{\hat{\jmath}}
\newcommand{\kvec}{\hat{k}}
%%%%%%%%%%%%%%%%%%%%%%%%%%%%%%%%%%%%%%%%%%%%%
\begin{document}

\maketitle

\begin{abstract}
Aqu\'{\i} se presentan los comandos personalizados para símbolos y funciones matemáticas que no se encuentran en los paquetes AMS.
\end{abstract}
Estos comandos hacen uso de comandos de los paquetes de AMS, \emph{ifthen} y \emph{xargs}.
\section{Conjuntos de n\'{u}meros}
Estos son de la forma:
\begin{center}
	\ttfamily
	\textbackslash numero[$ \inprod{complemento} $]
\end{center}
donde el \emph{complemento} es un super\'{\i}ndice.
\begin{center}
	\begin{tabular}{ll|ll}
		$ \cmath $ & \texttt{\textbackslash cmath} & $ \cmath[n] $ & \texttt{\textbackslash cmath[n]} \\
		$ \rmath $ & \texttt{\textbackslash rmath} & $ \rmath[n] $ & \texttt{\textbackslash rmath[n]}\\
		$ \qmath $ & \texttt{\textbackslash qmath} & $ \qmath[n] $ & \texttt{\textbackslash qmath[n]}\\
		$ \irrmath $ & \texttt{\textbackslash irrmath} & $ \irrmath[n] $ & \texttt{\textbackslash irrmath[n]}\\
		$ \zmath $ & \texttt{\textbackslash zmath} & $ \zmath[n] $ & \texttt{\textbackslash zmath[n]}\\
		$ \nmath $ & \texttt{\textbackslash nmath} & $ \nmath[n] $ & \texttt{\textbackslash nmath[n]}\\
		$ \pmath $ & \texttt{\textbackslash pmath} & $ \pmath[n] $ & \texttt{\textbackslash pmath[n]}\\
		$ \fmath $ & \texttt{\textbackslash fmath} & $ \fmath[n] $ & \texttt{\textbackslash fmath[n]}\\
		$ \hmath $ & \texttt{\textbackslash hmath} & $ \hmath[n] $ & \texttt{\textbackslash hmath[n]}\\
		$ \umath $ & \texttt{\textbackslash umath} & $ \umath[n] $ & \texttt{\textbackslash umath[n]}
	\end{tabular}
\end{center}
\section{Delimitadores}
\begin{center}
	\begin{tabular}{ll|ll}
		$ \pa{abc} $ & \texttt{\textbackslash pa\{abc\}} & $ \inprod{abc} $ & \texttt{\textbackslash inprod\{abc\}}\\
		$ \bracket{abc} $ & \texttt{\textbackslash bracket\{abc\}} & $ \floor{abc} $ & \texttt{\textbackslash floor\{abc\}}\\
		$ \set{abc} $ & \texttt{\textbackslash set\{abc\}} & $ \ceil{abc} $ & \texttt{\textbackslash ceil\{abc\}}\\
		$ \abs{abc} $ & \texttt{\textbackslash abs\{abc\}} & $ \upcor{abc} $ & \texttt{\textbackslash upcor\{abc\}}\\
		$ \norm{abc} $ & \texttt{\textbackslash norm\{abc\}} & $ \lowcor{abc} $ & \texttt{\textbackslash lowcor\{abc\}}\\
		$ \lopen{abc} $ & \texttt{\textbackslash lopen\{abc\}} & $ \ropen{abc} $ & \texttt{\textbackslash ropen\{abc\}}\\
		$ \Lopen{abc} $ & \texttt{\textbackslash Lopen\{abc\}} & $ \Ropen{abc} $ & \texttt{\textbackslash Ropen\{abc\}}
	\end{tabular}
\end{center}
Los delimitadores la incluyen el par \emph{left} y \emph{right}.
\section{Funciones est\'{a}ndar}
En estas la mayor\'{\i}a tiene la forma:
\begin{center}
	\ttfamily
	\textbackslash funcion[$ \inprod{delimitador} $]\{$ \inprod{argumento} $\}[$ \inprod{complemento} $]
\end{center}
donde el \emph{delimitador} indica si el argumento de la función estará delimitado por un paréntesis o no (incluye el par \emph{left} y \emph{right}). Esto se indica con $ 0 $ o vac\'{\i}o si será delimitado y con algún otro valor (de preferencia $ 1 $) si no.\par 
El \emph{complemento} es un sub\'{\i}ndice o super\'{\i}ndice según sea la función.
\begin{center}
	\begin{tabular}{ll|ll}
		$ \minf{A} $ & \texttt{\textbackslash minf\{A\}} & $ \minf[1]{A} $ & \texttt{\textbackslash minf[1]\{A\}} \\
		$ \ds\minf{f(x)}[x\in A] $ & \texttt{\textbackslash minf\{f(x)\}[x\textbackslash in A]} & $ \ds\minf[1]{f(x)}[x\in A] $ & \texttt{\textbackslash minf[1]\{f(x)\}[x\textbackslash in A]}\\
		$ \msup{A} $ & \texttt{\textbackslash msup\{A\}} & $ \msup[1]{A} $ & \texttt{\textbackslash msup[1]\{A\}} \\
		$ \ds\msup{f(x)}[x\in A] $ & \texttt{\textbackslash msup\{f(x)\}[x\textbackslash in A]} & $ \ds\msup[1]{f(x)}[x\in A] $ & \texttt{\textbackslash msup[1]\{f(x)\}[x\textbackslash in A]}\\
		$ \msin{x} $ & \texttt{\textbackslash msin\{x\}} & $ \msin[1]{x} $ & \texttt{\textbackslash msin[1]\{x\}}\\
		$ \msin{x}[n] $ & \texttt{\textbackslash msin\{x\}[n]} & $ \msin[1]{x}[n] $ & \texttt{\textbackslash msin[1]\{x\}[n]}\\
		$ \mcos{x} $ & \texttt{\textbackslash mcos\{x\}} & $ \mcos[1]{x} $ & \texttt{\textbackslash mcos[1]\{x\}}\\
		$ \mcos{x}[n] $ & \texttt{\textbackslash mcos\{x\}[n]} & $ \mcos[1]{x}[n] $ & \texttt{\textbackslash mcos[1]\{x\}[n]}\\
		$ \mtan{x} $ & \texttt{\textbackslash mtan\{x\}} & $ \mtan[1]{x} $ & \texttt{\textbackslash mtan[1]\{x\}}\\
		$ \mtan{x}[n] $ & \texttt{\textbackslash mtan\{x\}[n]} & $ \mtan[1]{x}[n] $ & \texttt{\textbackslash mtan[1]\{x\}[n]}\\
		$ \msec{x} $ & \texttt{\textbackslash msec\{x\}} & $ \msec[1]{x} $ & \texttt{\textbackslash msec[1]\{x\}}\\
		$ \msec{x}[n] $ & \texttt{\textbackslash msec\{x\}[n]} & $ \msec[1]{x}[n] $ & \texttt{\textbackslash msec[1]\{x\}[n]}\\
		$ \mcsc{x} $ & \texttt{\textbackslash mcsc\{x\}} & $ \mcsc[1]{x} $ & \texttt{\textbackslash mcsc[1]\{x\}}\\
		$ \mcsc{x}[n] $ & \texttt{\textbackslash mcsc\{x\}[n]} & $ \mcsc[1]{x}[n] $ & \texttt{\textbackslash mcsc[1]\{x\}[n]}\\
		$ \mcot{x} $ & \texttt{\textbackslash mcot\{x\}} & $ \mcot[1]{x} $ & \texttt{\textbackslash mcot[1]\{x\}}\\
		$ \mcot{x}[n] $ & \texttt{\textbackslash mcot\{x\}[n]} & $ \mcot[1]{x}[n] $ & \texttt{\textbackslash mcot[1]\{x\}[n]}\\
		$ \marcsin{x} $ & \texttt{\textbackslash marcsin\{x\}} & $ \marcsin[1]{x} $ & \texttt{\textbackslash marcsin[1]\{x\}}\\
		$ \marcsin{x}[n] $ & \texttt{\textbackslash marcsin\{x\}[n]} & $ \marcsin[1]{x}[n] $ & \texttt{\textbackslash marcsin[1]\{x\}[n]}\\
		$ \marccos{x} $ & \texttt{\textbackslash marccos\{x\}} & $ \marccos[1]{x} $ & \texttt{\textbackslash marccos[1]\{x\}}\\
		$ \marccos{x}[n] $ & \texttt{\textbackslash marccos\{x\}[n]} & $ \marccos[1]{x}[n] $ & \texttt{\textbackslash marccos[1]\{x\}[n]}\\
		$ \marctan{x} $ & \texttt{\textbackslash marctan\{x\}} & $ \marctan[1]{x} $ & \texttt{\textbackslash marctan[1]\{x\}}\\
		$ \marctan{x}[n] $ & \texttt{\textbackslash marctan\{x\}[n]} & $ \marctan[1]{x}[n] $ & \texttt{\textbackslash marctan[1]\{x\}[n]}\\
		$ \marcsec{x} $ & \texttt{\textbackslash marcsec\{x\}} & $ \marcsec[1]{x} $ & \texttt{\textbackslash marcsec[1]\{x\}}\\
		$ \marcsec{x}[n] $ & \texttt{\textbackslash marcsec\{x\}[n]} & $ \marcsec[1]{x}[n] $ & \texttt{\textbackslash marcsec[1]\{x\}[n]}\\
		$ \marccsc{x} $ & \texttt{\textbackslash marccsc\{x\}} & $ \marccsc[1]{x} $ & \texttt{\textbackslash marccsc[1]\{x\}}\\
		$ \marccsc{x}[n] $ & \texttt{\textbackslash marccsc\{x\}[n]} & $ \marccsc[1]{x}[n] $ & \texttt{\textbackslash marccsc[1]\{x\}[n]}\\
		$ \marccot{x} $ & \texttt{\textbackslash marccot\{x\}} & $ \marccot[1]{x} $ & \texttt{\textbackslash marccot[1]\{x\}}\\
		$ \marccot{x}[n] $ & \texttt{\textbackslash marccot\{x\}[n]} & $ \marccot[1]{x}[n] $ & \texttt{\textbackslash marccot[1]\{x\}[n]}\\
		$ \marg{x} $ & \texttt{\textbackslash marg\{x\}} & $ \marg[1]{x} $ & \texttt{\textbackslash marg[1]\{x\}}\\
		$ \mdeg{x} $ & \texttt{\textbackslash mdeg\{x\}} & $ \mdeg[1]{x} $ & \texttt{\textbackslash mdeg[1]\{x\}}\\
		$ \mdeg{v}[G] $ & \texttt{\textbackslash mdeg\{v\}[G]} & $ \mdeg[1]{v}[G] $ & \texttt{\textbackslash mdeg[1]\{v\}[G]}\\
		$ \mdet{A} $ & \texttt{\textbackslash mdet\{A\}} & $ \mdet[1]{A} $ & \texttt{\textbackslash mdet[1]\{A\}} \\
		$ \ds\mdet{f(x)}[x\in A] $ & \texttt{\textbackslash mdet\{f(x)\}[x\textbackslash in A]} & $ \ds\mdet[1]{f(x)}[x\in A] $ & \texttt{\textbackslash mdet[1]\{f(x)\}[x\textbackslash in A]}\\
		$ \mdim{x} $ & \texttt{\textbackslash mdim\{x\}} & $ \mdim[1]{x} $ & \texttt{\textbackslash mdim[1]\{x\}}\\
		$ \mexp{x} $ & \texttt{\textbackslash mexp\{x\}} & $ \mexp[1]{x} $ & \texttt{\textbackslash mexp[1]\{x\}}\\
		$ \mmcd{A} $ & \texttt{\textbackslash mmcd\{A\}} & $ \mmcd[1]{A} $ & \texttt{\textbackslash mmcd[1]\{A\}} \\
		$ \ds\mmcd{f(x)}[x\in A] $ & \texttt{\textbackslash mmcd\{f(x)\}[x\textbackslash in A]} & $ \ds\mmcd[1]{f(x)}[x\in A] $ & \texttt{\textbackslash mmcd[1]\{f(x)\}[x\textbackslash in A]}\\
		$ \mln{x} $ & \texttt{\textbackslash mln\{x\}} & $ \mln[1]{x} $ & \texttt{\textbackslash mln[1]\{x\}}\\
		$ \mlog{x} $ & \texttt{\textbackslash mlog\{x\}} & $ \mlog[1]{x} $ & \texttt{\textbackslash mlog[1]\{x\}} \\
		$ \ds\mlog{x}[a] $ & \texttt{\textbackslash mlog\{x\}[a]} & $ \ds\mlog[1]{x}[a] $ & \texttt{\textbackslash mlog[1]\{x\}[a]}\\
		$ \mmax{A} $ & \texttt{\textbackslash mmax\{A\}} & $ \mmax[1]{A} $ & \texttt{\textbackslash mmax[1]\{A\}} \\
		$ \ds\mmax{f(x)}[x\in A] $ & \texttt{\textbackslash mmax\{f(x)\}[x\textbackslash in A]} & $ \ds\mmax[1]{f(x)}[x\in A] $ & \texttt{\textbackslash mmax[1]\{f(x)\}[x\textbackslash in A]}\\
		$ \mmin{A} $ & \texttt{\textbackslash mmin\{A\}} & $ \mmin[1]{A} $ & \texttt{\textbackslash mmin[1]\{A\}} \\
		$ \ds\mmin{f(x)}[x\in A] $ & \texttt{\textbackslash mmin\{f(x)\}[x\textbackslash in A]} & $ \ds\mmin[1]{f(x)}[x\in A] $ & \texttt{\textbackslash mmin[1]\{f(x)\}[x\textbackslash in A]}\\
	\end{tabular}
\end{center}
\section{Otras funciones}
En estas la mayor\'{\i}a tiene la forma:
\begin{center}
	\ttfamily
	\textbackslash funcion[$ \inprod{delimitador} $]\{$ \inprod{argumento} $\}
\end{center}
donde el \emph{delimitador} indica si el argumento de la función estará delimitado por un paréntesis o no (incluye el par \emph{left} y \emph{right}). Esto se indica con $ 0 $ o vac\'{\i}o si será delimitado y con algún otro valor (de preferencia $ 1 $) si no.
\begin{center}
	\begin{tabular}{ll|ll}
		$ \mpow{A} $ & \texttt{\textbackslash mpow\{A\}} & $ \mpow[1]{A} $ & \texttt{\textbackslash mpow[1]\{A\}}\\
		$ \mnu{A} $ & \texttt{\textbackslash mnu\{A\}} & $ \mnu[1]{A} $ & \texttt{\textbackslash mnu[1]\{A\}}\\
		$ \mrho{A} $ & \texttt{\textbackslash mrho\{A\}} & $ \mrho[1]{A} $ & \texttt{\textbackslash mrho[1]\{A\}}\\
		$ \mnuc{A} $ & \texttt{\textbackslash mnuc\{A\}} & $ \mnuc[1]{A} $ & \texttt{\textbackslash mnuc[1]\{A\}}\\
		$ \mgen{A} $ & \texttt{\textbackslash mgen\{A\}} & $ \mgen[1]{A} $ & \texttt{\textbackslash mgen[1]\{A\}}\\
		$ \mtr{A} $ & \texttt{\textbackslash mtr\{A\}} & $ \mtr[1]{A} $ & \texttt{\textbackslash mtr[1]\{A\}}\\
		$ \mdom{A} $ & \texttt{\textbackslash mdom\{A\}} & $ \mdom[1]{A} $ & \texttt{\textbackslash mdom[1]\{A\}}\\
		$ \mran{A} $ & \texttt{\textbackslash mran\{A\}} & $ \mran[1]{A} $ & \texttt{\textbackslash mran[1]\{A\}}\\
		$ \mim{A} $ & \texttt{\textbackslash mim\{A\}} & $ \mim[1]{A} $ & \texttt{\textbackslash mim[1]\{A\}}\\
		$ \ipart{z} $ & \texttt{\textbackslash ipart\{z\}} & $ \ipart[1]{z} $ & \texttt{\textbackslash ipart[1]\{z\}}\\
		$ \rpart{z} $ & \texttt{\textbackslash rpart\{z\}} & $ \rpart[1]{z} $ & \texttt{\textbackslash rpart[1]\{z\}}\\
		$ \mprob{X} $ & \texttt{\textbackslash mprob\{X\}} & $ \mprob[1]{X} $ & \texttt{\textbackslash mprob[1]\{X\}}\\
		$ \mmean{X} $ & \texttt{\textbackslash mmean\{X\}} & $ \mmean[1]{X} $ & \texttt{\textbackslash mmean[1]\{X\}}\\
		$ \mvar{X} $ & \texttt{\textbackslash mvar\{X\}} & $ \mvar[1]{X} $ & \texttt{\textbackslash mvar[1]\{X\}}\\
		$ \mmcm{A} $ & \texttt{\textbackslash mmcm\{A\}} & $ \mmcm[1]{A} $ & \texttt{\textbackslash mmcm[1]\{A\}}\\
		$ \ds\mmcm{A}[x\in A] $ & \texttt{\textbackslash mmcm\{A\}[x\textbackslash in A]} & $ \ds\mmcm[1]{A}[x\in A] $ & \texttt{\textbackslash mmcm[1]\{A\}[x\textbackslash in A]}\\
		$ \mgrad{P} $ & \texttt{\textbackslash mgrad\{P\}} & $ \mgrad[1]{P} $ & \texttt{\textbackslash mgrad[1]\{P\}}\\
		$ \mgrad{v}[G] $ & \texttt{\textbackslash mgrad\{v\}[G]} & $ \mgrad[1]{v}[G] $ & \texttt{\textbackslash mgrad[1]\{v\}[G]}\\
		$ \mleng{T} $ & \texttt{\textbackslash mleng\{T\}} & $ \mleng[1]{T} $ & \texttt{\textbackslash mleng[1]\{T\}}\\
		$ \minte{A} $ & \texttt{\textbackslash minte\{A\}} & $ \minte[1]{A} $ & \texttt{\textbackslash minte[1]\{A\}}\\
		$ \ds\minte{f(x)}[x\in A] $ & \texttt{\textbackslash minte\{f(x)\}[x\textbackslash in A]} & $ \ds\minte[1]{f(x)}[x\in A] $ & \texttt{\textbackslash minte[1]\{f(x)\}[x\textbackslash in A]}\\
		$ \mexte{A} $ & \texttt{\textbackslash mexte\{A\}} & $ \mexte[1]{A} $ & \texttt{\textbackslash mexte[1]\{A\}}\\
		$ \ds\mexte{f(x)}[x\in A] $ & \texttt{\textbackslash mexte\{f(x)\}[x\textbackslash in A]} & $ \ds\mexte[1]{f(x)}[x\in A] $ & \texttt{\textbackslash mexte[1]\{f(x)\}[x\textbackslash in A]}\\
		$ \mfr{A} $ & \texttt{\textbackslash mfr\{A\}} & $ \mfr[1]{A} $ & \texttt{\textbackslash mfr[1]\{A\}}\\
		$ \ds\mfr{f(x)}[x\in A] $ & \texttt{\textbackslash mfr\{f(x)\}[x\textbackslash in A]} & $ \ds\mfr[1]{f(x)}[x\in A] $ & \texttt{\textbackslash mfr[1]\{f(x)\}[x\textbackslash in A]}\\
		$ \mcl{A} $ & \texttt{\textbackslash mcl\{A\}} & $ \mcl[1]{A} $ & \texttt{\textbackslash mcl[1]\{A\}}\\
		$ \ds\mcl{f(x)}[x\in A] $ & \texttt{\textbackslash mcl\{f(x)\}[x\textbackslash in A]} & $ \ds\mcl[1]{f(x)}[x\in A] $ & \texttt{\textbackslash mcl[1]\{f(x)\}[x\textbackslash in A]}\\
		$ \mder{A} $ & \texttt{\textbackslash mder\{A\}} & $ \mder[1]{A} $ & \texttt{\textbackslash mder[1]\{A\}}\\
		$ \ds\mder{f(x)}[x\in A] $ & \texttt{\textbackslash mder\{f(x)\}[x\textbackslash in A]} & $ \ds\mder[1]{f(x)}[x\in A] $ & \texttt{\textbackslash mder[1]\{f(x)\}[x\textbackslash in A]}\\
		$ \mtrans{A} $ & \texttt{\textbackslash mtrans\{A\}} & $ \mtrans[1]{A} $ & \texttt{\textbackslash mtrans[1]\{A\}}\\
		$ \mcomp{A} $ & \texttt{\textbackslash mcomp\{A\}} & $ \mcomp[1]{A} $ & \texttt{\textbackslash mcomp[1]\{A\}}\\
		$ \minv{A} $ & \texttt{\textbackslash minv\{A\}} & $ \minv[1]{A} $ & \texttt{\textbackslash minv[1]\{A\}}\\
		$ \mperp{A} $ & \texttt{\textbackslash mperp\{A\}} & $ \mperp[1]{A} $ & \texttt{\textbackslash mperp[1]\{A\}}\\
		$ \madj{A} $ & \texttt{\textbackslash madj\{A\}} & $ \madj[1]{A} $ & \texttt{\textbackslash madj[1]\{A\}}\\
		$ \meval{A}[\beta] $ & \texttt{\textbackslash meval\{A\}[\textbackslash beta]} & $ \meval{A}[\beta][\gamma] $ & \texttt{\textbackslash meval\{A\}[\textbackslash beta][\textbackslash gamma]}\\
		$ \ds\mint{f(u)}{u} $ & \texttt{\textbackslash mint\{f(u)\}\{u\}} & $ \ds\mint[1]{f(u)}{u} $ & \texttt{\textbackslash mint[1]\{f(u)\}\{u\}}\\
		$ \ds\mint{f(u)}{u}[a][b] $ & \texttt{\textbackslash mint\{f(u)\}\{u\}[a][b]} & $ \ds\mint[1]{f(u)}{u}[a][b] $ & \texttt{\textbackslash mint[1]\{f(u)\}\{u\}[a][b]}\\
		$ \mproj{u}{v} $ & \texttt{\textbackslash mproj\{u\}\{v\}} & $ \mmatrix{F}{nm} $ & \texttt{\textbackslash mmatrix\{F\}\{nm\}}\\
		$ \mlin{V} $ & \texttt{\textbackslash mlin\{V\}} & $ \mpol{F}{n} $ & \texttt{\textbackslash mpol\{F\}\{n\}}\\
		$ \mfunc{W} $ & \texttt{\textbackslash mfunc\{W\}} & $ \msinf{F} $ & \texttt{\textbackslash msinf\{F\}}\\
		$ \mssup{F} $ & \texttt{\textbackslash mssup\{F\}} & $ \mdis{x, y} $ & \texttt{\textbackslash mdis\{x, y\}}\\
		$\mmeas{A}$ & \textbackslash mmeas\{A\} & $\mclass[n]$ & \textbackslash mclass[n]
	\end{tabular}
\end{center}
\section{Otros s\'{\i}mbolos}
\begin{center}
	\begin{tabular}{ll|ll|ll}
		$ \void $ & \texttt{\textbackslash void} & $ \true $ & \texttt{\textbackslash true} & $ \false $ & \texttt{\textbackslash false}\\
		$ \ivec $ & \texttt{\textbackslash ivec} & $ \jvec $ & \texttt{\textbackslash jvec} & $ \kvec $ & \texttt{\textbackslash kvec}
	\end{tabular}
\end{center}
\end{document}

%----------------------------------------------------------------------------------------
%	FONTS
%----------------------------------------------------------------------------------------

\usepackage{avant} % Use the Avantgarde font for headings
%\usepackage{times} % Use the Times font for headings
\usepackage{mathptmx} % Use the Adobe Times Roman as the default text font together with math symbols from the Sym­bol, Chancery and Com­puter Modern fonts

\usepackage{microtype} % Slightly tweak font spacing for aesthetics
\usepackage[utf8]{inputenc} % Required for including letters with accents
\usepackage[T1]{fontenc} % Use 8-bit encoding that has 256 glyphs

%----------------------------------------------------------------------------------------
%	BIBLIOGRAPHY AND INDEX
%----------------------------------------------------------------------------------------

\usepackage{biblatex}
\addbibresource{bibliography.bib} % BibTeX bibliography file
\defbibheading{bibempty}{}

\usepackage{calc} % For simpler calculation - used for spacing the index letter headings correctly
\usepackage{makeidx} % Required to make an index
\makeindex % Tells LaTeX to create the files required for indexing

%----------------------------------------------------------------------------------------
%	MAIN TABLE OF CONTENTS
%----------------------------------------------------------------------------------------

\usepackage{titletoc} % Required for manipulating the table of contents

\contentsmargin{0cm} % Removes the default margin

% Part text styling
\titlecontents{part}[0cm]
{\addvspace{20pt}\centering\large\bfseries}
{}
{}
{}

% Chapter text styling
\titlecontents{chapter}[1.25cm] % Indentation
{\addvspace{12pt}\large\sffamily\bfseries} % Spacing and font options for chapters
{\color{pgay1!60}\contentslabel[\Large\thecontentslabel]{1.25cm}\color{pgay1}} % Chapter number
{\color{pgay1}}  
{\color{pgay1!60}\normalsize\;\titlerule*[.5pc]{.}\;\thecontentspage} % Page number

% Section text styling
\titlecontents{section}[1.25cm] % Indentation
{\addvspace{3pt}\sffamily\bfseries} % Spacing and font options for sections
{\contentslabel[\thecontentslabel]{1.25cm}} % Section number
{}
{\hfill\color{black}\thecontentspage} % Page number
[]

% Subsection text styling
\titlecontents{subsection}[1.25cm] % Indentation
{\addvspace{1pt}\sffamily\small} % Spacing and font options for subsections
{\contentslabel[\thecontentslabel]{1.25cm}} % Subsection number
{}
{\ \titlerule*[.5pc]{.}\;\thecontentspage} % Page number
[]

% List of figures
\titlecontents{figure}[0em]
{\addvspace{-5pt}\sffamily}
{\thecontentslabel\hspace*{1em}}
{}
{\ \titlerule*[.5pc]{.}\;\thecontentspage}
[]

% List of tables
\titlecontents{table}[0em]
{\addvspace{-5pt}\sffamily}
{\thecontentslabel\hspace*{1em}}
{}
{\ \titlerule*[.5pc]{.}\;\thecontentspage}
[]

%----------------------------------------------------------------------------------------
%	MINI TABLE OF CONTENTS IN PART HEADS
%----------------------------------------------------------------------------------------

% Chapter text styling
\titlecontents{lchapter}[0em] % Indenting
{\addvspace{15pt}\large\sffamily\bfseries} % Spacing and font options for chapters
{\color{pgay1}\contentslabel[\Large\thecontentslabel]{1.25cm}\color{pgay1}} % Chapter number
{}  
{\color{pgay1}\normalsize\sffamily\bfseries\;\titlerule*[.5pc]{.}\;\thecontentspage} % Page number

% Section text styling
\titlecontents{lsection}[0em] % Indenting
{\sffamily\small} % Spacing and font options for sections
{\contentslabel[\thecontentslabel]{1.25cm}} % Section number
{}
{}

% Subsection text styling
\titlecontents{lsubsection}[.5em] % Indentation
{\normalfont\footnotesize\sffamily} % Font settings
{}
{}
{}

%----------------------------------------------------------------------------------------
%	PAGE HEADERS
%----------------------------------------------------------------------------------------

\usepackage{fancyhdr} % Required for header and footer configuration

\pagestyle{fancy}
\renewcommand{\chaptermark}[1]{\markboth{\sffamily\normalsize\bfseries\chaptername\ \thechapter.\ #1}{}} % Chapter text font settings
\renewcommand{\sectionmark}[1]{\markright{\sffamily\normalsize\thesection\hspace{5pt}#1}{}} % Section text font settings
\fancyhf{} \fancyhead[LE,RO]{\sffamily\normalsize\thepage} % Font setting for the page number in the header
\fancyhead[LO]{\rightmark} % Print the nearest section name on the left side of odd pages
\fancyhead[RE]{\leftmark} % Print the current chapter name on the right side of even pages
\renewcommand{\headrulewidth}{0.5pt} % Width of the rule under the header
\addtolength{\headheight}{2.5pt} % Increase the spacing around the header slightly
\renewcommand{\footrulewidth}{0pt} % Removes the rule in the footer
\fancypagestyle{plain}{\fancyhead{}\renewcommand{\headrulewidth}{0pt}} % Style for when a plain pagestyle is specified

% Removes the header from odd empty pages at the end of chapters
\makeatletter
\renewcommand{\cleardoublepage}{
\clearpage\ifodd\c@page\else
\hbox{}
\vspace*{\fill}
\thispagestyle{empty}
\newpage
\fi}

%----------------------------------------------------------------------------------------
%	THEOREM STYLES
%----------------------------------------------------------------------------------------

% Boxed/framed environments
\newtheoremstyle{pgaynumbox}% % Theorem style name
{0pt}% Space above
{0pt}% Space below
{\normalfont}% % Body font
{}% Indent amount
{\small\bf\sffamily\color{pgay1}}% % Theorem head font
{\;}% Punctuation after theorem head
{0.25em}% Space after theorem head
{\small\sffamily\color{pgay1}\thmname{#1}\nobreakspace\thmnumber{\@ifnotempty{#1}{}\@upn{#2}}% Theorem text (e.g. Theorem 2.1)
\thmnote{\nobreakspace\the\thm@notefont\sffamily\bfseries\color{black}---\nobreakspace#3.}} % Optional theorem note
\renewcommand{\qedsymbol}{$\blacksquare$}% Optional qed square

\newtheoremstyle{blacknumex}% Theorem style name
{5pt}% Space above
{5pt}% Space below
{\normalfont}% Body font
{} % Indent amount
{\small\bf\sffamily}% Theorem head font
{\;}% Punctuation after theorem head
{0.25em}% Space after theorem head
{\small\sffamily{\tiny\ensuremath{\blacksquare}}\nobreakspace\thmname{#1}\nobreakspace\thmnumber{\@ifnotempty{#1}{}\@upn{#2}}% Theorem text (e.g. Theorem 2.1)
\thmnote{\nobreakspace\the\thm@notefont\sffamily\bfseries---\nobreakspace#3.}}% Optional theorem note

\newtheoremstyle{blacknumbox} % Theorem style name
{0pt}% Space above
{0pt}% Space below
{\normalfont}% Body font
{}% Indent amount
{\small\bf\sffamily}% Theorem head font
{\;}% Punctuation after theorem head
{0.25em}% Space after theorem head
{\small\sffamily\thmname{#1}\nobreakspace\thmnumber{\@ifnotempty{#1}{}\@upn{#2}}% Theorem text (e.g. Theorem 2.1)
\thmnote{\nobreakspace\the\thm@notefont\sffamily\bfseries---\nobreakspace#3.}}% Optional theorem note

% Non-boxed/non-framed environments
\newtheoremstyle{pgaynum}% % Theorem style name
{5pt}% Space above
{5pt}% Space below
{\normalfont}% % Body font
{}% Indent amount
{\small\bf\sffamily\color{pgay1}}% % Theorem head font
{\;}% Punctuation after theorem head
{0.25em}% Space after theorem head
{\small\sffamily\color{pgay1}\thmname{#1}\nobreakspace\thmnumber{\@ifnotempty{#1}{}\@upn{#2}}% Theorem text (e.g. Theorem 2.1)
\thmnote{\nobreakspace\the\thm@notefont\sffamily\bfseries\color{black}---\nobreakspace#3.}} % Optional theorem note
\renewcommand{\qedsymbol}{$\blacksquare$}% Optional qed square
\makeatother

% Defines the theorem text style for each type of theorem to one of the three styles above
\newcounter{dummy} 
\numberwithin{dummy}{section}

\theoremstyle{blacknumbox}
\newtheorem{theoremeT}[dummy]{Teorema}
\newtheorem{problem}{Problema}[chapter]
\newtheorem{exerciseT}{Ejercicio}[chapter]
\newtheorem{exampleT}{Ejemplo}[chapter]

\newtheorem{vocabulary}{Vocabulary}[chapter]
\newtheorem{axiomT}{Axioma}[chapter]
\newtheorem{definitionT}{Definición}[section]
\newtheorem{corollaryT}[dummy]{Corolario}
\newtheorem{lemmaT}[dummy]{Lema}

%----------------------------------------------------------------------------------------
%	DEFINITION OF COLORED BOXES
%----------------------------------------------------------------------------------------

% Theorem box
\newmdenv[skipabove=7pt,
skipbelow=7pt,
rightline=false,
leftline=true,
topline=false,
bottomline=false,
linecolor=pgay1,
backgroundcolor=black!5,
innerleftmargin=5pt,
innerrightmargin=5pt,
innertopmargin=5pt,
leftmargin=0cm,
rightmargin=0cm,
linewidth=4pt,
innerbottommargin=5pt]{tBox}

% Exercise box	  
\newmdenv[skipabove=7pt,
skipbelow=7pt,
rightline=false,
leftline=true,
topline=false,
bottomline=false,
linecolor=pgay6,
backgroundcolor=black!5,
innerleftmargin=5pt,
innerrightmargin=5pt,
innertopmargin=5pt,
leftmargin=0cm,
rightmargin=0cm,
linewidth=4pt,
innerbottommargin=5pt]{eBox}	

% Definition box
\newmdenv[skipabove=7pt,
skipbelow=7pt,
rightline=false,
leftline=true,
topline=false,
bottomline=false,
linecolor=pgay5,
backgroundcolor=black!5,
innerleftmargin=5pt,
innerrightmargin=5pt,
innertopmargin=5pt,
leftmargin=0cm,
rightmargin=0cm,
linewidth=4pt,
innerbottommargin=5pt]{dBox}	

% Corollary box
\newmdenv[skipabove=7pt,
skipbelow=7pt,
rightline=false,
leftline=true,
topline=false,
bottomline=false,
linecolor=pgay2,
backgroundcolor=black!5,
innerleftmargin=5pt,
innerrightmargin=5pt,
innertopmargin=5pt,
leftmargin=0cm,
rightmargin=0cm,
linewidth=4pt,
innerbottommargin=5ptt]{cBox}

% Axiom box
\newmdenv[skipabove=7pt,
skipbelow=7pt,
rightline=false,
leftline=true,
topline=false,
bottomline=false,
linecolor=pgay6,
backgroundcolor=black!5,
innerleftmargin=5pt,
innerrightmargin=5pt,
innertopmargin=5pt,
leftmargin=0cm,
rightmargin=0cm,
linewidth=4pt,
innerbottommargin=5pt]{aBox}

% Lemma box
\newmdenv[skipabove=7pt,
skipbelow=7pt,
rightline=false,
leftline=true,
topline=false,
bottomline=false,
linecolor=pgay3,
backgroundcolor=black!5,
innerleftmargin=5pt,
innerrightmargin=5pt,
innertopmargin=5pt,
leftmargin=0cm,
rightmargin=0cm,
linewidth=4pt,
innerbottommargin=5pt]{lBox}

% Creates an environment for each type of theorem and assigns it a theorem text style from the "Theorem Styles" section above and a colored box from above
\newenvironment{theorem}{\begin{tBox}\begin{theoremeT}}{\end{theoremeT}\end{tBox}}
\newenvironment{exercise}{\begin{eBox}\begin{exerciseT}}{\end{exerciseT}\end{eBox}}				  
\newenvironment{definition}{\begin{dBox}\begin{definitionT}}{\end{definitionT}\end{dBox}}	
\newenvironment{example}{\begin{exampleT}}{\end{exampleT}}		
\newenvironment{coll}{\begin{cBox}\begin{corollaryT}}{\end{corollaryT}\end{cBox}}	
\newenvironment{axiom}{\begin{aBox}\begin{axiomT}}{\end{axiomT}\end{aBox}}
\newenvironment{lemma}{\begin{lBox}\begin{lemmaT}}{\end{lemmaT}\end{lBox}}
%----------------------------------------------------------------------------------------
%	REMARK ENVIRONMENT
%----------------------------------------------------------------------------------------

\newtcolorbox{remark}[1][]{
	title={\scalebox{1.75}{\raisebox{-.25ex}{\ding{43}}}~\textbf{Observación}},
	colframe=pgay6!10,
	colback=pgay6!10,
	coltitle=black,
	fontupper=\sffamily,
	boxed title style={colback=pgay6!10},
	boxed title style={boxsep=1ex,sharp corners},%%
	overlay unbroken and first={ \node[below right,font=\normalsize,color=red,text width=.8\linewidth] at (title.north east) {#1};
	}
}
\newtcolorbox{note}[1][]{
	title={\scalebox{1.75}{\raisebox{-.25ex}{\ding{45}}}~Nota},
	colframe=purple!10,
	colback=purple!10,
	coltitle=black,
	fontupper=\sffamily,
	boxed title style={colback=purple!10},
	boxed title style={boxsep=1ex,sharp corners},%%
	overlay unbroken and first={ \node[below right,font=\normalsize,color=red,text width=.8\linewidth] at (title.north east) {#1};
	}
}
\newtcolorbox{notation}[1][]{
	title={\scalebox{1.75}{\raisebox{-.25ex}{\ding{46}}}~\textbf{Notación}},
	colframe=pgay1!10,
	colback=pgay1!10,
	coltitle=black,
	fontupper=\sffamily,
	boxed title style={colback=pgay1!10},
	boxed title style={boxsep=1ex,sharp corners},%%
	overlay unbroken and first={ \node[below right,font=\normalsize,color=red,text width=.8\linewidth] at (title.north east) {#1};
	}
}
\newenvironment{loexercisesT}{\textbf{Lista de ejercicios}}{}
\newenvironment{loexercises}{\begin{eBox}\begin{loexercisesT}}{\end{loexercisesT}\end{eBox}}
%----------------------------------------------------------------------------------------
%	SECTION NUMBERING IN THE MARGIN
%----------------------------------------------------------------------------------------

\makeatletter
\renewcommand{\@seccntformat}[1]{\llap{\textcolor{pgay1}{\csname the#1\endcsname}\hspace{1em}}}                    
\renewcommand{\section}{\@startsection{section}{1}{\z@}
{-4ex \@plus -1ex \@minus -.4ex}
{1ex \@plus.2ex }
{\normalfont\large\sffamily\bfseries}}
\renewcommand{\subsection}{\@startsection {subsection}{2}{\z@}
{-3ex \@plus -0.1ex \@minus -.4ex}
{0.5ex \@plus.2ex }
{\normalfont\sffamily\bfseries}}
\renewcommand{\subsubsection}{\@startsection {subsubsection}{3}{\z@}
{-2ex \@plus -0.1ex \@minus -.2ex}
{.2ex \@plus.2ex }
{\normalfont\small\sffamily\bfseries}}                        
\renewcommand\paragraph{\@startsection{paragraph}{4}{\z@}
{-2ex \@plus-.2ex \@minus .2ex}
{.1ex}
{\normalfont\small\sffamily\bfseries}}

%----------------------------------------------------------------------------------------
%	PART HEADINGS
%----------------------------------------------------------------------------------------

% numbered part in the table of contents
\newcommand{\@mypartnumtocformat}[2]{%
\setlength\fboxsep{0pt}%
\noindent\colorbox{pgay1!20}{\strut\parbox[c][.7cm]{\ecart}{\color{pgay1!70}\Large\sffamily\bfseries\centering#1}}\hskip\esp\colorbox{pgay1!40}{\strut\parbox[c][.7cm]{\linewidth-\ecart-\esp}{\Large\sffamily\centering#2}}}%
%%%%%%%%%%%%%%%%%%%%%%%%%%%%%%%%%%
% unnumbered part in the table of contents
\newcommand{\@myparttocformat}[1]{%
\setlength\fboxsep{0pt}%
\noindent\colorbox{pgay1!40}{\strut\parbox[c][.7cm]{\linewidth}{\Large\sffamily\centering#1}}}%
%%%%%%%%%%%%%%%%%%%%%%%%%%%%%%%%%%
\newlength\esp
\setlength\esp{4pt}
\newlength\ecart
\setlength\ecart{1.2cm-\esp}
\newcommand{\thepartimage}{}%
\newcommand{\partimage}[1]{\renewcommand{\thepartimage}{#1}}%
\def\@part[#1]#2{%
\ifnum \c@secnumdepth >-2\relax%
\refstepcounter{part}%
\addcontentsline{toc}{part}{\texorpdfstring{\protect\@mypartnumtocformat{\thepart}{#1}}{\partname~\thepart\ ---\ #1}}
\else%
\addcontentsline{toc}{part}{\texorpdfstring{\protect\@myparttocformat{#1}}{#1}}%
\fi%
\startcontents%
\markboth{}{}%
{\thispagestyle{empty}%
\begin{tikzpicture}[remember picture,overlay]%
\node at (current page.north west){\begin{tikzpicture}[remember picture,overlay]%	
\fill[pgay1!20](0cm,0cm) rectangle (\paperwidth,-\paperheight);
\node[anchor=north] at (4cm,-3.25cm){\color{pgay1!40}\fontsize{220}{100}\sffamily\bfseries\@Roman\c@part}; 
\node[anchor=south east] at (\paperwidth-1cm,-\paperheight+1cm){\parbox[t][][t]{8.5cm}{
\printcontents{l}{0}{\setcounter{tocdepth}{1}}%
}};
\node[anchor=north east] at (\paperwidth-1.5cm,-3.25cm){\parbox[t][][t]{15cm}{\strut\raggedleft\color{white}\fontsize{30}{30}\sffamily\bfseries#2}};
\end{tikzpicture}};
\end{tikzpicture}}%
\@endpart}
\def\@spart#1{%
\startcontents%
\phantomsection
{\thispagestyle{empty}%
\begin{tikzpicture}[remember picture,overlay]%
\node at (current page.north west){\begin{tikzpicture}[remember picture,overlay]%	
\fill[pgay1!20](0cm,0cm) rectangle (\paperwidth,-\paperheight);
\node[anchor=north east] at (\paperwidth-1.5cm,-3.25cm){\parbox[t][][t]{15cm}{\strut\raggedleft\color{white}\fontsize{30}{30}\sffamily\bfseries#1}};
\end{tikzpicture}};
\end{tikzpicture}}
\addcontentsline{toc}{part}{\texorpdfstring{%
\setlength\fboxsep{0pt}%
\noindent\protect\colorbox{pgay1!40}{\strut\protect\parbox[c][.7cm]{\linewidth}{\Large\sffamily\protect\centering #1\quad\mbox{}}}}{#1}}%
\@endpart}
\def\@endpart{\vfil\newpage
\if@twoside
\if@openright
\null
\thispagestyle{empty}%
\newpage
\fi
\fi
\if@tempswa
\twocolumn
\fi}

%----------------------------------------------------------------------------------------
%	CHAPTER HEADINGS
%----------------------------------------------------------------------------------------

\newcommand{\thechapterimage}{}%
\newcommand{\chapterimage}[1]{\renewcommand{\thechapterimage}{#1}}%
\def\@makechapterhead#1{%
{\parindent \z@ \raggedright \normalfont
\ifnum \c@secnumdepth >\m@ne
\if@mainmatter
\begin{tikzpicture}[remember picture,overlay]
\node at (current page.north west)
{\begin{tikzpicture}[remember picture,overlay]
\node[anchor=north west,inner sep=0pt] at (0,0) {\includegraphics[width=\paperwidth]{\thechapterimage}};
\draw[anchor=west] (\Gm@lmargin,-9cm) node [line width=2pt,rounded corners=15pt,draw=pgay1,fill=white,fill opacity=0.5,inner sep=15pt]{\strut\makebox[22cm]{}};
\draw[anchor=west] (\Gm@lmargin+.3cm,-9cm) node {\huge\sffamily\bfseries\color{black}\thechapter. #1\strut};
\end{tikzpicture}};
\end{tikzpicture}
\else
\begin{tikzpicture}[remember picture,overlay]
\node at (current page.north west)
{\begin{tikzpicture}[remember picture,overlay]
\node[anchor=north west,inner sep=0pt] at (0,0) {\includegraphics[width=\paperwidth]{\thechapterimage}};
\draw[anchor=west] (\Gm@lmargin,-9cm) node [line width=2pt,rounded corners=15pt,draw=pgay1,fill=white,fill opacity=0.5,inner sep=15pt]{\strut\makebox[22cm]{}};
\draw[anchor=west] (\Gm@lmargin+.3cm,-9cm) node {\huge\sffamily\bfseries\color{black}#1\strut};
\end{tikzpicture}};
\end{tikzpicture}
\fi\fi\par\vspace*{270\p@}}}

%-------------------------------------------

\def\@makeschapterhead#1{%
\begin{tikzpicture}[remember picture,overlay]
\node at (current page.north west)
{\begin{tikzpicture}[remember picture,overlay]
\node[anchor=north west,inner sep=0pt] at (0,0) {\includegraphics[width=\paperwidth]{\thechapterimage}};
\draw[anchor=west] (\Gm@lmargin,-9cm) node [line width=2pt,rounded corners=15pt,draw=pgay1,fill=white,fill opacity=0.5,inner sep=15pt]{\strut\makebox[22cm]{}};
\draw[anchor=west] (\Gm@lmargin+.3cm,-9cm) node {\huge\sffamily\bfseries\color{black}#1\strut};
\end{tikzpicture}};
\end{tikzpicture}
\par\vspace*{270\p@}}
\makeatother

%----------------------------------------------------------------------------------------
%	HYPERLINKS IN THE DOCUMENTS
%----------------------------------------------------------------------------------------

\usepackage{hyperref}
\hypersetup{hidelinks,backref=true,pagebackref=true,hyperindex=true,colorlinks=false,breaklinks=true,urlcolor= pgay1,bookmarks=true,bookmarksopen=false,pdftitle={Title},pdfauthor={Author}}
\usepackage{bookmark}
\bookmarksetup{
open,
numbered,
addtohook={%
\ifnum\bookmarkget{level}=0 % chapter
\bookmarksetup{bold}%
\fi
\ifnum\bookmarkget{level}=-1 % part
\bookmarksetup{color=pgay1,bold}%
\fi
}
} % Insert the commands.tex file which contains the majority of the structure behind the template

\numberwithin{equation}{section} % Number equations within sections (i.e. 1.1, 1.2, 2.1, 2.2 instead of 1, 2, 3, 4)
\numberwithin{figure}{section} % Number figures within sections (i.e. 1.1, 1.2, 2.1, 2.2 instead of 1, 2, 3, 4)
\numberwithin{table}{section} % Number tables within sections (i.e. 1.1, 1.2, 2.1, 2.2 instead of 1, 2, 3, 4)


\setlength\parindent{0pt} % Removes all indentation from paragraphs - comment this line for an assignment with lots of text

%%hasta aquí


\renewcommand{\familydefault}{\sfdefault}

\begin{document}

%----------------------------------------------------------------------------------------
%	TITLE PAGE
%----------------------------------------------------------------------------------------

\begingroup
\thispagestyle{empty}
\begin{tikzpicture}[remember picture,overlay]
\coordinate [below=12cm] (midpoint) at (current page.north);
\node at (current page.north west)
{\begin{tikzpicture}[remember picture,overlay]
%\node[anchor=north west,inner sep=0pt] at (0,0) {\includegraphics[width=\paperwidth]{gay.png}}; % Background image
\draw[anchor=north] (midpoint) node [fill=pgay1!30!white,fill opacity=0.6,text opacity=1,inner sep=1cm]{\Huge\centering\bfseries\sffamily\parbox[c][][t]{\paperwidth}{\centering Introducción al Cálculo\\[15pt] % Book title
{\Large Instituto Politécnico Nacional\\Escuela Superior de Física y Matemáticas\\Licenciatura en Matemática Algorítmica}\\[20pt] % Subtitle
{\huge Omar Porfirio García}}}; % Author name
\end{tikzpicture}};
\end{tikzpicture}
\vfill
\endgroup


%----------------------------------------------------------------------------------------
%	COPYRIGHT PAGE
%----------------------------------------------------------------------------------------

%\newpage
%~\vfill
%\thispagestyle{empty}

%\noindent Copyright \copyright\ 2013 John Smith\\ % Copyright notice

%\noindent \textsc{Published by Publisher}\\ % Publisher

%\noindent \textsc{book-website.com}\\ % URL

%\noindent Licensed under the Creative Commons Attribution-NonCommercial 3.0 Unported License (the ``License''). You may not use this file except in compliance with the License. You may obtain a copy of the License at \url{http://creativecommons.org/licenses/by-nc/3.0}. Unless required by applicable law or agreed to in writing, software distributed under the License is distributed on an \textsc{``as is'' basis, without warranties or conditions of any kind}, either express or implied. See the License for the specific language governing permissions and limitations under the License.\\ % License information

%\noindent \textit{First printing, March 2013} % Printing/edition date

%----------------------------------------------------------------------------------------
%	TABLE OF CONTENTS
%----------------------------------------------------------------------------------------

\chapterimage{gay.png} % Table of contents heading image

%\chapterimage{chapter_head_1.pdf} % Table of contents heading image

\pagestyle{empty} % No headers

 \tableofcontents % Print the table of contents itself

\cleardoublepage % Forces the first chapter to start on an odd page so it's on the right

\pagestyle{fancy} % Print headers again

%================================
%	Contenido
%================================

\part{Preliminares}
\chapterimage{Imagenes/Encabezados/Logica.pdf}
\chapter{Lógica y conjuntos}
\section{Lógica proposicional}
\begin{definition}[Proposición]
	Una \emph{proposición} es un conjunto de símbolos con un significado (tales como una oración, afirmación o expresión matemática) el cual puede determinarse como \emph{verdadero} o \emph{falso}, pero no ambas.
\end{definition}
Ejemplos de \emph{proposiciones} pueden ser las siguientes:
\begin{enumerate}
	\item $2 + 2 = 7$
	\item El día de hoy está soleado.
	\item Esto es un texto.
	\item Existe un infinito número de primos.
\end{enumerate}
pues cada una puede determinarse como verdadera o falsa, pero no ambas.
Hay algunas otras oraciones o expresiones que no se les puede determinar si son verdaderas o falsas, y por tanto no pueden considerarse proposiciones:
\begin{enumerate}
	\item ¿Qué día es hoy?
	\item ¡Vamos!
	\item Tú y yo.
	\item $+3* = /1-$
\end{enumerate}
Las oraciones interrogativas e imperativas (como las primeras dos oraciones) no se les puede asignar un \emph{valor de verdad} (verdadero o falso), por lo que no se les puede denominar proposiciones.\par 
En el caso de ``tú y yo'', aunque es una oración declarativa, dado que no es posible y no tiene sentido asignarle un valor de verdad, entonces no puede ser denominada proposición tampoco.\par
En el último caso de la expresión $+3* = /1-$, dado que no tiene sentido y por tanto no tiene un significado, no se le puede asignar un valor de verdad. Por lo tanto, no es una proposición.\par 
Un ejemplo muy peculiar y clásico de oración que parecería ser una proposición, pero que en realidad no lo es, es:
\begin{center}
	``Esta afirmación es falsa''
\end{center}
Supongamos que fuera verdadera, entonces por su afirmación sería falsa, por lo que no sería verdadera. Supongamos lo contrario, que es falsa, entonces afirmaría que no es falsa, es decir, verdadera, por lo que no sería falsa. Dado que nos lleva a contradicciones (algo que es pero no es al mismo tiempo), no es posible asignarle un valor de verdad y entonces no sería una proposición.\par 
Debemos tomar en cuenta que la definición dada de proposición no es muy precisa y vaga. Sin embargo, nos es útil para introducirnos al razonamiento matemático y en realidad lo esencial es poder determinar cuando una afirmación o expresión es verdadera o falsa, pues nos ayudará para poder hacer deducciones más adelante.
\begin{notation}
	Para denotar a una proposición, haremos uso de literales como $p, q, r, s,\ldots$.\par 
	Para denotar específicamente a qué proposición hace referencia cada literal haremos uso de la notación:
	\begin{center}
		$p:$ ``proposición''
	\end{center}
	Por ejemplo:
	\begin{center}
		$s:$ ``América es un continente''
	\end{center}
	significa que la literal $s$ hará referencia a la proposición ``América es un continente''.\par 
	Si no se especifica a qué proposición en particular una literal hace referencia, esta podrá hacer referencia a cualquier proposición y se denominará \emph{variable lógica}. Por lo general, literales distintas harán referencia a proposiciones distintas, a menos que se especifique lo contrario.
\end{notation}
\subsection{Conectores lógicos}
Para el estudio de las proposiciones es útil clasificar a estas en dos grupos: \emph{simples} o \emph{atómicas} y \emph{compuestas}.\par 
Las \emph{proposiciones simples} o \emph{atómicas} son aquellas que no pueden dividirse en proposiciones ``más pequeñas''; su valor de verdad no depende de otras proposiciones. Por ejemplo, considere la siguiente proposición:
\begin{center}
	$p:$ ``Hoy está soleado pero hace frío''
\end{center}
La proposición $p$ puede dividirse en dos proposiciones más pequeñas, las cuales son:
\begin{center}
	$p_{1}:$ ``Hoy está soleado''\\
	$p_{2}:$ ``Hace frío''
\end{center}
Dado que $p$ puede dividirse en $p_{1}$ y $p_{2}$, las cuales son proposiciones ``más pequeñas'', $p$ no se considera atómica. Sin embargo, las proposiciones $p_{1}$ y $p_{2}$ no pueden dividirse en proposiciones más pequeñas, por lo que estas si se consideran atómicas o simples.\par 
Por otro lado, las \emph{proposiciones compuestas} son aquellas que están conformadas de varias proposiciones atómicas, como el caso de $p$ que se conformaba de $p_{1}$ y $p_{2}$. Otro ejemplo es:
\begin{center}
	$q:$ ``Él sabe cocinar y yo sé patinar''
\end{center}
la cual puede dividirse en dos \emph{subproposiciones}:
\begin{center}
	$q_{1}:$ ``Él sabe cocinar''\\
	$q_{2}:$ ``Yo sé patinar''
\end{center}
Otros ejemplos de proposiciones compuestas son las siguientes:
\begin{enumerate}
	\item Un número real es o racional o irracional.
	\item La tierra no es plana pero si redonda.
	\item Si estudiar para el examen, lo apruebas.
	\item No eres ni malo ni bueno.
\end{enumerate}
Hay que notar que para unir proposiciones atómicas y así formar proposiciones compuestas hacemos uso de ciertas palabras como \emph{y}, \emph{o}; \emph{si$\ldots$, entonces$\ldots$}; \emph{o$\ldots$ o$\ldots$}, etc. A este tipo de palabras que nos ayudan a formar proposiciones compuestas se les denomina \emph{conectores lógicos}. Aunque se pueden enumerar varios conectores lógicos, los tres más básicos que podemos usar para formar proposiciones compuestas son la \emph{negación}, \emph{disyunción} y \emph{conjunción}.
\input{Secciones/Preliminares/LogicaConjuntos/ArgumentosLogicos.tex}
\input{Secciones/Preliminares/LogicaConjuntos/Conjuntos.tex}
\chapterimage{Imagenes/Encabezados/Predicados}
\chapter{Demostraciones}
\input{Secciones/Preliminares/Demostraciones/Predicados.tex}
\input{Secciones/Preliminares/Demostraciones/MetodosDemostracion.tex}

\part{El conjunto de los reales}
\chapterimage{Imagenes/Encabezados/Reales}
\chapter{Los números reales}
\input{Secciones/Reales/LosReales/AxiomasCampo.tex}
\input{Secciones/Reales/LosReales/AxiomasOrden.tex}
\input{Secciones/Reales/LosReales/AxiomaSupremo.tex}
\section{Algunos teoremas sobre los números reales}
\chapterimage{Imagenes/Encabezados/Funciones}
\chapter{Funciones reales en una variable}
\section{Funciones reales en una variable}
\section{Inyectividad, suprayectividad y biyectividad}
\input{Secciones/Reales/FuncionesReales/Inversas.tex}

\part{Sucesiones y series}
\chapterimage{Imagenes/Encabezados/Sucesiones}
\chapter{Sucesiones}
\input{Secciones/SucesionesSeries/Sucesiones/Convergencia.tex}
\section{Algunos teoremas de convergencia}
\input{Secciones/SucesionesSeries/Sucesiones/Subsucesiones.tex}
\input{Secciones/SucesionesSeries/Sucesiones/Cauchy.tex}
\input{Secciones/SucesionesSeries/Sucesiones/Divergencia.tex}

\chapterimage{Imagenes/Encabezados/Series}
\chapter{Series}
\input{Secciones/SucesionesSeries/Series/Convergencia.tex}
\input{Secciones/SucesionesSeries/Series/Criterios.tex}
\section{Convergencia absoluta y condicional}


\part{Límites y continuidad}
\chapterimage{Imagenes/Encabezados/Limites}
\chapter{Límites}
\section{Límites de funciones}
\section{Propiedades de los límites}
\section{Cálculo de derivadas}
\section{Límites al infinito}

\chapterimage{Imagenes/Encabezados/Continuidad}
\chapter{Continuidad}
\input{Secciones/LimitesContinuidad/Continuidad/Continuidad.tex}
\section{Teoremas fuertes de continuidad}
\input{Secciones/LimitesContinuidad/Continuidad/Uniforme.tex}


\part{Cálculo diferencial}
\chapterimage{Imagenes/Encabezados/Diferencial}
\chapter{Diferenciabilidad}
\input{Secciones/Diferencial/Diferencial/Diferenciabilidad.tex}
\section{Cálculo de derivadas}
\section{Derivadas de orden superior}
\section{Algunos teoremas de diferenciabilidad}


\chapter{Aplicaciones de la derivada}
\section{Máximos, mínimos y concavidad de una función}
\input{Secciones/Diferencial/Aplicaciones/Bosquejo.tex}
\input{Secciones/Diferencial/Aplicaciones/Razon.tex}
\input{Secciones/Diferencial/Aplicaciones/Diferenciales.tex}
\section{Regla de L'H\^{o}pital}


\part{Cálculo integral}
\chapterimage{Imagenes/Encabezados/Integral}
\chapter{Integrabilidad}
\section{Particiones de un intervalo}
\input{Secciones/Integral/Integrabilidad/Integrabilidad.tex}
\section{Propiedades de los límites}
\section{Teorema fundamental del Cálculo}
\section{Algunos teoremas de la integrabilidad}
\input{Secciones/Integral/Integrabilidad/Indefinidas.tex}

\chapter{Funciones trascendentes}
\section{La función logaritmo}
\section{La función exponencial}
\section{Funciones de potencias arbitrarias}
\section{Las funciones trigonométricas inversas}
\section{Las funciones hiperbólicas}

\chapter{Métodos de integración}
\input{Secciones/Integral/Metodos/Sustitucion.tex}
\section{Integración por partes}
\input{Secciones/Integral/Metodos/Trigonometrica.tex}
\input{Secciones/Integral/Metodos/Fracciones.tex}

\chapter{Aplicaciones de la integral}
\section{Cálculo de áreas y volúmenes}
\section{Volúmen de sólidos de revolución}
\input{Secciones/Integral/Aplicaciones/Centroide.tex}

%\part*{Apéndices}
%\chapter*{Apéndice A: Aproximación numérica}
%\section{Cálculo de $\pi$ y $e$}
%\input{Secciones/Apendice/A/Raíces.tex}
%\input{Secciones/Apendice/A/Integral.tex}







\end{document}