\begin{exercise}
	Sea $S =\set{v_{1} =\mtrans{\begin{bmatrix}
			1 & 1 & 1
	\end{bmatrix}}, v_{2} =\mtrans{\begin{bmatrix}
	1 & 2 & -3
	\end{bmatrix}}, v_{3} =\mtrans{\begin{bmatrix}
	5 & -4 & -1
	\end{bmatrix}}}$ un subconjunto de $\rmath[3]$.
	\begin{enumerate}[a)]
		\item Probar que $S$ es un conjunto ortogonal y que es una base de $\rmath[3]$.
		\item Escribir $v =\mtrans{\begin{bmatrix}
				1 & 5 & -7
		\end{bmatrix}}$ como combinación de los vectores que pertenecen a $S$.
	\end{enumerate}
\end{exercise}
\begin{solution}
	\begin{enumerate}[a)]
		\item Comprobemos que $S$ es ortogonal.
		\begin{align*}
			\begin{bmatrix}
				1 \\ 1 \\ 1
			\end{bmatrix}
			\cdot
			\begin{bmatrix}
				1 \\ 2 \\ -3
			\end{bmatrix}
			&= 1 + 2 - 3 = 0\\
			\begin{bmatrix}
				1 \\ 1 \\ 1
			\end{bmatrix}
			\cdot
			\begin{bmatrix}
				5 \\ -4 \\ -1
			\end{bmatrix}
			&= 5 - 4 - 1 = 0\\
			\begin{bmatrix}
				1 \\ 2 \\ -3
			\end{bmatrix}
			\cdot
			\begin{bmatrix}
				5 \\ -4 \\ -1
			\end{bmatrix}
			&= 5 - 8 + 3 = 0
		\end{align*}
		Por lo tanto, $S$ es ortogonal, por consiguiente L.I. y entonces $S$ es base de $\rmath[3]$.
		\item Sabemos que la descomposición ortogonal de $v$ en $S$ es la suma de sus proyecciones:
		\begin{align*}
			\mproj{v}{v_{1}} &=\dfrac{
				\begin{bmatrix}
					1 \\ 5 \\ -7
				\end{bmatrix}
				\cdot
				\begin{bmatrix}
					1 \\ 1 \\ 1
				\end{bmatrix}
			}{\norm{
				\begin{bmatrix}
					1 \\ 1 \\ 1
				\end{bmatrix}
			}^{2}}
			\begin{bmatrix}
				1 \\ 1 \\ 1
			\end{bmatrix} 
			=\dfrac{1 + 5 - 7}{1 + 1 + 1}
			\begin{bmatrix}
				1 \\ 1 \\ 1
			\end{bmatrix}
		 	= -\dfrac{1}{3}
			\begin{bmatrix}
				1 \\ 1 \\ 1
			\end{bmatrix}\\
			\mproj{v}{v_{2}} &=\dfrac{
				\begin{bmatrix}
					1 \\ 5 \\ -7
				\end{bmatrix}
				\cdot
				\begin{bmatrix}
					1 \\ 2 \\ -3
				\end{bmatrix}
			}{\norm{
				\begin{bmatrix}
					1 \\ 2 \\ -3
				\end{bmatrix}
				}^{2}}
			\begin{bmatrix}
				1 \\ 2 \\ -3
			\end{bmatrix} 
			=\dfrac{1 + 10 + 21}{1 + 4 + 9}
			\begin{bmatrix}
				1 \\ 2 \\ -3
			\end{bmatrix} 
			= \dfrac{16}{7}
			\begin{bmatrix}
				1 \\ 2 \\ -3
			\end{bmatrix}\\
			\mproj{v}{v_{3}} &=\dfrac{
				\begin{bmatrix}
					1 \\ 5 \\ -7
				\end{bmatrix}
				\cdot
				\begin{bmatrix}
					5 \\ -4 \\ -1
				\end{bmatrix}
			}{\norm{
				\begin{bmatrix}
					5 \\ -4 \\ -1
				\end{bmatrix}
			}^{2}}
			\begin{bmatrix}
				5 \\ -4 \\ -1
			\end{bmatrix} 
			=\dfrac{5 - 20 + 7}{25 + 16 + 1}
			\begin{bmatrix}
				5 \\ -4 \\ -1
			\end{bmatrix} 
			= -\dfrac{4}{21}
			\begin{bmatrix}
				5 \\ -4 \\ -1
			\end{bmatrix}\\
			\begin{bmatrix}
				1 \\ 5 \\ -7
			\end{bmatrix}
			&=
			-\dfrac{1}{3}
			\begin{bmatrix}
				1 \\ 1 \\ 1
			\end{bmatrix}
			+\dfrac{16}{7}
			\begin{bmatrix}
				1 \\ 2 \\ -3
			\end{bmatrix}
			-\dfrac{4}{21}
			\begin{bmatrix}
				5 \\ -4 \\ -1
			\end{bmatrix}
		\end{align*}
	\end{enumerate}
\end{solution}