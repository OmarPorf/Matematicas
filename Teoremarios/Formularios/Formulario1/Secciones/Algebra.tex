\section{Álgebra}
\subsection{Álgebra básica}
\begin{itemize}
	\item $\ds(a\pm b)^{2} = a^{2} \pm 2ab + b^{2} = (a \mp b)^{2} \pm 4ab$
	\item $\ds(a + b)(a + c) = a^{2} + a(b + c) + bc$
	\item $\ds(a + b)(a - b) = a^{2} - b^{2}$
	\item $\ds2(a^{2} + b^{2}) = (a + b)^{2} + (a - b)^{2}$
	\item $\ds4ab = (a + b)^{2} - (a - b)^{2}$
	\item $\ds(a + b)^{n} =\sum_{k = 0}^{n}\binom{n}{k}a^{n - k}b^{k}$
	\item $\ds\mlog{1}[a] = 0$
	\item $\ds\mlog{a}[a] = 1$
	\item $\ds\mlog{xy}[a] =\mlog{x}[a] +\mlog{y}[a]$
	\item $\ds\mlog{\dfrac{x}{y}}[a] =\mlog{x}[a] -\mlog{y}[a]$
	\item $\ds\mlog{x^{n}}[a] = n\mlog{x}[a]$
	\item $\ds\mlog{b}[a] =\dfrac{1}{\mlog{a}[b]}$
	\item $\ds\mlog{a^{n}}[b^{n}] =\mlog{a}[b]$
	\item $\ds\mlog{a^{m}}[a^{n}] =\dfrac{m}{n}$
	\item $\ds\mlog{a}[b] =\dfrac{\mlog{a}[x]}{\mlog{b}[x]}$
\end{itemize}
\subsection{Sumas y series}
\begin{itemize}
	\item $\ds\sum_{k = 1}^{n}cf(k) = c\sum_{k = 1}^{n}f(k)$
	\item $\ds\sum_{k = 1}^{n}\pa{f(k) + g(k)} =\sum_{k = 1}^{n}f(k) +\sum_{k = 1}g(k)$
	\item $\ds\sum_{k = 1}^{n}f(k) =\sum_{k = 1}^{m}f(k) +\sum_{k = m + 1}^{n}f(k)$
	\item $\ds\sum_{k = 1}^{n}c =nc$
	\item $\ds\sum_{k = 1}k =\dfrac{n(n + 1)}{2}$
	\item $\ds\sum_{k = 1}k^{2} =\dfrac{n(n + 1)(2n + 1)}{6}$
	\item $\ds\sum_{k = 1}k^{3} =\pa{\dfrac{n(n + 1)}{2}}^{2}$
	\item $\ds\sum_{k = 0}^{n}r^{k} =\dfrac{1 - r^{n + 1}}{1 - r} ; r\neq 1$
	\item $\ds\sum_{k = a}^{n}\pa{f(k) - f(k + 1)} = f(a) - f(n + 1)$
	\item $\ds\sum_{n = 0}^{\infty}\dfrac{f^{(n)}(a)}{n!}(x - a)^{n}$
	\item $\ds e^{x} =\sum_{n = 0}^{\infty}\dfrac{x^{n}}{n!}$
	\item $\ds\msin{x} =\sum_{n = 0}^{\infty}(-1)^{n}\dfrac{x^{2n + 1}}{(2n + 1)!}$
	\item $\ds\mcos{x} =\sum_{n = 0}^{\infty}(-1)^{n}\dfrac{x^{2n}}{(2n)!}$
	\item $\ds\mln{1 + x} =\sum_{n = 1}^{\infty}(-1)^{n + 1}\dfrac{x^{n}}{n}$
\end{itemize}
\subsection{Números complejos}
\begin{itemize}
	\item $\ds i^{n} =
	\begin{cases}
		1, & n = 0\mod 4\\
		i, & n = 1\mod 4\\
		-1, & n = 2\mod 4\\
		-i, & n = 3\mod 4
	\end{cases}; n\geq 0
	$
	\item $\overline{z + w} =\overline{z} +\overline{w}$
	\item $z +\overline{z} = 2\rpart{z}$
	\item $z -\overline{z} = 2\ipart{z}i$
	\item $\overline{zw} =\overline{z}\cdot\overline{w}$
	\item $\abs{z} =\sqrt{\rpart{z}^{2} +\ipart{z}^{2}}$
	\item $z\overline{z} =\abs{z}^{2}$
	\item $\minv{z} =\dfrac{\overline{z}}{\abs{z}^{2}}$
	\item $\marg{z} =\marctan{\dfrac{\ipart{z}}{\rpart{z}}}$
	\item $\abs{z} = 0\iff z = 0$
	\item $\abs{z + w}\leq\abs{z} +\abs{w}$
	\item $\abs{\abs{z} -\abs{w}}\leq\abs{z - w}$
	\item $\abs{zw} =\abs{z}\abs{w}$
	\item $z =\abs{z}e^{i\marg{z}} =\abs{z}\pa{\mcos{\marg{z}} +i\msin{\marg{z}}}$
	\item Raíces de la unidad\\
	Sea $n\in\nmath$, entonces para cada $0\leq k < n$, la $k$-ésima ráiz de $1$ dado $n$ es:
	\[ \omega_{k} =\mexp{\dfrac{2\pi k}{n}i} \]
	\item $\ds\sqrt[n]{z} =\bigcup_{k = 0}^{n - 1}\set{\abs{z}\omega_{k}}; n\in\nmath$
\end{itemize}