\section{Paletas de colores}
\begin{figure}[H]
	    \centering
	    \begin{tikzpicture}
	        \draw[fill = pprainbow_r] (0, 0) rectangle (1, 2);
            \draw[fill = pprainbow_o] (1, 0) rectangle (2, 2);
            \draw[fill = pprainbow_y] (2, 0) rectangle (3, 2);
            \draw[fill = pprainbow_g] (3, 0) rectangle (4, 2);
            \draw[fill = pprainbow_b] (4, 0) rectangle (5, 2);
            \draw[fill = pprainbow_v] (5, 0) rectangle (6, 2);
            %\draw[fill = pprainbow_v] (6, 0) rectangle (7, 2);
	    \end{tikzpicture}
	    \caption{Paleta orgullo clásico}
	    \label{fig: pprainbow}
	\end{figure}
	\begin{figure}[H]
	    \centering
	    \begin{tikzpicture}
	        \draw[fill = pgay1] (0, 0) rectangle (1, 2);
            \draw[fill = pgay2] (1, 0) rectangle (2, 2);
            \draw[fill = pgay3] (2, 0) rectangle (3, 2);
            \draw[fill = pgay4] (3, 0) rectangle (4, 2);
            \draw[fill = pgay5] (4, 0) rectangle (5, 2);
            \draw[fill = pgay6] (5, 0) rectangle (6, 2);
	    \end{tikzpicture}
	    \caption{Paleta orgullo gay}
	    \label{fig: pgay}
	\end{figure}
	\begin{figure}[H]
	    \centering
	    \begin{tikzpicture}
	        \draw[fill = plesb1] (0, 0) rectangle (1, 2);
            \draw[fill = plesb2] (1, 0) rectangle (2, 2);
            \draw[fill = plesb3] (2, 0) rectangle (3, 2);
            \draw[fill = plesb4] (3, 0) rectangle (4, 2);
            \draw[fill = plesb5] (4, 0) rectangle (5, 2);
            \draw[fill = plesb6] (5, 0) rectangle (6, 2);
	    \end{tikzpicture}
	    \caption{Paleta orgullo lésbico}
	    \label{fig: plesb}
	\end{figure}
	\begin{figure}[H]
	    \centering
	    \begin{tikzpicture}
	        \draw[fill = ppan1] (0, 0) rectangle (1, 2);
            \draw[fill = ppan2] (1, 0) rectangle (2, 2);
            \draw[fill = ppan3] (2, 0) rectangle (3, 2);
	    \end{tikzpicture}
	    \caption{Paleta orgullo pansexual}
	    \label{fig: ppan}
	\end{figure}
	\begin{figure}[H]
	    \centering
	    \begin{tikzpicture}
	        \draw[fill = pbi1] (0, 0) rectangle (1, 2);
            \draw[fill = pbi2] (1, 0) rectangle (2, 2);
            \draw[fill = pbi3] (2, 0) rectangle (3, 2);
	    \end{tikzpicture}
	    \caption{Paleta orgullo bisexual}
	    \label{fig: pbi}
	\end{figure}
	\begin{figure}[H]
	    \centering
	    \begin{tikzpicture}
	        \draw[fill = ptrans1] (0, 0) rectangle (1, 2);
            \draw[fill = ptrans2] (1, 0) rectangle (2, 2);
	    \end{tikzpicture}
	    \caption{Paleta orgullo trans}
	    \label{fig: ptrans}
	\end{figure}
	\begin{figure}[H]
	    \centering
	    \begin{tikzpicture}
	        \draw[fill = pnb1] (0, 0) rectangle (1, 2);
            \draw[fill = pnb2] (1, 0) rectangle (2, 2);
            \draw[fill = pnb3] (2, 0) rectangle (3, 2);
	    \end{tikzpicture}
	    \caption{Paleta orgullo no binario}
	    \label{fig: pnb}
	\end{figure}
	\begin{figure}[H]
	    \centering
	    \begin{tikzpicture}
	        \draw[fill = pgenfluid1] (0, 0) rectangle (1, 2);
            \draw[fill = pgenfluid2] (1, 0) rectangle (2, 2);
            \draw[fill = pgenfluid3] (2, 0) rectangle (3, 2);
	    \end{tikzpicture}
	    \caption{Paleta orgullo genero fluido}
	    \label{fig: pgenfluid}
	\end{figure}
	\begin{figure}[H]
	    \centering
	    \begin{tikzpicture}
	        \draw[fill = pasex1] (0, 0) rectangle (1, 2);
            \draw[fill = pasex2] (1, 0) rectangle (2, 2);
	    \end{tikzpicture}
	    \caption{Paleta orgullo asexual}
	    \label{fig: pasex}
	\end{figure}
	\begin{figure}[H]
	    \centering
	    \begin{tikzpicture}
	        \draw[fill = parr1] (0, 0) rectangle (1, 2);
            \draw[fill = parr2] (1, 0) rectangle (2, 2);
            \draw[fill = parr3] (2, 0) rectangle (3, 2);
	    \end{tikzpicture}
	    \caption{Paleta orgullo arromántico}
	    \label{fig: parr}
	\end{figure}